%Pour inspiration lors de revamp: https://hackaday.com/2020/12/24/circuit-vr-some-op-amps/
%
\documentclass[canadien,12pt,oneside,letterpaper]{article}

\usepackage[utf8]{inputenc}
\usepackage[T1]{fontenc}
\usepackage{babel}
\usepackage{lmodern}
\usepackage{graphicx}
\usepackage[letterpaper]{geometry}
\usepackage[americanvoltages,americancurrents,siunitx]{circuitikz}
\usetikzlibrary{babel}
\usepackage{amsmath}
\usepackage{caption}
\usepackage{subfig}
\usepackage{hyperref}
\usepackage[all]{hypcap}


\captionsetup{font=small,labelfont=bf,margin=0.1\textwidth}
\pagestyle{myheadings}
\markboth{GPH-2006/PHY-2002~---~Amplificateurs}{GPH-2006/PHY-2002~---~Amplificateurs}


\begin{document}


\title{\textbf{Complément}\\Amplificateurs}
\author{Jean-Raphaël Carrier \& Claudine Allen}
\date{}
\maketitle


\section{Introduction}

Les amplificateurs sont des éléments actifs. La différence entre \textit{composants actifs} et \textit{composants passifs} est fondamentale. Les éléments passifs sont comparables à des objets inanimés. Prenons l'exemple d'une roche. Si vous lancez la roche dans les airs, elle va monter à une hauteur qui sera déterminée par la force avec laquelle vous la lancez. Mais la roche ne s'élèvera pas dans les airs toute seule. Les éléments actifs peuvent quant à eux être comparés à des êtres humains. Vous pouvez dire à un être humain de sauter dans les airs, et il va sauter dans les airs. Pour s'élever dans les airs, il utilise sa propre énergie. La seule condition pour qu'il effectue l'action, c'est que votre commande verbale ait assez d'énergie pour qu'il vous entende. Cette analogie (étrange) étant terminée, retournons au merveilleux monde de l'électronique.

Les amplificateurs sont des éléments actifs et peuvent donc fournir une puissance de sortie supérieure à la puissance d'entrée. De là provient le terme \textit{amplificateur}, même s'il n'y a pas de réelle amplification, ce qui irait à l'encontre du principe de conservation de l'énergie.

\begin{figure}[h]
\begin{center}
\begin{circuitikz} \draw
(0,0) to[short] (2,0) to[short] (1.8,0.2)
(2,0) to[short] (1.8,-0.2)
(3,-0.5) to[short] (6,-0.5) to[short] (6,0.5) to[short] (3,0.5) to[short] (3,-0.5)
(7,0) to[short] (9,0) to[short] (8.8,0.2)
(9,0) to[short] (8.8,-0.2)
(3,1.5) to[short] (6,1.5) to[short] (6,2.5) to[short] (3,2.5) to[short] (3,1.5)
(4.5,1.3) to[short] (4.5,0.7) to[short] (4.7,0.9)
(4.5,0.7) to[short] (4.3,0.9)
(1,0) node[above]{$p_{\mathrm{in}}$}
(8,0) node[above]{$P_{\mathrm{out}}$}
(4.5,-0.3) node[above]{amplificateur}
(4.5,1.7) node[above]{alimentation}
;\end{circuitikz}
\end{center}
\caption{\label{sch-amplif}Schématisation d'un amplificateur. La présence d'une source d'alimentation permet un gain en puissance $p_{\mathrm{out}}/p_{\mathrm{in}}$ supérieur à $1$.}
\end{figure}


\section{Amplificateur différentiel}

Il existe différentes sortes d'amplificateurs, le plus simple étant un transistor. Dans le cas d'un transistor bipolaire, le courant circulant dans la base est l'entrée (signal faible) et le courant circulant du collecteur à l'émetteur constitue le signal de sortie (signal amplifié).

Un amplificateur différentiel est un circuit légèrement plus complexe, utilisant quelques résistances et une paire de transistors, un de ces transistors étant utilisé en mode inverseur et l'autre en mode non inverseur. L'amplificateur différentiel comporte deux entrées et une sortie ; cette dernière est proportionnelle à la différence de potentiel entre les deux entrées, mais est beaucoup plus grande dû au gain important. Ce type de circuit est toutefois «trop simple» et comporte plusieurs défauts, dont un manque de stabilité ; le comportement de l'amplificateur différentiel est donc approximatif et difficilement prévisible.


\section{Fonctionnement interne d'un ampli-op}

En complexifiant l'amplificateur différentiel, il est possible d'obtenir un amplificateur opérationnel, couramment appelé ampli-op. Ce dernier est très stable et prévisible. Les principales caractéristiques d'un ampli-op sont:
\begin{enumerate}
\item une très grande impédance d'entrée;
\item une très faible impédance de sortie;
\item un très grand gain en puissance.
\end{enumerate}
La troisième caractéristique est une conséquence des deux premières. La très haute impédance d'entrée impose que le signal entrant dans l'ampli-op (la commande) soit extrêmement faible. À l'opposé, la très faible impédance à la sortie fait en sorte que le signal de sortie peut être aussi grand que l'alimentation le permet.

Un amplificateur opérationnel idéal a un gain infini. Pour qu'il se comporte de façon contrôlée et prévisible, il faut donc nécessairement inclure une rétroaction dans le circuit. La rétroaction sera discutée plus loin.

Le schéma d'un amplificateur opérationnel étant relativement complexe\footnote{L'amplificateur opérationnel le plus couramment utilisé, le UA741, est formé de vingt transistors, d'onze résistances et d'un condensateur.}, et puisqu'il est très souvent utilisé, ce circuit est par simplicité représenté par un triangle (image ci-après) avec deux entrées et une sortie.

\begin{center}
\begin{circuitikz} \draw
(0,0) node[op amp]{}
;\end{circuitikz}
\end{center}

Pour respecter le principe de conservation de l'énergie, un amplificateur doit être relié à une source d'alimentation (DC) pour amplifier un signal. Néanmoins, cette connexion est très rarement explicitée dans le schéma d'un circuit. Lorsque c'est le cas, deux autres broches sont ajoutées au «triangle».

\begin{center}
\begin{circuitikz} \draw
(0,0) node[op amp](opamp){}
(opamp.+) node[left]{$v_{\mathrm{in}+}$}
(opamp.-) node[left]{$v_{\mathrm{in}-}$}
(opamp.out) node[right]{$v_{\mathrm{out}}$}
(opamp.down) ++ (0,-0.5) node[below]{$v_{\mathrm{s}-}$} -- (opamp.down)
(opamp.up) ++ (0,0.5) node[above] {$v_{\mathrm{s}+}$} -- (opamp.up)
;\end{circuitikz}
\end{center}

Les entrées -- et + correspondent respectivement aux entrées inverseuse et non inverseuse. La sortie sera positive si l'entrée non inverseuse (+) est plus grande positivement que l'entrée inverseuse (--) et \textit{vice versa}. Attention : les entrées n'ont PAS nécessairement une polarité opposée ; il ne s'agit PAS forcément d'une entrée positive et d'une entrée négative.


\section{Fonctionnement externe d'un ampli-op}

Étant donné qu'un amplificateur opérationnel amplifie considérablement le signal d'entrée, la moindre différence de potentiel à ses entrées provoquerait un signal de sortie maximal, ce qui ne serait pas très pratique. Un amplificateur opérationnel tout nu ne sert donc à rien. Par contre, en ajoutant quelques composants extérieurs ainsi qu'une rétroaction, les possibilités sont infinies! (ou presque)

La rétroaction (\textit{feedback}) est le principe de base de tous les systèmes de contrôle et circuits d'asservissement, c'est-à-dire partout où le système va s'ajuster en fonction du résultat. Ceci permet de stabiliser le circuit et d'obtenir l'effet désiré.

Les trois caractéristiques inhérentes aux amplificateurs opérationnels permettent de faire deux approximations qui facilitent la compréhension des circuits contenant un ampli-op et une rétroaction.
\begin{enumerate}
\item \textbf{Les deux entrées d'un ampli-op sont au même potentiel.} Avec une rétroaction, la sortie de l'ampli-op se comporte de façon à éliminer la différence de potentiel entre les deux entrées. De toute manière, le gain en tension dans un ampli-op est tellement grand que la différence de potentiel à l'entrée est négligeable vis-à-vis la tension de sortie.
\item \textbf{Le courant à l'entrée d'un ampli-op est nul.} En réalité, il existe un courant d'entrée non nul ; c'est d'ailleurs ce courant qui commande le signal à la sortie. Mais ce courant est si faible à cause de la très grande impédance d'entrée qu'il est négligeable pour les fins de l'analyse du circuit. Pour le UA741, il s'agit d'un courant d'entrée d'approximativement 80~nA. Le courant est réduit à quelques pA lorsque des transistors à effet de champ (FET) sont utilisés dans l'ampli-op au lieu de transistors bipolaires (BJT).
\end{enumerate}

Dans tous les circuits contenant un ampli-op, la réponse de ce dernier dépend uniquement des composants externes et de la façon dont il leur est relié, et non de l'\textit{intérieur} de l'ampli-op.


\section{Exemples de circuits avec ampli-op}

Les amplificateurs opérationnels peuvent être nombreux dans les circuits de la vie courante. Pour faciliter la compréhension de leur comportement, on se limitera ici à des circuits ne contenant qu'un seul ampli-op.


\subsection{Amplificateur inverseur}

Soit le circuit de la figure~\ref{ampli-inv}.

\begin{figure}[h]
\begin{center}
\begin{circuitikz} \draw
(0,0) node[op amp](opamp){}
(opamp.+) to[short] (-1.2,-0.5) to[short,*-] (-1.5,-0.5) node[ground]{}
(opamp.-) to[short] (-1.2,0.5) to[R,l_=$R_1$,*-] (-4,0.5)
(-1.2,0.5) to[short] (-1.2,1.5) to[R=$R_2$] (1.2,1.5) to[short] (1.2,0)
(opamp.out) to[short] (2,0)
{[anchor=west] (2,0) node{$v_{\mathrm{out}}$}}
{[anchor=east] (-4,0.5) node{$v_{\mathrm{in}}$}}
{[anchor=south east] (-1.2,-0.5) node{b} (-1.2,0.5) node{a}}
;\end{circuitikz}
\end{center}
\caption{\label{ampli-inv}Circuit d'un amplificateur inverseur.}
\end{figure}

L'analyse de ce circuit se fait en utilisant les deux approximations mentionnées précédemment. Tout d'abord, puisque le point \textbf{b} est relié à la terre (potentiel nul), il en est de même pour le point \textbf{a}. Le point \textbf{a} est alors considéré comme une mise à la terre virtuelle (\textit{virtual ground}). Puisque \textbf{a} et \textbf{b} sont au même potentiel et que le courant circulant dans l'ampli-op est négligeable, le circuit de l'amplificateur inverseur (figure~\ref{ampli-inv}) se simplifie au schéma suivant (figure~\ref{ampli-inv-simple}).

\begin{figure}[h]
\begin{center}
\begin{circuitikz} \draw
(0,0) node[left]{$v_{\mathrm{in}}$} to[R=$R_1$] (3,0) node[ground]{} to[R=$R_2$] (6,0) node[right]{$v_{\mathrm{out}}$}
;\end{circuitikz}
\end{center}
\caption{\label{ampli-inv-simple}Circuit simplifié équivalent à l'amplificateur inverseur.}
\end{figure}

Si on pose arbitrairement que le courant va de droite à gauche, alors la tension dans la résistance $R_2$ est $v_{\mathrm{out}}$ et la tension dans la résistance $R_1$ est $-v_{\mathrm{in}}$. Puisque le courant dans les deux résistances est le même, on a:
\begin{equation}
\frac{v_{\mathrm{out}}}{R_2}=\frac{-v_{\mathrm{in}}}{R_1}.
\end{equation}
On peut ensuite calculer le gain en tension, défini comme le rapport de $v_{\mathrm{out}}$ et $v_{\mathrm{in}}$:
\begin{equation}
gain\equiv\frac{v_{\mathrm{out}}}{v_{\mathrm{in}}}=\frac{-R_2}{R_1}.
\end{equation}
Le signe négatif implique que le gain de l'amplificateur est inversé.

Ce circuit plutôt simple a une impédance d'entrée relativement faible, ce qui est un grand inconvénient. Le circuit suivant (figure~\ref{ampli-non-inv}) permet de régler ce problème.


\subsection{Amplificateur non inverseur}

\begin{figure}[h]
\begin{center}
\begin{circuitikz} \draw
(0,0) node[op amp](opamp){}
(opamp.+) node[left]{$v_{\mathrm{in}}$}
(opamp.-) to[short] (-1.2,0.5) to[R,l_=$R_1$,*-] (-4,0.5) node[ground]{}
(-1.2,0.5) to[short] (-1.2,1.5) to[R=$R_2$] (1.2,1.5) to[short] (1.2,0)
(opamp.out) to[short] (2,0)
{[anchor=west] (2,0) node{$v_{\mathrm{out}}$}}
{[anchor=north east] (-1.2,0.5) node{a}}
;\end{circuitikz}
\end{center}
\caption{\label{ampli-non-inv}Circuit d'un amplificateur non inverseur.}
\end{figure}

L'analyse du circuit de l'amplificateur non inverseur n'est pas beaucoup plus compliquée que la précédente. De la première approximation, on sait que la tension au point \textbf{a} équivaut à la tension d'entrée $v_{\mathrm{in}}$. De plus, les résistances $R_1$ et $R_2$ divisent la tension entre $v_{\mathrm{out}}$ et la mise à la terre. L'équation suivante peut alors être établie:
\begin{equation}
v_{\mathrm{in}}=v_a=\frac{R_1}{R_1+R_2}\,v_{\mathrm{out}}.
\end{equation}
Par la suite, on peut isoler le gain:
\begin{equation}
gain\equiv\frac{v_{\mathrm{out}}}{v_{\mathrm{in}}}=1+\frac{R_2}{R_1}.
\end{equation}

Ce circuit a l'avantage d'avoir une impédance d'entrée très élevée et une impédance de sortie très basse.

Dans le cas particulier où $R_2=0$ et $R_1\rightarrow\infty$, le gain devient unitaire. Ce circuit s'appelle un \textit{suiveur de tension}.


\subsection{Amplificateur différentiel}

\begin{figure}[h]
\begin{center}
\begin{circuitikz} \draw
(0,0) node[op amp](opamp){}
(opamp.+) to[short] (-1.2,-0.5) to[short] (-1.2,0) to[R,l_=$R_1$] (-4,0) node[left]{$v_2$}
(-1.2,-0.5) to[short] (-1.2,-1) to[R=$R_2$] (-4,-1) node[ground]{}
(opamp.-) to[short] (-1.2,0.5) to[short] (-1.2,1.5)
(-4,1.5) node[left]{$v_1$} to[R=$R_1$] (-1.2,1.5) to[R=$R_2$] (1.2,1.5) to[short] (1.2,0)
(opamp.out) to[short] (2,0)
{[anchor=west] (2,0) node{$v_{\mathrm{out}}$}}
;\end{circuitikz}
\end{center}
\caption{\label{ampli-diff}Circuit d'un amplificateur différentiel.}
\end{figure}

Comme il a été dit, l'amplificateur opérationnel tout nu amplifie énormément la différence de tension à ses entrées. Toutefois, il existe une façon d'amplifier la différence de tension aux entrées avec un gain choisi : ce circuit s'appelle aussi un amplificateur différentiel (figure~\ref{ampli-diff}). Ce gain différentiel est égal au rapport des résistances $R_2$ et $R_1$.
\begin{gather}
gain=\frac{R_2}{R_1}\\
v_{\mathrm{out}}=\frac{R_2}{R_1}\left(v_2-v_1\right)
\end{gather}


\subsection{Additionneur (sommateur)}

Soit un circuit constitué de trois branches en parallèle, chacune comprenant une source de tension et une résistance en série, tel qu'illustré à la figure~\ref{moy}. Le théorème de Millman, qui est une forme particulière de la loi des n{\oe}uds, stipule que la tension aux bornes des branches est égale à la somme des courants dans chacune, divisée par la somme des admittances.
\begin{equation}
\label{millman}
\Delta V=V_{\mathrm{out}}=\frac{\sum I_k}{\sum Y_k}=\frac{\sum \frac{V_k}{Z_k}}{\sum \frac{1}{Z_k}}
\end{equation}
En appliquant l'équation~\ref{millman} au circuit de la figure~\ref{moy}, on obtient:
\begin{equation}
v_{\mathrm{out}}=\frac{\frac{v_1}{R_1}+\frac{v_2}{R_2}+\frac{v_3}{R_3}}{\frac{1}{R_1}+\frac{1}{R_2}+\frac{1}{R_3}}.
\end{equation}

\begin{figure}[h]
\begin{center}
\begin{circuitikz} \draw
(-2,1.5) to(-2,3) 
(0,3) to[V, l_=$v_1$] (-2,3)
(0,3) to[R=$R_1$] 
(3,3) to[short] (3,1.5)
(-3,1.5) node[ground]{}
(-3,1.5) to(-2,1.5)
(0,1.5) to[V, l_=$v_2$] (-2,1.5)
(0,1.5) to[R=$R_2$] 
(3,1.5) to[short,-o] 
(4,1.5) node[right]{$v_{\mathrm{out}}$}
(-2,1.5) to(-2,0)
(0,0) to[V, l_=$v_3$] (-2,0)
(0,0) to[R=$R_3$] 
(3,0) to[short] (3,1.5)
;\end{circuitikz}
\end{center}
\caption{\label{moy}Application du théorème de Millman.}
\end{figure}

Si les trois résistances sont identiques ($R_1=R_2=R_3$), la tension aux bornes des branches devient égale à la moyenne des trois sources de tensions.
\begin{equation}
v_{\mathrm{out}}=\frac{v_1+v_2+v_3}{3}
\end{equation}
On parle alors d'un \textit{moyenneur passif}, puisqu'il est constitué uniquement de composants passifs.

En plaçant un tel circuit à l'entrée d'un amplificateur non inverseur dont le gain est de 3, le circuit devient celui d'un additionneur, avec $v_{\mathrm{out}}=v_1+v_2+v_3$, comme le montre le schéma de la figure~\ref{ampli-add}, avec $R_B=2\,R_A$ (pour avoir un gain de 3).

\begin{figure}[h]
\begin{center}
\begin{circuitikz} \draw
(0,0) node[op amp](opamp){}
(opamp.+) to[short] (-1.2,-0.5)
(-4,0) node[left]{$v_1$} to[R=$R$] (-1.5,0) to[short] (-1.5,-0.5) to[short] (-1.2,-0.5)
(-4,-1) node[left]{$v_2$} to[R=$R$] (-1.5,-1)
(-4,-2) node[left]{$v_3$} to[R=$R$] (-1.5,-2) to[short] (-1.5,-0.5)
(opamp.-) to[short] (-1.2,0.5) to[short] (-1.2,1.5)
(-4,1.5) node[ground]{} to[R=$R_A$] (-1.2,1.5) to[R=$R_B$] (1.2,1.5) to[short] (1.2,0)
(opamp.out) to[short] (2,0)
{[anchor=west] (2,0) node{$v_{\mathrm{out}}$}}
;\end{circuitikz}
\end{center}
\caption{\label{ampli-add}Circuit d'un additionneur.}
\end{figure}

Le sommateur est très polyvalent puisque l'amplification peut être ajustée simplement en changeant le rapport des résistances $R_A$ et $R_B$.


\subsection{Régulateur de tension}

Il a été vu dans le complément \textit{Composants non linéaires} qu'une diode Zener pouvait être utilisée pour limiter la tension. On parle alors d'un circuit régulateur de tension. La figure~\ref{reg-1} en montre un exemple. Il s'agit en fait d'un diviseur de tension où la tension aux bornes de la diode Zener, parcourue d'un courant inverse, est quasi constante. Ainsi, peu importe la charge et peu importe la puissance fournie par la source\footnote{Le courant doit toutefois être continu et non alternatif, sans quoi la diode Zener ne serait pas en tout temps utilisée en sens inverse, ce qui est nécessaire pour obtenir une tension constante.}, la tension aux bornes de la charge sera toujours égale à la tension de Zener de la diode.

\begin{figure}[h]
\begin{center}
\begin{circuitikz} \draw
(3,3) to[short] (4,3)
(0,3) to[R=$R_1$] (3,3)
(3,0) node[ground]{} to[zD] (3,3)
{[anchor=east] (0,3) node{$v_{\mathrm{in}}$}}
{[anchor=west] (4,3) node{$v_{\mathrm{out}}$}}
;\end{circuitikz}
\end{center}
\caption{\label{reg-1}Circuit régulateur de tension simple.}
\end{figure}

Ce circuit plutôt simpliste comporte plusieurs inconvénients. D'une part, il est très limité puisque la tension de sortie ne peut pas excéder la tension de Zener de la diode ; ainsi, il n'est possible de stabiliser que les circuits de faible puissance. D'autre part, si la charge est retirée, le courant traversant la diode pourrait dépasser la valeur maximale admissible par cette dernière, qui pourrait alors être endommagée. De plus, ce circuit demeure sensible aux variations de température et même de la source. En fait, ce circuit est plutôt un stabilisateur de tension et non un régulateur très précis.

Il est possible toutefois de réaliser un meilleur régulateur de tension en complexifiant le circuit précédent avec l'ajout d'un amplificateur opérationnel (figure~\ref{reg-2}). La rétroaction permet d'ajuster l'amplificateur opérationnel afin que la tension de sortie demeure en tout temps constante, indépendamment des variations de la source, de la charge ou même de la température. De plus, la résistance variable ($R_2$) permet quant à elle de varier la tension de sortie ; le circuit est un régulateur de tension variable. La tension de sortie peut ainsi être fixée à n'importe quelle valeur comprise entre $v_Z$, la tension de Zener de la diode utilisée, et $v_{\mathrm{in}}$, la tension d'entrée.

\begin{figure}[h]
\begin{center}
\begin{circuitikz} \draw
(8,7) to[short] (9,7)
(0,7) to[short] (1,7) to[R=$R_V$] (1,4) to[short] (2.8,4)
(1,0) node[ground]{} to[short] (8,0)
(8,7) to[Tnpn,n=npn] (4,7) to[short] (1,7)
(1,0) to[zD] (1,4)
(8,0) to[R,l_=$R_3$] (8,2) to[pR,l_=$R_2$,n=R2] (8,4) to[R,l_=$R_1$] (8,7)
(npn.B) to[short] (6,4.5)
(4,4.5) node[op amp](opamp){}
(opamp.+) to[short] (2.8,4)
(opamp.-) to[short] (2.8,5)
(opamp.out) to[short] (6,4.5);
\draw[dashed] (R2.wiper) -| (opamp.-)
{[anchor=east] (0,7) node{$v_{\mathrm{in}}$}}
{[anchor=west] (9,7) node{$v_{\mathrm{out}}$}}
;\end{circuitikz}
\end{center}
\caption{\label{reg-2}Circuit de régulation de tension amélioré.}
\end{figure}


\subsection{Convertisseur courant à tension (transimpédance)}

L'amplificateur à transimpédance est un circuit qui permet de convertir un courant en tension, et ce sans affecter la valeur de ce courant. Il est notamment utilisé dans les cellules solaires (piles photovoltaïques) et les détecteurs optiques en général pour produire une différence de potentiel à partir du photocourant généré par les photodiodes. Dans sa forme illustrée à la figure~\ref{ampli-trans}, l'amplificateur à transimpédance est équivalent à un amplificateur inverseur (figure~\ref{ampli-inv}) avec $R_1=0$. La tension de sortie est alors donnée par:
\begin{equation}
v_{\mathrm{out}}=-i_{\mathrm{in}}\,R.
\end{equation}

Dans sa forme plus générale, la résistance $R$ est remplacée par une résistance en parallèle avec un condensateur.

\begin{figure}[h]
\begin{center}
\begin{circuitikz} \draw
(0,0) node[op amp](opamp){}
(opamp.+) to[short] (-1.2,-0.5) to[short] (-1.5,-0.5) node[ground]{}
(opamp.-) to[short] (-1.2,0.5)
(-3,0.5) to[short,i^=$i_{\mathrm{in}}$] (-1.2,0.5)
(-1.2,0.5) to[short] (-1.2,1.5) to[R=$R$] (1.2,1.5) to[short] (1.2,0)
(opamp.out) to[short] (2,0)
{[anchor=west] (2,0) node{$v_{\mathrm{out}}$}}
;\end{circuitikz}
\end{center}
\caption{\label{ampli-trans}Circuit d'un amplificateur à transimpédance pour convertir un courant ($i_{\mathrm{in}}$) en tension ($v_{\mathrm{out}}$).}
\end{figure}


\subsection{Source de courant}

\begin{figure}[h]
\begin{center}
\begin{circuitikz} \draw
(0,0) node[op amp](opamp){}
(opamp.+) to[short] 
(-1.2,-0.5) to[short] 
(-1.2,-1.5) to[short] (-4,-1.5)
(opamp.-) to[short] (-1.2,0.5)
(opamp.out) to[short] 
(1.2,0) to[short] 
(1.2,1.5) to[european resistor,l_=charge~quelconque] 
(-1.2,1.5) to[short] 
(-1.2,0.5) to[R,l_=$R$] (-4,0.5)
(-4, -1.5) to[V,l_=$v_{\textrm{in}}$] (-4,0.5)
(-4,-1.5) node[ground]{}
;\end{circuitikz}
\end{center}
\caption{\label{sch-source-courant}Circuit pour envoyer un courant constant à une charge quelconque.}
\end{figure}

Les amplificateurs opérationnels peuvent aussi être utilisés afin de créer une source de courant constant à partir d'une tension fixe. La figure~\ref{sch-source-courant} représente une des façons de procéder afin d'alimenter avec un courant constant un circuit quelconque.

La valeur du courant fourni est donnée par:
\begin{equation}
i=\frac{v_{\mathrm{in}}}{R}.
\end{equation}

Pour ce faire, cependant, la tension d'alimentation de l'ampli-op doit être suffisamment élevée relativement à l'impédance du circuit quelconque, autrement la valeur réelle du courant fourni sera inférieure à celle escomptée. Notez aussi que la source de tension $v_{\mathrm{in}}$ doit être placée à l'envers, afin d'imposer une tension négative.


\end{document}

Écrit par Jean-Raphaël Carrier
Dernière modification : 11 janvier 2014

