%Écrit par Jean-Raphaël Carrier en collaboration avec Claudine Allen
%Dernière modification JRC: 19 mars 2014
%Élimination du labo de résistivité des matériaux (IV) à la fin de l'ère JRC + regroupement des labos VI & VII en début de pandémie COVID-19 => renumérotation VII & VIII -> VI maintenant
%Dernière modification CA: 22 janvier 2023
%
%ToDo : 
% [] Les Q1 et Q2 (faire un check expérimental de cette dernière d'ailleurs) d'analyse reposent au fond sur des notions de filtrage qui ne sont pas encore couvertes, alors s'assurer que cela soit adapté ou autrement organisé pour servir d'introduction-segway au cours 7 quitte à s'inspirer de la source originale en GEL qui ne devait pas toucher aux réponses en fréquence à ce point-là.
% [] Partie 3 gardée optionnelle, GARDER UN OEIL SI LE TEMPS DU LABO PERMET DE RAMENER OBLIGATOIRE
% [] ayant inversé les parties 4 et 5, vérifier que les instructions de branchement d'alimentation, surtout avec le COM sont bien détaillées plus en 4 qu'en 5.
% [] Pas mal de texte ici qui devrait aller en compléments, particulièrement des objectifs et du début des manip d'ampli-op->circuits intégrés et sans oublier beaucoup de stock de la Q4 d'analyse! D'ailleurs enlever sa division en a) et b) lors de cette révision.
%%%

\RequirePackage[l2tabu, orthodox]{nag} %Check for obsolete commands
\documentclass[canadien,12pt,oneside,letterpaper]{article}
%
%-----------------------------------------------------
%Loading packages
%
\usepackage[utf8]{inputenc}
\usepackage[T1]{fontenc}
\usepackage{babel}
\usepackage{lmodern}
\usepackage{textcomp}
\usepackage{amsmath,amssymb}
\usepackage{siunitx}
\usepackage{xcolor}
\PassOptionsToPackage{hyphens}{url}\usepackage[colorlinks=true,allcolors=blue]{hyperref}
\usepackage[all]{hypcap}
\usepackage{graphicx}
\usepackage[oldvoltagedirection,americanvoltages,americancurrents,siunitx]{circuitikz}
\usetikzlibrary{babel}
\usepackage{caption}
\usepackage[letterpaper,headheight=15pt]{geometry}
\usepackage{fancyhdr}
\usepackage{setspace}
%
%----------------------------------------------------
%Other configurations and layout
%
\sisetup{separate-uncertainty}
\captionsetup{font=small,labelfont=bf,margin=0.1\textwidth}
\pagestyle{fancy}
\fancyhf{}
\lhead{\textsl{GPH-2006/PHY-2002~---~Laboratoire~VI}}
\rhead{\textsl{Page \thepage}}
\setcounter{secnumdepth}{0}
\setlength{\parskip}{1.5ex plus0.5ex minus0.2ex}
%\onehalfspacing
\interfootnotelinepenalty=10000 %To avoid footnotes spreading on several pages.
%
%---------------------------------------------------
%
\title{\textbf{Laboratoire VI}\\Circuits non linéaires \& intégrés:\\rectification \& amplification\thanks{Auteurs: Claudine Allen \& Jean-Raphaël Carrier}}
\renewcommand\footnotemark{}
\date{}


\begin{document}

\maketitle \vspace{-2cm}

\section{Objectifs}

Le laboratoire revient aux composants non linéaires pour explorer de nouvelles fonctions qu'ils peuvent accomplir avec des circuits passifs et actifs, \textit{i.e.} sans et avec branchements d'alimentation respectivement. La polarisation des diodes permettra le redressement d’un courant alternatif alors que l'alimentation du transistor aux broches collecteur-émetteur permettra l'amplification du signal à sa borne d'entrée, soit sa base. Ensuite, les manipulations focaliseront sur les principes de l’amplificateur opérationnel et ses fonctionnalités versatiles pour de nombreuses applications électroniques analogues. À la base, ce circuit amplifie les différences de signal entre ses entrées inverseuse et non inverseuse, ce qui sera d'abord exploré avec la simulation d’un amplificateur différentiel avant de passer au circuit miniature et complexe caché dans une puce électronique contenant l'amplificateur opérationnel intégré qui pousse le gain d'amplification vers l'infini. Ce mode d’opération \textit{saturé} en boucle ouverte sera utilisé dans la fonction de comparateur avant de poursuivre avec le mode d’amplification linéaire avec rétroaction. Notons finalement que le laboratoire permettra d'initier l'étudiant.e à la manipulation de circuits intégrés dans les puces électroniques alimentées.

%Version A20:
% Le laboratoire revient aux composants non linéaires pour explorer de nouvelles fonctions qu'ils peuvent accomplir avec des circuits passifs et actifs, \textit{i.e.} sans et avec alimentation respectivement. La polarisation des diodes permettra le redressement d’un courant alternatif alors que l'alimentation du transistor aux broches collecteur-émetteur permettra l'amplification du signal à son entrée, soit sa base. Ensuite, les manipulations focaliseront sur les principes de l’amplificateur opérationnel et sa fonctionnalité très versatile, utilisée pour de nombreuses applications électroniques. L’idée générale se base sur des différences de signal qui sont amplifiées. Ce concept sera étudié en comparant la simulation d’un amplificateur différentiel de base avec un amplificateur opérationnel intégré pour comparer leur performance en suiveur de tension. Le gain en puissance d’un amplificateur opérationnel idéal est infini, d’où la tension de sortie sature à la valeur maximale fournie par la source d’alimentation à la moindre différence de tension entre les entrées du composant. Ce mode d’opération \textit{saturé} en boucle ouverte sera utilisé dans la fonction de comparateur avant de poursuivre avec le mode d’amplification linéaire avec rétroaction. Tout l’intérêt de ce mode d’amplification vient du gain qui est alors déterminé et stabilisé par des composants linéaires externes et non par les spécifications internes variables des amplificateurs opérationnels de différents manufacturiers. Ce mode d’opération linéaire en boucle fermée sera exploré avec des applications en régulation de tension, contrôle ou asservissement. Finalement, de tels circuits actifs qui deviennent de plus en plus complexes sont généralement miniaturisés sous forme de puces électroniques alimentées; le laboratoire initiera donc l'étudiant.e à la manipulation de ces circuits intégrés.
%(RAMENER LES FONCTIONS-APPLICATION EN CONTRÔLE-ASSERVISSEMENT, ET P-E RÉGULATION DE TENSION SELON LE LIEN AVEC LE DÉBUT DE FONCTIONNALITÉ DU LAB III, ET QU'ON METTRA AU FINAL DANS LE PROTOCOLE. PLUSIEURS ÉLÉMENTS COUPÉS EN H21 DEVRAIENT PLUTÔT ALLER EN COMPLÉMENTS.)
% et les phrases qui en avaient déjà été coupées:

%La trame de fond de ce laboratoire est de préparer l’étudiant.e à l’examen de laboratoire en appliquant l’adage “C’est en forgeant qu’on devient forgeron” à la réalisation de circuits électriques. : POSSIBLEMENT RAMENER LORSQU'ON REVIENDRA EN PRÉSENTIEL AVEC UN LABO AVANT L'EXAMEN.

%L’étudiant.e sera aussi initié.e à une nouvelle méthode de mesure plus juste grâce à des ponts: ce type de circuit linéaire se rapproche de l’instrument idéal en minimisant le courant dans la branche de mesure, tendant alors vers un circuit ouvert qui ne perturbe pratiquement plus. Une fois le pont équilibré (\textit{balancé}) à zéro courant en indiquant l’égalité des tensions, la résolution de mesure est essentiellement déterminée par le choix de l’étalon de référence.

%L’amplificateur opérationnel tire d’ailleurs son nom des opérations mathématiques réalisées en circuits analogiques à l’aube de l’informatique.

\section[Lectures préparatoires]{Lectures préparatoires} \label{sec:prep}

\begin{itemize} \itemsep4pt
\item le complément \textit{Circuits de base} sauf la section 2;
%\item le complément \textit{Transformateur};
\item le complément \textit{Amplificateurs} sauf les sections 5.4 et 5.5;
\item le protocole de ce laboratoire.
\end{itemize}
\vspace{1ex}
\noindent\framebox{\parbox{\dimexpr\linewidth-2\fboxsep-2\fboxrule}{En plus de ces lectures, n'hésitez pas à noter vos réflexions et avancer votre préparation dans le cahier de recherche.}}

\section{Matériel}

\noindent La réalisation de ce laboratoire requiert l'utilisation de:
\vspace{1ex}
\begin{itemize} \itemsep4pt
\item un multimètre à 4\textonehalf~chiffres;
\item un multimètre à 6\textonehalf~chiffres;
\item un bloc d'alimentation;
\item un oscilloscope avec générateur de fonctions;
\item un transformateur avec prise médiane;
\item un moteur DC basse tension HC385G-302;
%\item résistances de 12~$\Omega$, 1~k$\Omega$ (deux fois), 100~k$\Omega$ (deux fois) et 1~M$\Omega$;
\item un gros paquet de résistances; %(une de 270~$\Omega$, une de 560~$\Omega$, quatre de 1~k$\Omega$, quatre de 10~k$\Omega$ et deux de 100~k$\Omega$);
%\item deux potentiomètres linéaires de 1~k$\Omega$ (code 102);
%\item un condensateur de 1~$\mu\mathrm{F}\pm20$~\% et un condensateur de précision de $1~\mu\mathrm{F}\pm5$~\%;
%\item une bobine de 1~mH;
\item quatre diodes standards 914;
\item une diode électroluminescente (LED);
\item un transistor 2222A;
\item un régulateur de tension intégré;
\item un amplificateur opérationnel intégré (puce 741), voir son point de couleur qui l'identifie dans la liste d'inventaire des composants \& instruments;
\item une plaquette de montage.
\end{itemize}


\section{Manipulations}

\setlength{\parskip}{1ex plus 0.5ex minus 0.2ex}

\subsection{Partie 1 --- Redressement d'un courant alternatif}

Dans cette partie, vous convertirez le signal alternatif d'Hydro-Québec en un signal continu quasi constant. Votre but est de faire tourner un moteur qui ne tourne pas lorsqu'il est alimenté par une tension alternative.

Le transformateur que vous utiliserez fournit une tension efficace (RMS) allant d'environ $-8$~V à 8~V, oscillant à la même fréquence que la tension fournie à son entrée. Par mesure de sécurité, ne mettez le transformateur sous tension que lorsque votre circuit est monté. Avant de le modifier, ouvrez toujours l'interrupteur du transformateur. Et faites attention si vous laissez le circuit alimenté pendant une longue période de temps ; certains composants peuvent devenir assez chauds, notamment les diodes!

a) Faites le circuit de la figure~\ref{sch-init}. Notez vos observations. Puisque la tension est alternative, le petit moteur devrait osciller autour d'un même point, au lieu d'effectuer des rotations complètes. Ensuite, remplacez le moteur par une résistance de 1~M$\Omega$.

\begin{figure}[h]
\centering
\begin{circuitikz} \draw
(0,-2.1) to[sV,l=Hydro] (0,0)
(2,0) node[transformer,anchor=base](T){}
(T.A1) to[short] (0,0)
(T.A2) to[short] (0,-2.1)
(T.B1) to[short,*-] (6,0) to[european resistor,l^=moteur] (6,-2.1) to[short] (4.5,-2.1) to[short] (4.5,-1.05) to[short,-*] (3,-1.05)
(T.B2) to[short,*-] (3,-2.1)
{[anchor=south] (T.B1) node{$+$}}
{[anchor=north] (T.B2) node{$-$}}
;\end{circuitikz}
\caption{\label{sch-init}Branchement initial.}
\end{figure}

b) À l'aide de l'oscilloscope, affichez la tension aux bornes de la résistance de 1~M$\Omega$ pour mesurer la tension moyenne et la tension efficace en modes continu CC et alternatif CA. Vous referez ces mesures après chaque modification du circuit.

c) Rajoutez une diode standard en polarité directe avec la charge, tel qu'indiqué à la figure~\ref{sch-half}. Rappelez-vous que la diode ne laisse passer que le courant circulant dans un sens, celui de son symbole en flèche, pas dans l'autre. Refaites l'étape b).

\begin{figure}[h]
\centering
\begin{circuitikz} \draw
(0,-2.1) to[sV, l=Hydro] (0,0)
(2,0) node[transformer, anchor=base](T){}
(T.A1) to[short] (0,0)
(T.A2) to[short] (0,-2.1)
(T.B1) to[D,*-] (6,0) to[R=1~M$\Omega$] (6,-2.1) to[short] (4.5,-2.1) to[short] (4.5,-1.05) to[short,-*] (3,-1.05)
(T.B2) to[short,*-] (3,-2.1)
{[anchor=south] (T.B1) node{$+$}}
{[anchor=north] (T.B2) node{$-$}}
;\end{circuitikz}
\caption{\label{sch-half}Redresseur simple alternance.}
\end{figure}

d) Modifiez le circuit précédent en ajoutant une deuxième diode, tel qu'illustré à la figure~\ref{sch-full-1}. Lorsque le circuit est monté, refaites les observations de l'étape b).

\begin{figure}[h]
\centering
\begin{circuitikz} \draw
(0,-2.1) to[sV, l=Hydro] (0,0)
(2,0) node[transformer, anchor=base](T){}
(T.A1) to[short] (0,0)
(T.A2) to[short] (0,-2.1)
(T.B1) to[D,*-] (6,0) to[R=1~M$\Omega$] (6,-1.8) to[short] (5.4,-1.8) to[short] (5.4,-1.05) to[short,-*] (3,-1.05)
(T.B2) to[D,*-] (6,-2.1) to[short] (8,-2.1) to[short] (8,0) to[short] (6,0)
{[anchor=south] (T.B1) node{$+$}}
{[anchor=north] (T.B2) node{$-$}}
;\end{circuitikz}
\caption{\label{sch-full-1}Redresseur double alternance.}
\end{figure}

Ce circuit laisse passer autant le courant positif que négatif, en redressant le courant négatif : il agit comme une fonction valeur absolue. Son désavantage est qu'il nécessite un transformateur ayant trois prises, ce qui n'est pas courant dans les applications de tous les jours. Le prochain circuit redressera aussi le courant, mais sans cette troisième connexion. Par contre, deux diodes supplémentaires devront être utilisées.

%La mesure avec la sonde, qui n'est pas différentielle, ne devrait pas bien fonctionner pour mesurer aux bornes de la charge de 1 M$\Omega$ à partir d'ici, mais pourtant...
e) Sur la plaquette de montage, réalisez le circuit de la figure~\ref{sch-pontgraetz}, puis refaites les mesures de l'étape b).
Malgré toutes ces améliorations, la tension aux bornes de la charge oscille toujours!

\begin{figure}[h]
\centering
\begin{circuitikz} \draw
(0,-2.1) to[sV, l=Hydro] (0,0)
(2,0) node[transformer, anchor=base](T){}
(T.A1) to[short] (0,0)
(T.A2) to[short] (0,-2.1)
(T.B1) to[short,*-] (5,0) to[D] (7,2) to[short] (11,2) to[R=1~M$\Omega$] (11,-2) to[short] (7,-2) to[D] (5,0)
(T.B2) to[short,*-] (3,-2.1)
(7,-2) to[D] (9,0) to[D] (7,2)
(9,0) to[short] (9.5,0) node[ground]{}
(3,-1.05) to[short,*-] (4,-1.05) node[ground]{}
{[anchor=south] (T.B1) node{$+$}}
{[anchor=north] (T.B2) node{$-$}}
;\end{circuitikz}
\caption[]{\label{sch-pontgraetz}Pont de diodes. Il n'y a bien qu'une seule mise à la terre: par convention pour simplifier les schémas de circuits, tous les points de référence de potentiel illustrés sont reliés par des fils, donc ont le même potentiel au final.}
\end{figure}

f) Ajoutez un condensateur à votre montage, en parallèle avec la charge, pour correspondre au circuit de la figure~\ref{sch-pontgraetz-modif}. Le condensateur permettra de stabiliser significativement la tension aux bornes de la charge ; elle deviendra presque constante. Observez-la une dernière fois à l'oscilloscope en faisant les mesures de l'étape b).

\begin{figure}[h]
\centering
\begin{circuitikz} \draw
(0,-2.1) to[sV,l=Hydro] (0,0)
(2,0) node[transformer, anchor=base](T){}
(T.A1) to[short] (0,0)
(T.A2) to[short] (0,-2.1)
(T.B1) to[short,*-] (5,0) to[D] (7,2) to[short] (11,2) to[R=1~M$\Omega$] (11,-2) to[short] (7,-2) to[D] (5,0)
(T.B2) to[short,*-] (3,-2.1)
(7,-2) to[D] (9,0) to[D] (7,2)
(9,0) to[short] (9.5,0) node[ground]{}
(3,-1.05) to[short,*-] (4,-1.05) node[ground]{}
(7,2) to[C=$\!$1~$\mu$F] (7,-2)
{[anchor=south] (T.B1) node{$+$}}
{[anchor=north] (T.B2) node{$-$}}
;\end{circuitikz}
\caption{\label{sch-pontgraetz-modif}Pont de diodes modifié.}
\end{figure}

g) Remplacez maintenant la résistance de 1~M$\Omega$ par le petit moteur et notez vos observations. Rendu à ce point, il devrait fonctionner à merveille! Si ce n'est pas le cas, pleurez en silence\dots~ou demandez l'aide d'un sympathique membre de l'équipe d'enseignement.


\subsection{Partie 2 --- Transistor non inverseur (collecteur commun)}

a) Faites le montage de la figure~\ref{sch-trans-non-inv} sur la plaquette de montage en utilisant une charge $R_{\mathrm{ch}}$ de 1~k$\Omega$. Pour $v_{\mathrm{s}}$, utilisez le bloc d'alimentation et fournissez une tension de 4~V. Pour $v_{\mathrm{g}}$, utilisez le générateur de fonctions et générez un signal rampe allant de 0~V à 3~V et oscillant à 100~Hz. Rappelez-vous que "l'amplitude" du générateur est en fait une valeur crête à crête pour éviter le piège d'un minimum à -1.5~V par défaut lorsque le décalage n'est pas ajusté.

\begin{figure}[h]
\centering
\begin{circuitikz} \draw
(0,0) node[ground]{} to[short] (3,0) to[V,l_=$v_{\mathrm{s}}$~(\textsc{dc})] (3,4) to[short] (0,4)
(0,0) to[R,l_=$R_{\mathrm{ch}}$] (0,2) to[Tnpn,n=T] (0,4)
(0,0) to[short] (-3,0) to[V,l^=$v_{\mathrm{g}}$~(\textsc{ac})] (-3,3) to[short] (T.B)
(0,2) to[short,-o] (1,2) node[right]{$v_{\mathrm{out}}$}
;\end{circuitikz}
\caption{\label{sch-trans-non-inv}Transistor non inverseur à collecteur commun. Rappelez-vous qu'une connexion à un seul  $v_{\mathrm{out}}$ indique quand même la mesure d'une différence de potentiel électrique par rapport à la référence, soit la mise à la terre par convention.}
\end{figure}

b) Sur le canal 1 de l'oscilloscope, affichez la tension fournie par le générateur, $v_{\mathrm{g}}$. Sur le canal 2, affichez le signal de sortie $v_{\mathrm{out}}$. Comparez les deux signaux à l'oscilloscope en mode de temps normal (sans rien changer) et en mode~XY (bouton du menu \textbf{Horiz}). Voyez-vous comment ce deuxième mode affiche la tension du canal 2 (ordonnée) en fonction du canal 1 en abscisse? Évaluez la pente de la droite en mode XY.


\subsection{Partie 3 optionnelle --- Transistor inverseur (émetteur commun)}

a) Modifiez le montage précédent afin d'obtenir celui de la figure~\ref{sch-trans-inv}. Utilisez la même charge de 1~k$\Omega$ ; seule sa position dans le circuit doit être modifiée. La résistance de 100~k$\Omega$ n'affecte pas le fonctionnement du transistor. En fait, elle est ajoutée seulement pour augmenter l'impédance d'entrée de ce circuit qui autrement serait trop faible. Utilisez les mêmes tensions $v_{\mathrm{s}}$ et $v_{\mathrm{g}}$ que dans la partie précédente.

\begin{figure}[h]
\centering
\begin{circuitikz} \draw
(0,0) node[ground]{} to[short] (3,0) to[V,l_=$v_{\mathrm{s}}$~(\textsc{dc})] (3,4) to[R,l_=$R_{\mathrm{ch}}$] (0,4)
(0,0) to[Tnpn,n=T] (0,4)
(0,0) to[short] (-3,0) to[V=$v_{\mathrm{g}}$~(\textsc{ac})] (-3,2) to[R=100~k$\Omega$] (T.B)
(0,4) to[short,-o] (-1,4) node[left]{$v_{\mathrm{out}}$}
;\end{circuitikz}
\caption{\label{sch-trans-inv}Transistor inverseur à émetteur commun.}
\end{figure}

b) Comparez à nouveau, à l'aide de l'oscilloscope, les tensions $v_{\mathrm{g}}$ (canal~1) et $v_{\mathrm{out}}$ (canal~2) en modes de temps normal et XY, puis évaluez la pente de la droite observée dans ce deuxième mode.


\subsection{Partie 4 --- Amplificateur opérationnel (ampli-op) \& son principe différentiel} \label{sec:ampli-diff}

Le monde des circuits intégrés est extrêmement vaste et complexe\footnote{Sans pouvoir remplacer des programmes de baccalauréat complets entre le génie électrique et le génie informatique, je recommande la série télévisée de fiction \textit{Halt and Catch Fire} qui montre l'intense aspect humain et commercial dans les premiers temps de la microélectronique et des ordinateurs avec quelques bons aspects documentaires.}, d'où la leçon essentielle est qu'il faut se référer à la fiche technique du manufacturier pour chaque puce. Ceci dit, l'élégance fondamentale demeure que tous ces circuits, allant jusqu'à nos ordinateurs et téléphones(!), sont constitués de versions miniaturisées des composants étudiés jusqu'à maintenant: résistance, condensateur, bobine d'inductance, diode et transistor. Il existe à peine quelques exceptions dont les relais électromécaniques et les vibrations de cristaux de quartz. 

Nous sommes donc ici en transition vers la deuxième moitié du cours qui survole quelques fonctions de base réalisables avec les principaux circuits électroniques ainsi que leur conception. Lorsque possible, nous prendrons l'approche physique de se ramener aux composants fondamentaux pour focaliser sur le principe de fonctionnement des puces plutôt que le détail de leur design.

a) Pour vous initier au fonctionnement de l'amplificateur opérationnel, réalisez le circuit suiveur de tension illustré à la figure~\ref{fig:suiveux}. La mise à la terre COM (abbréviation de \textit{common}) est définie comme référence de potentiel électrique pour tous les appareils, explicitement indiquée par les connecteurs en noir sur la source d'alimentation. Mettez la puce sous tension seulement à la fin, en reliant les bornes $v_{\mathrm{cc}+}$ et $v_{\mathrm{cc}-}$ aux sorties $\pm25$~V de la source pour y appliquer des tensions de +10~V et $-10$~V respectivement.

\begin{figure}[h]
\centering
\begin{circuitikz} \draw
(0,0) node[op amp](opamp){}
(opamp.+)
(opamp.-) to[short] (-1.2,0.5)
(-1.2,0.5) to[short] (-1.2,2) to[short] (1.2,2) to[short] (1.2,0)
(opamp.out) to[short,-o] (1.7,0) node[right]{$v_{\mathrm{out}}$}
(opamp.down) ++ (0,-0.5) node[below]{$v_{\mathrm{cc}-}$} -- (opamp.down)
(opamp.up) ++ (0,0.5) node[above]{$v_{\mathrm{cc}+}$} -- (opamp.up)
(-4.5,0) to[short,*-] (-4,0) node[ground]{~COM}
;\end{circuitikz}
\caption{\label{fig:suiveux}Suiveur de tension avec puce ampli-op intégrée.}
\end{figure}

b) Pour vérifier que la tension est bien suivie, \textit{i.e.} théoriquement identique, entre l'entrée non inverseuse et la sortie $v_{\mathrm{out}}$ de l'ampli-op, exécutez une seule des deux caractérisations suivantes:
\begin{enumerate}
    \item Reliez l'entrée non inverseuse (+) à la sortie +6~V du bloc d'alimentation et variez la tension de 0~V à 2~V par bonds de 0,2~V. Notez la tension de sortie à chaque bond en reliant un multimètre à $v_{\mathrm{out}}$.\\[0.8ex]
    OU
    \item Reliez l'entrée non inverseuse (+) au générateur de fonctions pour y envoyer une rampe [0,2]~V à une fréquence de 1~kHz. Observez à l'oscilloscope que cette tension en entrée se superpose à celle à la sortie $v_{\mathrm{out}}$, puis évaluez la pente entre les deux canaux à l'aide du mode XY (bouton du menu \textbf{Horiz}).
\end{enumerate}

% \begin{center}
%     \large \textbf{Le reste de cette Partie 4 peut être complété avant ou après la séance de laboratoire.}
% \end{center}

% Revenez maintenant à la simulation que vous avez faite en préparation du laboratoire. En utilisant deux transistors simultanément dans le circuit de la figure~\ref{ampli-diff-simple}, l'un inverseur et l'autre non inverseur, il devient possible d'obtenir un courant constant amplifié\footnote{\url{https://www.allaboutcircuits.com/textbook/experiments/chpt-5/differential-amplifier/}}. Ce principe est à la base de l'amplificateur différentiel, qui est lui-même la pierre angulaire de l'amplificateur opérationnel dans la puce noire que vous venez d'étudier.

% c) Alimentez (\texttt{Add Voltage Source}) l'amplificateur différentiel dans votre simulation avec $v_{\mathrm{cc}+}=10$~V et $v_{\mathrm{cc}-}=-10$~V.

% d) Vérifiez (\texttt{View in Scope}) à la sortie $v_{\mathrm{out}}$ de l'amplificateur que l'entrée $v_{\mathrm{in},+}$ est non inverseuse et l'entrée $v_{\mathrm{in},-}$ est inverseuse en y appliquant une tension CC fixe de 1~V et une rampe (\texttt{Sawtooth}) sur la plage [0,2]~V respectivement. Refaites la même vérification en échangeant les tensions sur les entrées.

% %Voir si je ne pourrais pas améliorer la compréhension de la mise à la terre commune COM vs +/- potentiel en ramenant le ground dans la figure ici et en prep + exercice de batteries inspiré du vidéo que j'ai sauvegardé sur YouTube.

% \begin{figure}[h]
% \centering
% \begin{circuitikz} \draw
% (0,8) node[left]{$v_{\mathrm{cc}+}$} to[short,*-] (6,8) to[R=100~k$\Omega$] (6,6) to[short,-*] (7,6) node[right]{sortie~($v_{\mathrm{out}}$)}
% (3.5,4) to[R=1~k$\Omega$] (6,4) to[Tnpn, mirror,n=Q2] (6,6)
% (Q2.B) to[short,-*] (7,5) node[right]{entrée~-~($v_{\mathrm{in},-}$)}
% (3.5,4) to[R,l_=1~k$\Omega$] (1,4) to[Tnpn,n=Q1] (1,6) to[short] (1,8)
% (Q1.B) to[short,-*] (0,5) node[left]{entrée~+~($v_{\mathrm{in},+}$)}
% (3.5,4) to[R=100~k$\Omega$,-*] (3.5,1.5) node[right]{$v_{\mathrm{cc}-}$}
% %(1,2.5) to[short,*-] (1.5,2.5) node[ground]{}
% ;\end{circuitikz}
% \caption{\label{ampli-diff-simple}Amplificateur différentiel simpliste.}
% \end{figure}

% e) Maintenant, rajoutez une rétroaction négative afin que l'amplificateur se comporte comme un suiveur de tension. Pour ce faire, remplacez la source de tension à l' entrée inverseuse - par un fil pour la relier à la sortie de l'amplificateur différentiel.

% f) Vérifiez que la sortie suit effectivement la tension de votre choix à l'entrée non inverseuse. N'hésitez pas à essayer plusieurs formes de signal!
%Capsule vidéo montrant que ça marche comme de la bouette en pratique! Checker la version "pré-faite" de falstad dans menu Draw->Transistor et celle de AllAboutCircuits <https://www.allaboutcircuits.com/textbook/experiments/chpt-5/differential-amplifier/>

\subsection{Partie 5 --- Comparateur}
%Idée d'utiliser une sortie de son téléphone pour aller y chercher l'accéléromètre pour comparer son signal ;)
Un comparateur, comme son nom l'indique, permet de comparer deux signaux.  Il s'agit d'un des circuits les plus simples contenant un amplificateur opérationnel et qui est aussi utilisé dans de nombreuses applications.

\begin{figure}[h]
\centering
\begin{circuitikz} \draw
(0,0) node[op amp](opamp){}
(opamp.out) to[leD] (4,0) to[R,l_=1~k$\Omega$] (4,-2.5) node[ground]{}
(opamp.down) ++ (0,-0.5) node[below]{$-5$~V} -- (opamp.down)
(opamp.up) ++ (0,0.5) node[above]{$+5$~V} -- (opamp.up)
(-3,-2.5) node[ground]{} to[sV=0~V~--~2~V] (-3,-0.5) to[short] (-1.2,-0.5)
(-6,-2.5) node[ground]{} to[V=1~V] (-6,0.5) to[short] (-1.2,0.5)
;\end{circuitikz}
\caption{\label{comparateur-1}Circuit pour l'analyse du comparateur.}
\end{figure}

a) Sur la plaquette de montage, réalisez le circuit de la figure~\ref{comparateur-1}. Utilisez l'amplificateur opérationnel sur circuit intégré. Prenez la sortie $+6$~V du bloc d'alimentation pour fournir la tension de 1~V à l'entrée inverseuse de l'amplificateur. Prenez les sorties $+25$~V et $-25$~V pour alimenter l'ampli-op à $\pm5$~V. Utilisez le générateur de fonctions pour fournir une tension sinusoïdale, oscillant entre 0~V et 2~V, à l'entrée non inverseuse de l'amplificateur opérationnel. Affichez aussi le signal généré à l'oscilloscope. Sur ce dernier, faites afficher la mesure de la tension efficace (CC EFF -- plein écran).

Pour l'instant, utilisez une fréquence de 500~mHz. Utilisez la plus petite échelle de temps (5~ns par division) pour que le signal affiché ne soit qu'une ligne horizontale qui monte et qui redescend. Ainsi, la valeur efficace affichée correspond à la valeur en temps réel de la tension fournie.

b) Observez conjointement la LED et la valeur affichée à l'oscilloscope.

c) Une dernière manipulation optionnelle: Augmentez progressivement la fréquence du signal généré par l'oscilloscope. Trouvez la fréquence limite à partir de laquelle vos yeux ne peuvent plus détecter les oscillations. Rappelez-vous de cette valeur la prochaine fois que vous irez acheter un téléviseur ; si le vendeur vous assure que payer pour un téléviseur à trois milliards de hertz en vaut la peine, vous pourrez lever un sourcil montrant votre scepticisme\ldots
%%e) Manipulation avec l'accéléromètre, signal obtenu d'un téléphone intelligent.

%%\subsection{Partie ? --- Contrôle, Circuit d'asservissement...}
%%À voir si ça entre dans le temps pour ajouter un de ces quatre :(
%LA PARTIE 6 CI-DESSOUS EST MAINTENANT RENDUE AU LABORATOIRE III, SUPPRIMER ÉVENTUELLEMENT ICI UN COUP QU'ON EST CONVAINCUS QUE ÇA FITTE :)
% \subsection{Partie 6 optionnelle --- Régulateur de tension intégré}

% a) Sur la plaquette de montage, montez le circuit suivant (figure~\ref{sch-reg-1}). Consultez la fiche technique du régulateur de tension afin de faire les bons branchements.

% \begin{figure}[h]
% \centering
% \begin{circuitikz} \draw
% (0,0) to[V=$v_{\textrm{s}}$] (0,2.5) to[R=1~k$\Omega$] (3,2.5) to[european resistor,l^=regulateur] (5,2.5) node[right]{$v_{\mathrm{out}}$}
% (0,0) node[ground]{} to[short] (4,0) to[short] (4,2.26)
% ;\end{circuitikz}
% \caption{\label{sch-reg-1}Régulateur de tension avec puce.}
% \end{figure}

% b) Faites varier la source de 0~volt à 15~volts par incréments de 1~volt. À l'aide d'un multimètre en mode voltmètre (DCV), mesurez à chaque fois la tension $v_{\mathrm{out}}$.


\section{Questions d'analyse VI} \label{sec:grade}
\vspace{-0.5cm}
\noindent$\rightarrow$ À remettre en équipe sur \href{https://www.gradescope.com/}{Gradescope} en ajoutant votre \href{https://help.gradescope.com/article/m5qz2xsnjy-student-add-group-members}{binôme de laboratoire} $\leftarrow$
\begin{enumerate}
\item Pour chaque signal ci-après, dites si la tension efficace prise en mode CC est supérieure, inférieure ou égale à la tension efficace prise en mode CA.\\a) Signal sinusoïdal.\\b) Signal sinusoïdal redressé en simple alternance.\\c) Signal sinusoïdal redressé en double alternance.
\item Dans la première partie du laboratoire, après avoir modifié le pont de diodes afin d'avoir un signal moyenné, vous avez sûrement remarqué, à l'aide de l'oscilloscope, que ce dernier n'était pas encore tout à fait constant : les oscillations étaient encore notables. Parmi les options suivantes, laquelle ou lesquelles permet(tent) de \textit{réduire} ces oscillations?
\begin{itemize}
    \item Augmenter la capacité du condensateur.
    \item Augmenter la fréquence du signal.
    \item Augmenter l'amplitude en tension du signal d'entrée.
    \item Augmenter la teneur en fibres de votre alimentation.
\end{itemize}
\item À la partie 2, le transistor est utilisé dans son mode d'amplification linéaire, parfois dit \textit{mode actif}. Quel est leur facteur d'amplification, défini par $\Delta v_{\mathrm{out}}/\Delta v_{\mathrm{in}}$, où $v_{\mathrm{in}}$ est la tension appliquée sur la base du transistor? %Ramener question sur partie 3 si possible, en faisant remarquer la pertinence des noms inverseur et non inverseur.

\begin{figure}[h]
\centering
\begin{circuitikz} \draw
(0,8) node[left]{$v_{\mathrm{cc}+}$} to[short,*-] (6,8) to[R=100~k$\Omega$] (6,6) to[short,-*] (7,6) node[right]{sortie~($v_{\mathrm{out}}$)}
(3.5,4) to[R=1~k$\Omega$] (6,4) to[Tnpn, mirror,n=Q2] (6,6)
(Q2.B) to[short,-*] (7,5) node[right]{entrée~-~($v_{\mathrm{in},-}$)}
(3.5,4) to[R,l_=1~k$\Omega$] (1,4) to[Tnpn,n=Q1] (1,6) to[short] (1,8)
(Q1.B) to[short,-*] (0,5) node[left]{entrée~+~($v_{\mathrm{in},+}$)}
(3.5,4) to[R=100~k$\Omega$,-*] (3.5,1.5) node[right]{$v_{\mathrm{cc}-}$}
%(1,2.5) to[short,*-] (1.5,2.5) node[ground]{}
;\end{circuitikz}
\caption{\label{ampli-diff-simple}Amplificateur différentiel simpliste.}
\end{figure}

\item À la partie 4 du laboratoire, vous avez utilisé un ampli-op dans un circuit intégré. À la figure \ref{ampli-diff-simple}, un circuit composé de deux transistors, l’un inverseur et l’autre non inverseur, illustre comment il est possible d'obtenir un courant amplifié à partir de composants de base. Ce circuit, appelé \textit{paire différentielle}, se trouve au coeur de la plupart des ampli-op. Pour étudier son comportement, simulez-le sur le site <\url{https://www.falstad.com/circuit/}>.
\\a) Alimentez (\texttt{Add Voltage Source}) l'amplificateur différentiel dans votre simulation avec $v_{\mathrm{cc}+}=10~V$ et $v_{\mathrm{cc}-}=-10~V$.
\\b) Vérifiez que les entrées $v_{\mathrm{in},+}$ (non inverseuse) et $v_{\mathrm{in},-}$ (inverseuse) ont des effets opposés sur la sortie $v_{\mathrm{out}}$. Pour ce, appliquez une tension CC fixe de $1~V$ sur l'entrée non inverseuse et une rampe ($\texttt{Sawtooth}$) sur la plage $[0,2]~V$ sur l'entrée inverseuse. Notez l'effet de la rampe sur $v_{\mathrm{out}}$ en affichant simultanément les deux signaux dans des oscilloscopes, puis répéter en interchangeant les sources sur les entrées.

Vous aurez pu observer que bien que la tension de sortie ne soit pas une fonction linéaire de la différence de tension entre $v_{\mathrm{in},+}$ et $v_{\mathrm{in},-}$, la première entrée tend à faire varier $v_{\mathrm{out}}$ dans la même direction qu'elle, tandis que la seconde pousse $v_{\mathrm{out}}$ dans la direction opposée. Le comportement de ce circuit est donc loin du comportement idéal qu'il est possible d'approcher avec les circuits plus complexes des ampli-op, mais il illustre néanmoins le principe de base d'un amplificateur différentiel qu'il est possible d'obtenir avec seulement deux transistors.

Pour l'évaluation, remettez un lien vers votre simulation complétée du circuit (\texttt{File} $\rightarrow$ \texttt{Export As Link...}) ainsi qu'une capture d'écran qui inclut une vue des signaux sur les oscilloscopes.
\item En pratique, un suiveur de tension est\dots
\begin{itemize}
    \item \dots inutile puisqu'il ne fait rien concrètement, il ne fait que donner une tension de sortie égale à l'entrée.
    \item \dots utile puisqu'il sert de tampon (\textit{buffer}) pour fixer la tension à un point en fonction d'un autre sans en affecter le comportement du circuit, ce qui peut permettre de construire une source de tension commandée, par exemple.
    \item \dots très utile puisqu'il permet de connaître la tension artérielle d'un patient.
\end{itemize}
%\item Comparez les régulateurs de tension du laboratoire~III et de ce laboratoire. Tracez pour chacun la courbe de la tension de sortie $v_{\mathrm{out}}$ en fonction de la tension d'entrée $v_{\mathrm{s}}$. Afin de faire une meilleure comparaison, normalisez vos valeurs (divisez les tensions de sortie de chaque composant par la valeur maximale obtenue) et affichez les deux courbes sur le même graphique.
\item Cette question met en contraste une approche déductive en (a) avec une approche de conception en (b) qui deviendra progressivement prépondérante pendant le reste du cours pour favoriser le développement de votre autonomie expérimentale.
    \begin{enumerate}
    \item Exprimez algébriquement la tension $v_{\mathrm{out}}$ en fonction de $v_1$, $v_2$ et $R$ pour le circuit à la figure~\ref{sch-prep}.
    \begin{figure}[h]
    \centering
    \begin{circuitikz} \draw
    (0,0) node[op amp](opamp){}
    (opamp.+) to[short] (-1.2,-0.5) to[short] (-1.2,-1) node[ground]{}
    (opamp.-) to[short] (-1.2,0.5) to[short] (-1.2,2)
    (-4,0.5) node[left]{$v_2$} to[R=$R$] (-1.2,0.5)
    (-4,2) node[left]{$v_1$} to[R=$R$] (-1.2,2) to[R=$R$] (1.2,2) to[short] (1.2,0)
    (opamp.out) to[short,-o] (2,0) node[right]{$v_{\mathrm{out}}$}
    ;\end{circuitikz}
    \caption{\label{sch-prep} Circuit non linéaire d'ampli-op à analyser.}
    \end{figure}
    \item Considérez deux points à des tensions différentes $v_1$ et $v_2$. Dessinez un circuit utilisant un amplificateur opérationnel et quelques résistances pour lequel la tension de sortie de l'ampli-op serait définie par l'équation suivante: $v_{\mathrm{out}}=2\,v_1+0,\!5\,v_2.$
    \end{enumerate}
\end{enumerate}

\end{document}