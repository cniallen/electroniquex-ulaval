\RequirePackage[l2tabu, orthodox]{nag} %Check for obsolete commands
\documentclass[canadien,12pt,oneside,letterpaper]{article}
%
%-----------------------------------------------------
%Loading packages
%
\usepackage[utf8]{inputenc}
\usepackage[T1]{fontenc}
\usepackage[canadien]{babel}
\usepackage{lmodern}
\usepackage{textcomp}
\usepackage{amsmath,amssymb}
\usepackage{siunitx}
\usepackage[svgnames]{xcolor}
\usepackage[colorlinks=true,allcolors=blue]{hyperref}
\usepackage[all]{hypcap}
\usepackage{graphicx}
\usepackage[americanvoltages,americancurrents,siunitx]{circuitikz}
\usetikzlibrary{babel}
\usepackage{pgfplots}
\pgfplotsset{compat=1.15}
\usepackage{mathrsfs}
\usetikzlibrary{arrows}
\usepackage{caption}
%\usepackage{subcaption}
\usepackage{subfig}
\usepackage[letterpaper,headheight=15pt]{geometry}
\usepackage{fancyhdr}
\usepackage{setspace}
%
\captionsetup{font=small,labelfont=bf,margin=0.1\textwidth}
\pagestyle{myheadings}
\markboth{GPH-2006/PHY-2002~---~Introduction~aux~signaux~numériques}{GPH-2006/PHY-2002~---~Introduction~aux~signaux~numériques}

\begin{document}
 
\title{\textbf{Complément}\\Introduction aux signaux numériques}
\author{Jérémie Guilbert \& Claudine Allen}
\date{}
\maketitle


Il a déjà été vu comment utiliser un amplificateur opérationnel ou, plus simplement, un transistor, afin de commander un signal à partir d'un autre signal, généralement plus faible. Ces circuits sont appelés des circuits actifs, puisqu'ils «génèrent» un signal et participent ainsi activement au fonctionnement du circuit, contrairement aux composants passifs (comme les résistances) qui ne font que «subir» les signaux. La grande sensibilité des composants actifs peut faire en sorte qu'ils réagissent fortement à de faibles variations de courant ou de tension dues, par exemple, à la présence de bruit. Les circuits logiques, qui utilisent des signaux numériques, n'ont pas ce désavantage.

Contrairement aux signaux analogiques, dont l'amplitude peut prendre n'importe quelle valeur réelle, les signaux numériques ont une quantité finie d'amplitudes possibles. En d'autres mots, les amplitudes sont discrétisées. En algèbre de Boole, seules deux valeurs sont possibles qui, selon les applications, portent des noms différents : marche/arrêt, un/zéro, vrai/faux, oui/non, blanc/noir, sud/nord, niveau haut/niveau bas, circuit ouvert/circuit fermé, etc. En pratique, ces niveaux sont habituellement bien distincts et ne se chevauchent pas. Un signal dont l'amplitude est 0~V ou 5~V, par exemple, pourrait être considéré \textit{faux} si la tension est inférieure à 0,8~V et \textit{vrai} si elle est supérieure à 2~V. La zone entre 0,8~V et 2~V ne serait alors pas considérée. Ainsi, même si une source de bruit fait en sorte que le signal théorique de 5~V n'est en fait que 4,2~V, le comportement du circuit n'en subira aucune conséquence.


\section{Portes logiques, algèbre de Boole et tables de vérité}

Les portes logiques, aussi appelées fonctions logiques, effectuent des opérations sur les signaux à deux niveaux (signaux binaires). L'algèbre booléen est la transcription mathématique de ces opérations. Les tables de vérité, quant à elles, permettent de voir dans un tableau toutes les possibilités qu'offre une porte donnée. Voyons les principales portes\footnote{Dans ce document, seules les versions les plus simples de chaque porte sont abordées. Toutefois, il existe aussi des versions de ces dernières avec des entrées supplémentaires ; leur fonctionnement général demeure le même.} et opérations.


\subsection{Inversion --- Porte NON}

Une porte NON (\textit{NOT}), aussi appelée \textit{inverseur}, prend un signal et rend le signal opposé. Schématiquement, une porte NON est représentée par un triangle suivi d'une bulle. La bulle à la sortie signifie que celle-ci est inversée (les bulles signifient toujours une inversion). Mathématiquement, on représente l'inversion en plaçant une barre au dessus du signal. L'opposé de $A$ s'écrit ainsi $\overline{A}$.

\begin{center}
\begin{circuitikz} \draw
(0,0) node[not port](not){}
(not.in) node[left]{$A$}
(not.out) node[right]{$\overline{A}$}
;\end{circuitikz}
\end{center}

\begin{center}
\begin{tabular}{|c|c|}
\hline
Entrée & Sortie \\
\hline
$A$ & $\overline{A}$ \\
\hline
0 & 1 \\
\hline
1 & 0 \\
\hline
\end{tabular}
\end{center}


\subsection{Multiplication --- Porte ET}

Une porte ET (\textit{AND}) fournit un signal \textit{vrai} si et seulement si ses deux entrées sont simultanément \textit{vraies}. Avec la notation un/zéro, cela revient à une multiplication. Ainsi, pour deux entrées $A$ et $B$, la sortie sera $A\cdot B$.

\begin{center}
\begin{circuitikz} \draw
(0,0) node[and port](and){}
(and.in 1) node[left]{$A$}
(and.in 2) node[left]{$B$}
(and.out) node[right]{$A\cdot B$}
;\end{circuitikz}
\end{center}

\begin{center}
\begin{tabular}{|c|c|c|}
\hline
\multicolumn{2}{|c|}{Entrées} & Sortie \\
\hline
$A$ & $B$ & $A\cdot B$ \\
\hline
0 & 0 & 0 \\
\hline
0 & 1 & 0 \\
\hline
1 & 0 & 0 \\
\hline
1 & 1 & 1 \\
\hline
\end{tabular}
\end{center}

\subsubsection{Porte NON-ET}

Une porte NON-ET (\textit{NAND}) permet d'obtenir l'exact opposé d'une porte ET. Il ne s'agit en fait que d'une porte ET dont la sortie est inversée.

\begin{center}
\begin{circuitikz} \draw
(0,0) node[nand port](nand){}
(nand.in 1) node[left]{$A$}
(nand.in 2) node[left]{$B$}
(nand.out) node[right]{$\overline{A\cdot B}$}
;\end{circuitikz}
\end{center}

\begin{center}
\begin{tabular}{|c|c|c|}
\hline
\multicolumn{2}{|c|}{Entrées} & Sortie \\
\hline
$A$ & $B$ & $\overline{A\cdot B}$ \\
\hline
0 & 0 & 1 \\
\hline
0 & 1 & 1 \\
\hline
1 & 0 & 1 \\
\hline
1 & 1 & 0 \\
\hline
\end{tabular}
\end{center}


\subsection{Addition --- Porte OU}

La porte OU (\textit{OR}) permet d'additionner deux signaux (avec la représentation un/zéro). Lorsque les deux entrées valent 1, la sortie vaudra aussi 1 (jamais 2!).

\begin{center}
\begin{circuitikz} \draw
(0,0) node[or port](or){}
(or.in 1) node[left]{$A$}
(or.in 2) node[left]{$B$}
(or.out) node[right]{$A+B$}
;\end{circuitikz}
\end{center}

\begin{center}
\begin{tabular}{|c|c|c|}
\hline
\multicolumn{2}{|c|}{Entrées} & Sortie \\
\hline
$A$ & $B$ & $A+B$ \\
\hline
0 & 0 & 0 \\
\hline
0 & 1 & 1 \\
\hline
1 & 0 & 1 \\
\hline
1 & 1 & 1 \\
\hline
\end{tabular}
\end{center}

Similairement, il existe une porte NON-OU (\textit{NOR}), qui a le comportement opposé d'une porte OU. Sa sortie s'écrit $\overline{A+B}$. Sur le schéma, il suffit de rajouter une bulle à la sortie de la porte OU pour avoir une NON-OU (la porte NON-OU et sa table de vérité ne sont pas représentées ici pour ne pas alourdir inutilement le document).


\subsection{Addition spéciale --- Porte OU exclusif}

La porte OU exclusif (\textit{XOR}) fait aussi l'addition des deux entrées, sauf qu'elle renvoit une valeur \textit{fausse} lorsque les deux entrées sont \textit{vraies}. La notation algébrique utilise un symbole d'addition inscrit dans un cercle. Ainsi, avec deux entrées $A$ et $B$, la sortie d'un OU~exclusif s'écrira $A\oplus B$.

\begin{center}
\begin{circuitikz} \draw
(0,0) node[xor port](xor){}
(xor.in 1) node[left]{$A$}
(xor.in 2) node[left]{$B$}
(xor.out) node[right]{$A\oplus B$}
;\end{circuitikz}
\end{center}

\begin{center}
\begin{tabular}{|c|c|c|}
\hline
\multicolumn{2}{|c|}{Entrées} & Sortie \\
\hline
$A$ & $B$ & $A\oplus B$ \\
\hline
0 & 0 & 0 \\
\hline
0 & 1 & 1 \\
\hline
1 & 0 & 1 \\
\hline
1 & 1 & 0 \\
\hline
\end{tabular}
\end{center}

Il existe aussi une porte NON-OU exclusif (\textit{XNOR}) qui renvoie une sortie vraie si et seulement si les deux entrées ont la même valeur (pas représentée).


\subsection{Équivalences de De Morgan}

Les lois d'équivalence de De Morgan permettent de substituer une porte ET par une porte OU, et \textit{vice versa}, en ajoutant ou retirant certains inverseurs. Une porte NON-ET est ainsi équivalente à une porte OU dont les deux entrées sont inversées. Aussi, une porte NON-OU équivaut à une porte ET dont les deux entrées sont inversées. Dans la langue mathématique, les lois d'équivalence de De Morgan s'écrivent ainsi:
\begin{subequations}
\begin{gather}
\overline{A\cdot B}=\overline{A}+\overline{B} \\
\overline{A+B}=\overline{A}\cdot\overline{B}.
\end{gather}
\end{subequations}

Dans la langue française, ces deux équivalences s'énoncent ainsi:\\
\emph{--- La négation de la conjonction de deux propositions est équivalente à la disjonction des négations des deux propositions.\\
--- La négation de la disjonction de deux propositions est équivalente à la conjonction des négations des deux propositions.}

Grâce aux deux lois de De Morgan, il est possible de remplacer n'importe quelle porte logique par une combinaison de portes NON-ET (\textit{NAND}) ou par une combinaison de portes NON-OU (\textit{NOR}) ; ces deux portes sont des \textit{portes universelles}. L'ordinateur embarqué de navigation et de pilotage des missions Apollo, l'\textit{Apollo Guidance Computer}, a été construit exclusivement de portes NON-OU, soit environ 5600 dans les dernières versions.


\subsection{Logique réversible}

En connaissant une fonction ainsi que ses entrées, il est toujours possible d'en déduire les sorties. Une fonction logique est dite \textit{réversible} si la réciproque est vraie, c'est-à-dire si les entrées peuvent être déduites connaissant la fonction et ses sorties. Pour être réversible, une porte logique doit obligatoirement avoir le même nombre d'entrées que de sorties et tous les états doivent être présents, autant à l'entrée qu'à la sortie, quoiqu'ils puissent être dans un ordre différent. Parmi les exemples précédents, seule la porte NON est réversible.

Les portes réversibles dissipent moins de chaleur (en principe : aucune). Dans les portes classiques non réversibles, il y a moins d'information dans les sorties que dans les entrées : les états d'entrée sont perdus. Cette perte d'information équivaut à une perte d'énergie, dissipée sous forme de chaleur, selon le deuxième principe de la thermodynamique (entropie de Shannon). Une porte réversible, quant à elle, ne fait qu'intervertir les états, sans perte d'information : l'énergie est conservée.


\subsubsection{Porte de Toffoli}

La porte de Toffoli (\textit{CCNOT} pour \textit{controlled-controlled-not}) est un porte inverseuse à double contrôle. Elle a trois entrées et trois sorties. Les deux premières entrées, les entrées de contrôle, demeurent inchangées. Lorsqu'elles valent toutes deux \textit{un}, alors la troisième entrée est inversée.

\begin{center}
\begin{tabular}{|c|c|c|c|c|c|}
\hline
\multicolumn{3}{|c|}{Entrées} & \multicolumn{3}{|c|}{Sorties} \\
\hline
$C_1$ & $C_2$ & $E$ & $C_1$ & $C_2$ & $S$ \\
\hline
0 & 0 & 0 & 0 & 0 & 0 \\
\hline
0 & 0 & 1 & 0 & 0 & 1 \\
\hline
0 & 1 & 0 & 0 & 1 & 0\\
\hline
0 & 1 & 1 & 0 & 1 & 1 \\
\hline
1 & 0 & 0 & 1 & 0 & 0 \\
\hline
1 & 0 & 1 & 1 & 0 & 1 \\
\hline
1 & 1 & 0 & 1 & 1 & 1 \\
\hline
1 & 1 & 1 & 1 & 1 & 0 \\
\hline
\end{tabular}
\end{center}

La sortie $S$ peut alors se calculer ainsi:
\begin{equation}
S=E\oplus \left(C_1\cdot C_2\right).
\end{equation}

En plus d'être \textit{réversible}, la porte de Toffoli est \textit{universelle} : elle peut être utilisée pour construire un circuit réversible pour représenter n'importe quelle fonction logique.


\subsubsection{Porte de Fredkin}

La porte de Fredkin (\textit{CSWAP} pour \textit{controlled swap}) est un autre exemple de porte réversible universelle. L'entrée de contrôle est inchangée. Lorsqu'elle vaut \textit{un}, alors les deux autres entrées sont interchangées.

\begin{center}
\begin{tabular}{|c|c|c|c|c|c|}
\hline
\multicolumn{3}{|c|}{Entrées} & \multicolumn{3}{|c|}{Sorties} \\
\hline
$C$ & $E_1$ & $E_2$ & $C$ & $S_1$ & $S_2$ \\
\hline
0 & 0 & 0 & 0 & 0 & 0 \\
\hline
0 & 0 & 1 & 0 & 0 & 1 \\
\hline
0 & 1 & 0 & 0 & 1 & 0\\
\hline
0 & 1 & 1 & 0 & 1 & 1 \\
\hline
1 & 0 & 0 & 1 & 0 & 0 \\
\hline
1 & 0 & 1 & 1 & 1 & 0 \\
\hline
1 & 1 & 0 & 1 & 0 & 1 \\
\hline
1 & 1 & 1 & 1 & 1 & 1 \\
\hline
\end{tabular}
\end{center}

Les sorties peuvent alors se calculer ainsi:
\begin{subequations}
\begin{gather}
S_1=\overline{C}\,E_1+C\,E_2\\
S_2=C\,E_1+\overline{C}\,E_2.
\end{gather}
\end{subequations}

Pour prouver que la porte de Fredkin est universelle, il suffit de montrer qu'elle permet de construire la fonction NON ainsi que la fonction ET, puisque la combinaison de ces deux fonctions donne une porte universelle (NON-ET).
\begin{itemize}
\item Si on fixe $E_1=0$ et $E_2=1$, alors la sortie $S_2$ inverse l'entrée $C$, soit : $S_2=\overline{C}$.
\item Si on fixe $E_2=0$, alors les entrées $E_1$ et $C$ forment une porte ET avec la sortie $S_2$, soit: $S_2=E_1\cdot C$.
\end{itemize}


\subsection{Puces TTL et CMOS}

Il existe différentes façons de construire une porte logique donnée dans un circuit intégré. Deux grandes familles de puces sont principalement utilisées: les TTL et les CMOS.

Les puces TTL (\textit{transistor-transistor-logic}) sont formées à partir de transistors bipolaires (BJT). Leurs principaux avantages sont leurs faibles coûts de construction et leur très grande résistance, autant électrique que thermique. Les circuits logiques TTL doivent fonctionner avec un niveau bas à 0~V et un niveau haut à 5~V.

Les circuits intégrés CMOS (\textit{complementary metal-oxide-semiconductor}) utilisent quant à eux des transistors à effet de champ (MOSFET). Les circuits CMOS sont ainsi beaucoup plus sensibles et ne dissipent que très peu de puissance. Ils sont aussi très peu bruyants. Les CMOS ont un niveau bas aussi à 0~V, alors que leur niveau haut peut être choisi dans une plage d'habituellement 3~V à 15~V.


\section{Numérotations}


\subsection{Base 1 --- numérotation unaire}

Le système de numérotation unaire est le plus simple qui soit. Il consiste tout simplement à répéter un même symbole. Si ce symbole est une barre verticale, le nombre deux s'écrira \verb+||+, trois s'écrira \verb+|||+ et huit pourra s'écrire \verb+|||||+~\verb+|||+. L'écriture devient cependant très lourde pour les grands nombres.


\subsection{Base 10 --- numérotation décimale}

Le système de numérotation utilisant la base 10, le système décimal, est de loin le plus répandu à travers le monde. Si la numérotation arabe (0, 1, 2, 3, 4, 5, 6, 7, 8, 9) est la plus utilisée, ce n'est pas la seule à utiliser la base 10 : les Égyptiens de l'Antiquité utilisaient aussi un système décimal. Historiquement, le système décimal vient du fait que les êtres humains ont (habituellement) dix doigts.


\subsection{Base 2 --- numérotation binaire}

Si le système décimal est pratique pour compter sur ses doigts, le système binaire devient quant à lui tout à fait approprié dans l'étude des circuits logiques et, par extension, partout en informatique. La numérotation binaire n'utilise que deux chiffres: 0 et 1. L'information contenue dans un chiffre s'appelle \textit{bit}. Le terme bit vient de la contraction de l'anglais \textit{binary digit}, avec un jeu de mots avec le mot \textit{bit} qui signifie «petit morceau». Un bit est nécessaire pour représenter une valeur vraie ou fausse. L'ensemble des nombres de 0 à 7 a besoin quant à lui de trois bits pour être représenté. À noter que le terme anglais pour \textit{bit} est aussi \textit{bit} (et non \textit{byte}, qui signifie habituellement \textit{octet} --- un octet correspond à huit bits).

Le principal inconvénient avec la numérotation binaire est qu'elle devient rapidement lourde lorsque les nombres augmentent, par exemple:
\begin{equation*}
\left(999\right)_{10}=\left(1111100111\right)_2.
\end{equation*}


\subsection{Base 8 --- numérotation octale}

La base 8 offre la compatibilité avec la logique binaire tout en permettant une écriture plus compacte. En effet, un chiffre en numérotation octale équivaut à trois bits. La conversion de la numérotation binaire à la numérotation octale est ainsi très simple ; il suffit de rassembler les bits d'un nombre binaire en groupes de trois et à convertir chaque groupe de trois bits.
\begin{equation*}
\left(1111100111\right)_2=\left(~\left(001\right)~\left(111\right)~\left(100\right)~\left(111\right)~\right)_2=\left(1747\right)_8=\left(999\right)_{10}
\end{equation*}

Bien qu'il devienne pertinent avec la venue des circuits logiques et de l'informatique, le système octal a toutefois une histoire beaucoup plus ancienne ; Charles~XII de Suède l'aurait inventé au début du XVIII\up{e} siècle. Ce système aurait été plus approprié pour faire le décompte de l'inventaire de poudre à canon, les boîtes étant cubiques.

Mais le décompte octal a pu être utilisé bien avant, à la place du décompte décimal : soit en comptant sur les doigts différents des pouces, soit avec les trous entre les doigts. Ceci expliquerait pourquoi $9$ s'appelle \textit{neuf}, qui aurait alors été un nouveau chiffre.


\subsection{Base 16 --- numérotation hexadécimale}

Le système hexadécimal utilise la base 16 afin de rendre l'écriture encore plus compacte. Les seize caractères utilisés sont les dix chiffres arabes, auxquels sont rajoutés les six premières lettres de l'alphabet latin. Un caractère en numérotation hexadécimale est appelé \textit{nibble} et contient quatre bits. Un nibble est alors équivalent à la moitié d'un octet. Pour convertir un nombre binaire en nombre hexadécimal, il suffit de regrouper les bits par quatre.
\begin{equation*}
\left(1111100111\right)_2=\left(~\left(0011\right)~\left(1110\right)~\left(0111\right)~\right)_2=\left(3\textrm{E}7\right)_{16}=\left(999\right)_{10}
\end{equation*}

La compacité de la numérotation hexadécimale a été mise à profit en informatique: deux seuls caractères suffisent à couvrir les nombres décimaux de 0 à 255. C'est pour cette raison que le système de coloration RGB (\textit{Red Green Blue}) va jusqu'à 255, de même que les statistiques maximales au jeu \textit{Final Fantasy} (255 donne FF en numérotation hexadécimale).


\subsection{Conversion entre les bases}

La conversion de la base binaire aux bases octale et hexadécimale se fait en regroupant les bits par groupes de trois ou de quatre, respectivement. Le lien avec la base décimale ou tout autre base n'étant pas une puissance de deux est cependant moins direct: il faut alors passer par la méthode mathématique. Voici quelques exemples:

\begin{equation*}
\left(10101\right)_2=1\cdot2^4+1\cdot2^2+1\cdot2^0=16+4+1=21
\end{equation*}
\begin{equation*}
\left(25\right)_8=2\cdot8^1+5\cdot8^0=16+5=21
\end{equation*}
\begin{equation*}
77=64+8+4+1=1\cdot2^6+1\cdot2^3+1\cdot2^2+1\cdot2^0=\left(100111\right)_2
\end{equation*}
\begin{equation*}
178=2\cdot64+6\cdot8+2=2\cdot8^2+6\cdot8^1+2\cdot8^0=\left(262\right)_8.
\end{equation*}


\begin{table}[h]
\begin{center}
\begin{tabular}{|c|c|c|c|}
\hline
\multicolumn{4}{|c|}{Numérotations} \\
\hline
Binaire & Octale & Décimale & Hexadécimale \\
\hline
0 & 0 & 0 & 0 \\
\hline
1 & 1 & 1 & 1 \\
\hline
10 & 2 & 2 & 2 \\
\hline
11 & 3 & 3 & 3 \\
\hline
100 & 4 & 4 & 4 \\
\hline
101 & 5 & 5 & 5 \\
\hline
110 & 6 & 6 & 6 \\
\hline
111 & 7 & 7 & 7 \\
\hline
1000 & 10 & 8 & 8 \\
\hline
1001 & 11 & 9 & 9 \\
\hline
1010 & 12 & 10 & A \\
\hline
1011 & 13 & 11 & B \\
\hline
1100 & 14 & 12 & C \\
\hline
1101 & 15 & 13 & D \\
\hline
1110 & 16 & 14 & E \\
\hline
1111 & 17 & 15 & F \\
\hline
10000 & 20 & 16 & 10 \\
\hline
11111111 & 377 & 255 & FF \\
\hline
\end{tabular}
\end{center}
\caption{\label{bases-2-10}Équivalences entre les différentes numérotations.}
\end{table}

\subsection{Bit, multiples de bits et système international}

Comme il a été mentionné, un bit correspond à un caractère dans le système binaire. Ainsi, le bit est une unité de mesure de la quantité d'information. L'octet, qui vaut exactement huit bits, est une autre unité pour quantifier l'information. Plusieurs conventions sont présentement utilisées pour noter ces deux unités. Tout dépendamment des individus et des compagnies, le bit peut être noté \textit{bit} ou \textit{b}, alors que l'octet peut être noté \textit{o}, \textit{B} ou \textit{byte}. Toutes ces conventions peuvent devenir mélangeantes, surtout si on considère que \textit{b} et \textit{B} sont déjà deux unités du système international (respectivement le barn et le bel). De plus, le terme \textit{byte} ne réfère pas toujours exactement à huit bits ; à l'occasion il peut être associé à une quantité variant entre six et neuf bits.

Utiliser \textit{bit} pour bit et \textit{o} pour octet n'induit pas de confusion. D'ailleurs le terme \textit{octet} commence même à être utilisé dans certains pays non francophones pour éviter toute ambiguïté.

Les multiples du bit et de l'octet causent aussi quelques problèmes. En informatique, par exemple, on utilise couramment les termes \textit{kilobit (kbit)} et \textit{mégaoctet (Mo)}. Selon les définitions des préfixes du système international, ces deux termes devraient signifier $10^3$~bits et $10^6$~octets, respectivement. Or, en réalité, ils signifient plutôt $2^{10}$~bits et $2^{20}$~octets.

Pour résoudre ce problème, le système international propose d'utiliser des préfixes distincts pour les puissances de $10^3$ et les puissances de $2^{10}$ (voir tableau~\ref{prefixes-si}). Ainsi, au lieu de \textit{kilobit} et \textit{mégaoctet}, on utilisera \textit{kibibit} et \textit{mébioctet}, qui représentent correctement la quantité décrite.

\begin{table}[h]
\begin{center}
\begin{tabular}{|c|c|c|c|}
\hline
\multicolumn{2}{|c|}{Préfixes décimaux} & \multicolumn{2}{|c|}{Préfixes binaires} \\
\hline
Nom (symbole) & Valeur & Nom (symbole) & Valeur \\
\hline
kilo (k) & $\left(10^3\right)^1$ & kibi (Ki) & $\left(2^{10}\right)^1$ \\
\hline
méga (M) & $\left(10^3\right)^2$ & mébi (Mi) & $\left(2^{10}\right)^2$ \\
\hline
giga (G) & $\left(10^3\right)^3$ & gibi (Gi) & $\left(2^{10}\right)^3$ \\
\hline
téra (T) & $\left(10^3\right)^4$ & tébi (Ti) & $\left(2^{10}\right)^4$ \\
\hline
péta (P) & $\left(10^3\right)^5$ & pébi (Pi) & $\left(2^{10}\right)^5$ \\
\hline
exa (E) & $\left(10^3\right)^6$ & exbi (Ei) & $\left(2^{10}\right)^6$ \\
\hline
zetta (Z) & $\left(10^3\right)^7$ & zébi (Zi) & $\left(2^{10}\right)^7$ \\
\hline
yotta (Y) & $\left(10^3\right)^8$ & yobi (Yi) & $\left(2^{10}\right)^8$ \\
\hline
\end{tabular}
\end{center}
\caption{\label{prefixes-si}Préfixes SI pour les puissances de $10^3$ et de $2^{10}$.}
\end{table}

\section{Bascule asynchrone SR}

\begin{figure}[h]
\centering
\begin{circuitikz} \draw
(0,2) node[nor port](norR){}
(norR.in 1) to[short] (-1.5,2.28) node[left]{$R$}
(norR.in 2) to[short] (-2.5,1.72) to[short] (-2.5,-1) to[short] (0.5,-1) to[short] (0.5,0)
(norR.out) to[short] (1,2) node[right]{$Q$}
(0,0) node[nor port](norS){}
(norS.in 1) to[short] (-1.5,0.28) to[short] (-1.5,0.5) to[short] (0.5,1.75) to[short] (0.5,2)
(norS.in 2) to[short] (-1.5,-0.28) node[left]{$S$}
(norS.out) to[short] (1,0) node[right]{$\overline{Q}$}
;\end{circuitikz}
\caption{\label{sch-RS}Bascule asynchrone SR (\textit{latch} en anglais, parfois appelée \textit{flip-flop}) créée à partir de deux portes NON-OU. Ce circuit est bistable puisqu'il possède deux états stables dits \textit{set} et \textit{reset}, voir la table de vérité~\ref{table-RS}. Une \href{https://upload.wikimedia.org/wikipedia/commons/c/c6/R-S_mk2.gif}{animation} publiée sur le site Wikipédia par \href{https://commons.wikimedia.org/wiki/User:Napalm_Llama}{Napalm Llama} peut aider à visualiser la fonctionnement de cette bascule.}
\end{figure}

Jetons un {\oe}il rapide sur le fonctionnement de cette bascule. Tout d'abord, considérons le cas où l'entrée $R$ (\textit{reset}) est à \textit{un} et l'entrée $S$ à \textit{zéro}. Lorsque $R=1$, la sortie $Q$ est nécessairement à \textit{zéro}, comme nous l'indique la table de vérité de la porte NON-OU. Ceci implique aussi que les entrées de l'autre porte NON-OU (celle du bas) sont à \textit{zéro}, donc $\overline{Q}=1$ et la DEL s'allume. Puis, si $R$ est mis à \textit{zéro}, les sorties $Q$ et $\overline{Q}$ vont conserver leur état, c'est-à-dire $Q=0$ et $\overline{Q}=1$.

Considérons maintenant le cas où l'entrée $S$ (\textit{set}) est à \textit{un} et l'entrée $R$ est à \textit{zéro}. Lorsque $S=1$, la sortie $\overline{Q}$ est automatiquement à \textit{zéro}, ce qui implique que la sortie $Q$ devient \textit{un} et la DEL s'allume. Si de là on retourne à l'état $S=R=0$, l'état des sorties va à nouveau être conservé : elles vont donc être $Q=1$ et $\overline{Q}=0$. Ainsi, l'état des sorties lorsque $S=R=0$ ne dépend pas uniquement de l'état présent des entrées, mais aussi de l'état précédent des entrées.

Dans tous les cas considérés précédemment et résumés dans la table de vérité~\ref{table-RS}, les deux sorties ont toujours un comportement opposé : elles sont complémentaires. Autre constat : la mémoire de ce circuit est une \textit{mémoire vive}, elle est volatile. Elle sera effacée du moment qu'une des entrées $S$ ou $R$ sera mise à \textit{un}, ou encore lorsque l'alimentation du circuit sera interrompue.

\begin{table}
\centering
\begin{tabular}{|c|c|c|c|c|}
\cline{1-4}
\multicolumn{2}{|c|}{\textbf{Entrées}} & \multicolumn{2}{|c|}{\textbf{Sorties}} \\
\hline
$\mathbf{S}$ & $\mathbf{R}$ & $\mathbf{Q}$ & $\mathbf{\overline{Q}}$ & \textbf{Commentaires} \\
\hline
0 & 0 & $q$ & $\overline{q}$ & en mémoire \\
\hline
0 & 1 & 0 & 1 & mise à zéro (\textit{reset}) \\
\hline
1 & 0 & 1 & 0 & mise à un (\textit{set}) \\
\hline
\end{tabular}
\caption{\label{table-RS}Table de vérité partielle de la bascule asynchrone SR.}
\end{table}

\end{document}

Écrit par Jean-Raphaël Carrier
Dernière modification : 12 janvier 2014