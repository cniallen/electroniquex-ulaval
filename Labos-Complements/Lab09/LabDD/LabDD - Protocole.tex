%Créé par Annie-Claude Parent en collaboration avec Dominique Trottier-Beaulieu, Tigran Galstian et Claudine 
% dernière modification: 
%********************
%ToDo:
% - 

\RequirePackage[l2tabu, orthodox]{nag} %Check for obsolete commands
\documentclass[canadien,12pt,oneside,letterpaper]{article}
%
%-----------------------------------------------------
%Loading packages
%
\usepackage[utf8]{inputenc}
\usepackage[T1]{fontenc}
\usepackage[canadien]{babel}
\usepackage{lmodern}
\usepackage{textcomp}
\usepackage{amsmath,amssymb}
\usepackage{siunitx}
\usepackage{xcolor}
\usepackage[colorlinks=true,allcolors=blue]{hyperref}
\usepackage[all]{hypcap}
\usepackage{graphicx}
\usepackage{float}
\usepackage[americanvoltages,americancurrents,siunitx]{circuitikz}
\usetikzlibrary{babel}
\usepackage{caption}
\usepackage[letterpaper,headheight=15pt]{geometry}
\usepackage{fancyhdr}
\usepackage{setspace}
%
%----------------------------------------------------
%Other configurations and layout
%
\sisetup{separate-uncertainty}
\captionsetup{font=small,labelfont=bf,margin=0.1\textwidth}
\pagestyle{fancy}
\fancyhf{}
\lhead{\textsl{GPH-2006/PHY-2002~---~Atelier~IX}}
\rhead{\textsl{Page \thepage}}
\setcounter{secnumdepth}{0}
\setlength{\parskip}{1.5ex plus0.5ex minus0.2ex}
%\onehalfspacing
\interfootnotelinepenalty=10000 %To avoid footnotes spreading on several pages.
%
%---------------------------------------------------
%
\title{\textbf{Atelier IX}\\Énergie durable\thanks{Auteurs: Annie-Claude Parent, Dominique Trottier-Bealieu \& Tigran Galstian.}}
\renewcommand\footnotemark{}
\date{}

\begin{document}

\maketitle \vspace{-17ex}
\section{\textit{\textbf{Prélude de contexte et déroulement des ateliers}}}
\vspace{-2ex}
\textit{Les ateliers se rapprochent d’un mode de travail expérimental plus autonome,
s’éloignant du contexte pédagogique pour se rapprocher du milieu professionnel. En
effet, après avoir étudié tous les éléments de la section Lectures préparatoires, vous
devez concevoir votre propre protocole dans votre cahier de recherche afin d’atteindre
les objectifs de chaque partie. Les questions à explorer et la section Préparation vous
guideront dans votre rédaction du protocole, puis l’équipe d’enseignement vous assistera
au laboratoire avec toute explication souhaitée et des suggestions pour bonifier
vos manipulations. S’il reste du temps à la fin de chaque atelier, profitez-en pour
avancer votre projet de conception final afin de mieux respirer en fin de session et
de bénéficier directement de support de l’équipe.
Les équipes pour ces ateliers et le projet de conception passent de 2 à 4 personnes
pour partager les tâches de modélisation, conception et simulation en plus des manipulations
expérimentales, des mesures à prendre et de leur analyse. Vous pouvez
même utilisez des postes et plaquettes de montage en parallèle afin que personne ne
se tourne les pouces ! Consignez soigneusement tout votre travail dans votre cahier
de recherche et assurez-vous que le superviseur de l’équipe d’enseignement qui vous
encadre évalue si vous avez atteint les objectifs avant de démonter vos circuits. Il
n’y a aucune autre remise à faire pour les ateliers afin de vous donner le temps de
travailler sur votre projet final.}

\section{Thématique}\label{sec:thematique}
\vspace{-2ex}
Il existe une myriade de processus permettant de transformer l’énergie qui nous entoure en énergie consommable pour nos fins. Les sources à notre disposition sont nombreuses, que ce soit le soleil, le vent, l’eau, la chaleur du sol, etc. On les qualifie de “renouvelables”, car elles sont disponibles naturellement et n’ont pas à être générées d’une quelconque façon. Il n’y a qu’à déterminer les processus électroniques pour capter l'énergie à la source et en produire une énergie utilisable. Ce laboratoire d'énergie durable a pour objectif de transformer l'énergie d'une source lumineuse pour l'alimentation électrique d'un instrument, en l'occurrence une ampoule. L'expérience fait ainsi écho au cycle de l'énergie lumineuse, de la source (le soleil) à la borne (l'ampoule). Dans ce laboratoire, il s'agit d'explorer chacun des composants d'un système complet d'énergie durable, de les caractériser et de tester certains éléments causant de la variabilité dans le système. Les éléments principaux sont: la source lumineuse, le dispositif de captation (cellule photovoltaïque), la transformation et la régulation de l'énergie en électricité, le stockage en charge électrique, et l'alimentation et la consommation de l'instrument.

\section{Lectures préparatoires}\label{sec:lectures preparatoires}
\vspace{-2ex}
\begin{itemize}
\item complément \textit{Mesures lumineuses};
\item complément \textit{Batteries};
\end{itemize}

\section{Préparation}\label{sec:preparation}
\vspace{-2ex}
AVANT la séance d’atelier, écrivez sommairement dans votre cahier de recherche
tout ce que vous prévoyez faire au laboratoire pour atteindre les objectifs. N’oubliez
pas de calculer toute valeur ou modèle de référence donné en complément aux fins de
comparaison à l’expérience. En plus des manipulations expérimentales, des mesures
à prendre et de leur analyse, notez aussi les étapes de modélisation, conception et
simulation nécessaires. 


\section{Partie 1: Luminosité}\label{sec:partie1}
\vspace{-2ex}
Les objectifs à atteindre pour la caractérisation lumineuse sont:
\begin{itemize}
\item caractériser la puissance de la source lumineuse
\item caractériser la sortie du panneau solaire et mesurer son efficacité
\item analyser l'influence de la rotation du panneau solaire sur la sortie
\end{itemize}

%figure du montage: montrer les endroits où prendre des mesures et les positions du panneau solaire

\section{Questions à explorer dans le première partie}
\begin{itemize}
\item Quels sont les facteurs qui influencent l'efficacité d'un panneau solaire?
\item Comment la rotation du panneau solaire impacte-t-elle la sortie électrique du panneau solaire?
\end{itemize}

\section{Partie 2: Entreposage de l'énergie}\label{sec:partie2}
\vspace{-2ex}
Les objectifs à atteindre pour l'entreposage de l'énergie sont:
\begin{itemize}
\item caractériser la charge et la décharge de la batterie
\item analyser l'influence du maintien de la charge de la batterie en fonction de la température
\item caractériser la consommation de l'instrument
\end{itemize}

\section{Questions à explorer dans le deuxième partie}
\begin{itemize}
\item Quelle est la courbe de charge/décharge d'une batterie: en condition optimale? selon l'intensité lumineuse (moment de la journée)? selon la température de (chaud, froid)?
\item Quelle est la perte énergétique entre la source lumineuse et l'ampoule? 
\end{itemize}


\end{document}

