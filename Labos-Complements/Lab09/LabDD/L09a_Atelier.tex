%Créé par Annie-Claude Parent en collaboration avec Dominique Trottier-Beaulieu, Tigran Galstian et Claudine 
% dernière modification: 
%********************
%ToDo:
% - 

\RequirePackage[l2tabu, orthodox]{nag} %Check for obsolete commands
\documentclass[canadien,12pt,oneside,letterpaper]{article}
%
%-----------------------------------------------------
%Loading packages
%
\usepackage[utf8]{inputenc}
\usepackage[T1]{fontenc}
\usepackage[canadien]{babel}
\usepackage{lmodern}
\usepackage{textcomp}
\usepackage{amsmath,amssymb}
\usepackage{siunitx}
\usepackage{xcolor}
\usepackage[colorlinks=true,allcolors=blue]{hyperref}
\usepackage[all]{hypcap}
\usepackage{graphicx}
\usepackage{float}
\usepackage[americanvoltages,americancurrents,siunitx]{circuitikz}
\usetikzlibrary{babel}
\usepackage{caption}
\usepackage[letterpaper,headheight=15pt]{geometry}
\usepackage{fancyhdr}
\usepackage{setspace}
%
%----------------------------------------------------
%Other configurations and layout
%
\sisetup{separate-uncertainty}
\captionsetup{font=small,labelfont=bf,margin=0.1\textwidth}
\pagestyle{fancy}
\fancyhf{}
\lhead{\textsl{GPH-2006/PHY-2002~---~Atelier~IX}}
\rhead{\textsl{Page \thepage}}
\setcounter{secnumdepth}{0}
\setlength{\parskip}{1.5ex plus0.5ex minus0.2ex}
%\onehalfspacing
\interfootnotelinepenalty=10000 %To avoid footnotes spreading on several pages.
%
%---------------------------------------------------
%
\title{\textbf{Atelier IX}\\Énergie durable\thanks{Auteurs: Nicolas Payeur, Annie-Claude Parent, Dominique Trottier-Beaulieu, Claudine Allen \& Tigran Galstian.}}
\renewcommand\footnotemark{}
\date{}

\begin{document}

\maketitle \vspace{-17ex}
\section{\textit{\textbf{Prélude de contexte et déroulement des ateliers}}}
\vspace{-2ex}
Les ateliers se rapprochent d’un mode de travail expérimental plus autonome,
s’éloignant du contexte pédagogique pour se rapprocher du milieu professionnel. Les questions à explorer et la section Préparation vous guideront dans la rédaction de votre propre protocole, puis l’équipe d’enseignement vous assistera au laboratoire avec toute explication souhaitée et des suggestions pour bonifier vos manipulations. 

Les équipes pour ces ateliers passent de 2 à 4 personnes pour partager les tâches de modélisation, de conception et de simulation en plus des manipulations expérimentales, des mesures à prendre et de leur analyse. Vous pouvez même utiliser des postes et plaquettes de montage en parallèle afin que personne ne se tourne les pouces ! Consignez soigneusement tout votre travail dans votre cahier
de recherche et assurez-vous que le superviseur de l’équipe d’enseignement qui vous
encadre évalue si vous avez atteint les objectifs avant de démonter vos circuits. Il
n’y a aucune autre remise à faire pour les ateliers.

S’il reste du temps à la fin de chaque atelier, profitez-en pour
avancer votre projet de conception final afin de mieux respirer en fin de session et
de bénéficier directement de support de l’équipe.
\newpage
\section{Thématique}\label{sec:thematique}
\vspace{-2ex}
Il existe une myriade de processus permettant de transformer l’énergie qui nous entoure en énergie consommable pour nos fins. Les sources à notre disposition sont nombreuses, que ce soit le soleil, le vent, l’eau, la chaleur du sol, etc. On les qualifie de “renouvelables”, car elles sont disponibles naturellement et n’ont pas à être générées d’une quelconque façon. Il n’y a qu’à déterminer les processus électroniques pour capter l'énergie à la source et en produire une énergie utilisable. Ce laboratoire d'énergie durable a pour objectif de transformer l'énergie d'une source lumineuse pour l'alimentation électrique d'un instrument, en l'occurrence une ampoule. L'expérience fait ainsi écho au cycle de l'énergie lumineuse, de la source (le soleil) à la borne (l'ampoule). D'autres instruments sont proposés pour comprendre le rendement électrique d'un système, par exemple: un chargeur, un ventilateur ou encore une radio. Comme l'énergie disponible n'est pas constante, celle-ci doit être entreposée dans une batterie qui libérera à son tour l'énergie selon la demande. Bien entendu, plusieurs facteurs environnants sont sources de variabilité, notamment la température, l'angle d'incidence des rayons, la consommation fluctuante de l'instrument, etc.

Dans ce laboratoire, il s'agit d'explorer chacun des composants d'un système complet d'énergie durable, de les caractériser et de tester certains éléments causant de la variabilité dans le système. Les éléments principaux sont: la source lumineuse, le dispositif de captation (cellule photovoltaïque), la transformation et la régulation de l'énergie en électricité, le stockage en charge électrique, l'alimentation et la consommation de l'instrument.

\section{Lectures préparatoires}\label{sec:lectures preparatoires}
\vspace{-2ex}
\begin{itemize}
\item complément \textit{Mesures lumineuses};
\item complément \textit{Batteries};
\item fiche technique de la pile.
\end{itemize}

\section{Préparation}\label{sec:preparation}
\vspace{-2ex}
AVANT la séance d’atelier, écrivez sommairement dans votre cahier de recherche
tout ce que vous prévoyez faire au laboratoire pour atteindre les objectifs. N’oubliez
pas de calculer toute valeur ou modèle de référence donné en complément aux fins de
comparaison à l’expérience. En plus des manipulations expérimentales, des mesures
à prendre et de leur analyse, notez aussi les étapes de modélisation, conception et
simulation nécessaires. 


\section{Partie 1: Rendement d'un système d'énergie durable}\label{sec:partie1}
\vspace{-2ex}

Cette partie vise à étudier la puissance fournie par un panneau solaire éclairé et les effets des conditions non idéales qui sont forcément présentes lors d'utilisation réelle. Il vous sera fourni une boite contenant un panneau solaire pouvant être pivoté et une lampe DEL. Vos mesures devraient vous permettre de qualifier le rendement des panneaux solaires et l'énergie qu'ils peuvent produire de façon réaliste. La figure \ref{fig:mesures panneau} montre les branchements à effectuer pour ce montage. Vous pouvez vous référer aux expériences déjà faites qui contenaient des objectifs similaires.


%todo: demander le w/m2 de la lampe

 
Les objectifs à atteindre pour évaluer le rendement du système sont:
\begin{itemize}
\item Obtenir les courbes i-v et p-v du panneau solaire selon l'angle incident de la lumière sur le panneau. Votre montage permet de créer un angle de 0 à 60 degrées.
\item Obtenir les courbes i-v et p-v du panneau solaire avec et sans nuages (modélisé par un filtre 50\%).
\item Déterminer le rendement optimal du panneau solaire
\item Déterminer la charge maximisant le transfert de puissance dans divers scénarios.
\end{itemize}

%figure du montage: montrer les endroits où prendre des mesures et les positions du panneau solaire


\begin{figure}[h]
    \centering
    \begin{circuitikz} \draw
    %(0,0) node[ground]{} 
    (0,3) to[V] (0,0)  
    (0,3) to[short] 
    (2,3) to[ammeter] (5,3)
    (5,0) to[pR,n=pR] (5,3)
    (4,0) to[short] (0,0)
    (4,0) to[short] (4,1.5)
    (4,1.5) to[short, n=pR] (4.5,1.5)
    (2,0) to[voltmeter] (2,3)
    node[label=right:0-10k$\Omega$] at (5,1.5)
    node[rotate=90] at (-0.75,1.5) {Panneau solaire}
    ;\end{circuitikz}
    \caption{Circuit pour l'obtention des courbes i-v et p-v du panneau solaire}
    \label{fig:mesures panneau}
\end{figure}

Vous aurez à prendre beaucoup de mesures, alors un script labVIEW vous est fourni. Celui-ci moyenne le voltage, le courant et la puissance et les ajoute à la suite des autres valeurs dans un fichier .xlsx spécifié à chaque fois que le VI est exécuté. Il vous sera facile de créer les courbes ensuite. Prenez un DAQ comme voltmètre et le multimètre 6$\frac{1}{2}$ chiffres comme ampèremètre.


\section{Questions à explorer dans la première partie}
\begin{itemize}
\item Quels sont les facteurs qui influencent l'efficacité d'un panneau solaire? 
\item Comment la rotation du panneau solaire impacte-t-elle la sortie électrique du panneau solaire?
\end{itemize}

\section{Partie 2: Entreposage de l'énergie}\label{sec:partie2}
\vspace{-2ex}
Lorsqu'on parle de panneau solaire comme solution énergétique, on parle également d'entreposage de cette énergie, donc de batteries. Celles-ci, comme expliqué dans le complément \textit{Batteries}, ont un fonctionnement impliquant plusieurs considérations importantes. Vous devrez dans cet atelier analyser une d'elles: les courbes de charge et de décharge. Vous devez concevoir un script permettant au arduino de votre coffre de mesurer la tension aux bornes de la pile fournie lors de sa décharge et de sa recharge. La recharge se fait directement à partir de la sortie 3 ou 5 volts de votre arduino, alors que la décharge nécessite de faire passer un courant dans une résistance.

Les objectifs à atteindre pour l'entreposage de l'énergie sont:
\begin{itemize}
\item caractériser la charge de la pile selon plusieurs tensions de charge
\item caractériser la décharge de la pile selon plusieurs courants fournis 
\end{itemize}

\section{Questions à explorer dans le deuxième partie}
\begin{itemize}
\item Quelle tension permet la charge de la pile la plus rapide ? et la plus efficace ?
\item Quel courant permet la meilleure efficacité énergétique de la pile ?
\end{itemize}


\end{document}

