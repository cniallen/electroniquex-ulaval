\documentclass[12pt,oneside,letterpaper]{article}

\usepackage[canadien]{babel}
\usepackage[utf8]{inputenc}
\usepackage[T1]{fontenc}
\usepackage{lmodern}
\usepackage{graphicx}
\usepackage[letterpaper]{geometry}
\usepackage[americanvoltages,americancurrents, siunitx]{circuitikz}
\usetikzlibrary{babel}
\usepackage{amsmath}
\usepackage{caption}
\usepackage{subfig}
\usepackage{hyperref}
\usepackage[all]{hypcap}


\captionsetup{font=small,labelfont=bf,margin=0.1\textwidth}
\pagestyle{myheadings}
\markboth{GPH-2006/PHY-2002~---~Analyse~fréquentielle}{GPH-2006/PHY-2002~---~Analyse~fréquentielle}


\begin{document}


\title{\textbf{Complément}\\Mesure de la lumière}
\author{Annie-Claude Parent}
\date{}
\maketitle


\section{Intensité d'une source lumineuse}

Dans tout système instrumental utilisant une énergie, la source de cette énergie doit être quantifiée. Dans un système à énergie solaire, la luminosité réelle de la source doit être mesurée. 

\begin{itemize}
\item L'\textit{éclairement lumineux} est une mesure de la façon dont l'humain perçoit la lumière sur une surface et se mesure en lux. Par exemple, la lumière du soleil varie de 5 à 120 000 lux. Un lux est l'énergie produite par un lumen incident sur une surface de 1 m\textsuperscript{2}. 
\item La \textit{luminosité} est la quantité totale d'énergie émise par unité de temps. Elle représente la brillance d'un objet et se mesure en lumens (lm). 
\item La \textit{puissance} est la quantité d'énergie fournie par unité de temps et s'exprime en watts $W$. 
\end{itemize}

Lorsque la source est une ampoule à incandescence, l’indicateur utilisé pour la quantifier est la puissance, exprimée en watts (W), par exemple : 20 W, 60 W ou 100 W. Dans le cas d’une ampoule LED, sa puissance peut aussi être exprimée en watts mais n’est qu’une approximation de sa luminosité réelle. Contrairement à une ampoule à incandescence, une ampoule LED peut produire une luminosité différente avec la même consommation électrique. Ce sont les composants, de qualités différentes, qui font en sorte qu’une ampoule LED a une meilleure efficacité qu’une autre. 
Une source lumineuse émet un flux lumineux (ΦV), qui s’exprime en lumens (lm). Les lumens sont utilisés pour comparer la luminosité de diverses sources lumineuses indépendamment de leur efficacité et des composants. L'efficacité (η) est exprimée en lumens par watts (lm/W). La puissance en watts (W) se calcule comme suit : 
\begin{equation}
P_{(W)}=\phi_{V(lm)}/η_{(lm/W)}		
\end{equation}
Il est possible de se référer aux tableaux suivants pour estimer la puissance d’une source lumineuse. 
Tableau 1. Table d’efficacité lumineuse
Tableau 2. Table de Lumens en Watts





\end{document}

