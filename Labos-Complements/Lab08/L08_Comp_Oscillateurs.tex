%Créé par Claudine Allen en collaboration avec Jérémie Guilbert
%Dernière modification JG: 11 novembre 2020
%Dernière modification CA: 15 novembre 2020
%
\RequirePackage[l2tabu, orthodox]{nag} %Check for obsolete commands
\documentclass[canadien,12pt,oneside,letterpaper]{article}
%
%-----------------------------------------------------
%Loading packages
%
\usepackage[utf8]{inputenc}
\usepackage[T1]{fontenc}
\usepackage[canadien]{babel}
\usepackage{lmodern}
\usepackage{textcomp}
\usepackage{amsmath,amssymb}
\usepackage{siunitx}
\usepackage[svgnames]{xcolor}
\usepackage{hyperref}
\usepackage[all]{hypcap}
\usepackage{graphicx}
\usepackage[nooldvoltagedirection,americanvoltages,americancurrents,siunitx]{circuitikz}
\usetikzlibrary{babel}
\usepackage{pgfplots}
\pgfplotsset{compat=1.15}
\usepackage{mathrsfs}
\usetikzlibrary{arrows}
\usepackage{caption}
%\usepackage{subcaption}
\usepackage{subfig}
\usepackage[letterpaper,headheight=15pt]{geometry}
\usepackage{fancyhdr}
\usepackage{setspace}
%
\captionsetup{font=small,labelfont=bf,margin=0.1\textwidth}
\pagestyle{myheadings}
\markboth{PHY-2002/GPH-2006~---~Oscillateurs électroniques}{PHY-2002/GPH-2006~---~Oscillateurs électroniques}

\begin{document}
 
\title{\textbf{Complément}\\Oscillateurs électroniques}
\author{Jérémie Guilbert \& Claudine Allen}
\date{}
\maketitle

% En plus d'atténuer les signaux alternatifs dans une bande de fréquences donnée, les filtres peuvent être utilisés en combinaison avec des amplificateurs opérationnels afin de générer activement des signaux oscillants à des fréquences voulues. La génération de ces signaux fait intervenir la fonction de comparateur d'un ampli-op afin de contrôler le cycle de charge-décharge d'un condensateur, puis celle d'amplification avec rétroaction afin de sélectionner la fréquence d'oscillation avec un filtre passe-bande. Des fonctions d’intérêt pour les circuits oscillateurs étudiés sont d’actionner un transducteur afin d’obtenir des ondes mécaniques, ou encore de définir un étalon de temps lorsque l’oscillation est pratiquement harmonique avec une seule fréquence.

\section{Oscillateur à relaxation avec comparateur}
\begin{figure}[h]
\centering
\begin{circuitikz} \draw
(0,0) node[op amp](opamp){}
(opamp.+) to[short] (-1.2,-0.5) to[short] (-1.2,-2.2)
(opamp.-) to[short] (-1.2,0.5) to[short] (-1.2,2.2)
(opamp.out) to[short,-o] (2.1,0) node[right]{$v_{\mathrm{out}}$}
(opamp.down) ++ (0,-0.5) node[below]{$-5$~V} -- (opamp.down)
(opamp.up) ++ (0,0.5) node[above]{5~V} -- (opamp.up)
(-4,2.2) node[ground]{} to[C=$C$] (-1.2,2.2) to[R=$R$] (1.6,2.2) to[short] (1.6,-2.2) to[R=10~k$\Omega$] (-1.2,-2.2) to[R=10~k$\Omega$] (-4,-2.2) node[ground]{}
;\end{circuitikz}
\caption{\label{sch-osc-relax}Oscillateur à relaxation avec comparateur}
\end{figure}
%Ancienne légende de la figure 1
Le circuit illustré à la figure \ref{sch-osc-relax}
est basé sur l'utilisation de la fonction comparateur de l'ampli-op qui permet de contrôler le cycle de charge-décharge d'un condensateur afin de générer des oscillations à une fréquence déterminée par les valeurs des composants du circuit. Le condensateur qui se charge cause éventuellement l'entrée inverseuse de l'amplificateur opérationnel à dépasser celle non inverseuse et la tension de sortie de ce comparateur tombe alors négative, renversant ainsi la polarité appliquée sur le condensateur. Celui-ci se décharge maintenant jusqu'à ce que l'entrée inverseuse repasse sous celle non inverseuse et remet la sortie positive pour recommencer le cycle. 

\section{Oscillateur harmonique à rétroaction - pont de Wien}
\begin{figure}[h]
\centering
\begin{circuitikz} \draw
(0,0) node[op amp](opamp){}
(opamp.+) to[short] (-1.2,-0.5) to[short] (-1.2,-2.2)
(opamp.-) to[short] (-1.2,0.5) to[short] (-1.2,2.2)
(opamp.out) to[short,-o] (2.1,0) node[right]{$v_{\mathrm{out}}$}
(opamp.down) ++ (0,-0.5) node[below]{$-5$~V} -- (opamp.down)
(opamp.up) ++ (0,0.5) node[above]{5~V} -- (opamp.up)
(-4,2.2) node[ground]{} to[R=1~k$\Omega$] (-1.2,2.2) to[R=10~k$\Omega$] (1.6,2.2) to[short] (1.6,0) to[C=$C$] (1.6,-2.2) to[R=$R$] (-1.2,-2.2)
(-1.2,-2.2) to[C=$C$] (-3,-2.2) node[ground]{} to[R=$R$] (-3,-0.5) to[short] (-1.2,-0.5)
;\end{circuitikz}
\caption{\label{sch-osc-Wien}Oscillateur harmonique à rétroaction - pont de Wien}
\end{figure}
Le circuit à la figure \ref{sch-osc-Wien} ci-dessus utilise l'ampli-op en mode rétroaction afin de sélectionner la fréquence du signal qui entrera à l'entrée non inverseuse au moyen d'un filtre passe-bande, permettant par le fait-même de contrôler la fréquence des oscillations à la sortie.

Le signal à l'entrée non inverseuse, qui n'est \textit{a priori} que du bruit car aucune source n'est présente, est amplifié puis passe au travers d'un pont de Wien qui le filtre. Le signal ainsi filtré retourne ensuite à l'entrée de l'amplificateur pour être à nouveau amplifié et filtré, et ainsi de suite, \textit{ad infinitum}. Le pont de Wien est \textit{grosso modo} l'addition d'un filtre passe-bas et d'un filtre passe-haut pour former un filtre passe-bande. La fréquence du signal de sortie est alors $f=\left(2\,\pi\,R\,C\right)^{-1}$. L'amplitude finale du signal est fixée par la source d'alimentation de l'ampli-op. % Cet oscillateur doit son oscillation au chargement/déchargement du condensateur qui relie l'entrée non inverseuse de l'amplificateur au \textit{ground}.

\section{Oscillateur harmonique à rétroaction (LC)}
%Ancienne intro partie 3
Un autre oscillateur reposant sur le même principe de filtrage et amplification en boucle du signal est l'oscillateur LC. Vous avez étudié au dernier atelier le filtre RLC passe-bande dont le gain et la fréquence de résonance peuvent être déterminés analytiquement à partir des valeurs de résistance $R$, d'inductance $L$ et de capacitance $C$ du circuit. On peut donc remplacer le pont de Wien du dernier oscillateur par un circuit RLC afin d'obtenir, avec un choix judicieux des composants du circuit, une fréquence de résonance et un gain quelconques. 

\subsection{Résonance et équation différentielle des oscillateurs}
On peut déterminer la fréquence de résonance d'un circuit oscillatoire passif en déterminant l'équation différentielle linéaire décrivant la tension ou le courant du circuit au point d'intérêt. Pour ce faire, le principe demeure le même que l'analyse de tout circuit réactif tel que modélisé précédemment à l'aide de la transformation de Laplace, mais ici l'accent est mis sur l'équation différentielle ordinaire (EDO) de départ afin de mieux faire ressortir les liens avec la théorie étudiée dans vos cours de physique mathématique. Les étapes générales d'analyse sont donc les suivantes:
\renewcommand{\labelenumi}{\Roman{enumi}.}
\begin{enumerate}
    \item déterminer le modèle gouvernant le circuit à partir des lois de Kirchhoff,
    \item substituer les relations constitutives tension/courant pour chacun des composants du circuit, soit
    \begin{align}
        \text{résistance : } v_R(t) &= R\,i(t) \quad\text{ou}\quad i_R(t)=\frac{v_R(t)}{i_R(t)}\,,\\
        \text{condensateur : } i_C(t) &= C\,\frac{\mathrm{d}v(t)}{\mathrm{d}t} \quad\text{ou}\quad v_C(t) = v(0)+\frac{1}{C}\int_0^ti(t')\,dt'\,,\\
        \text{bobine : } v_L(t) &= L\,\frac{\mathrm{d}i(t)}{\mathrm{d}t} \quad\text{ou}\quad i_L(t)  = i(0)+\frac{1}{L}\int_0^tv(t')\,dt'\,,
    \end{align}
    \item exprimer l'équation différentielle en termes de la variable d'intérêt $i(t)$ ou $v(t)$ au point désiré et résoudre l'équation.
\end{enumerate}

\begin{figure}[h]
\centering
\begin{circuitikz} \draw
(0,0) to[V=$V_S$,invert] (0,3) to[C=$C$] (3,3) to[R=$R$,i>^=$I$] (3,0) to[L=$L$] (0,0);
\end{circuitikz}
\caption{Circuit RLC en série.}
\label{circuitRLC-serie}
\end{figure}

L'oscillateur le plus simple est le circuit RLC en série de la figure \ref{circuitRLC-serie}. En effet, le comportement transitoire du courant $i(t)$ dans l'unique maille s'analyse ainsi:
\begin{enumerate}
    \item $v_S(t)=v_C(t)+v_R(t)+v_L(t)$ avec la loi des mailles de Kirchhoff,
    \item $\displaystyle v_S(t)=v(0)+\frac{1}{C}\int_0^ti(t')\,dt'+Ri(t)+L\frac{\mathrm{d}i(t)}{\mathrm{d}t}$ avec les relations constitutives,
%\end{enumerate}
%Si on suppose maintenant que la source produit une tension constante, on a, en prenant la dérivée de l'équation précédente, 
%\begin{enumerate}
%  \setcounter{enumi}{2}
  \item $\displaystyle \frac{\mathrm{d}v_S(t)}{\mathrm{d}t} = L\,\frac{\mathrm{d}^2i(t)}{\mathrm{d}t^2}+R\,\frac{\mathrm{d}i(t)}{\mathrm{d}t}+\frac{1}{C}\,i(t)$ en prenant la dérivée.
\end{enumerate}
En exprimant cette équation sous forme canonique $\ddot y(t) + 2\alpha\,\dot y(t) + \omega_0^2\,y(t) = f(t)$ avec un coefficient unitaire du terme en dérivée seconde, on a bien une EDO d'ordre~2 \textit{non homogène}
\begin{equation}\label{eq:RCL}
    \frac{\mathrm{d}^2i(t)}{\mathrm{d}t^2}+\frac{R}{L}\,\frac{\mathrm{d}i(t)}{\mathrm{d}t}+\frac{1}{LC}\,i(t) = \frac{\mathrm{d}v_S(t)}{\mathrm{d}t}\,.
\end{equation}

\begin{figure}[h]
    \centering
    \includegraphics[width=0.55\textwidth]{Labos-Complements/Lab08/masse_ressort.png}
    \caption{Système masse-ressort sur une surface sans friction dans un milieu visqueux. Trois forces agissent sur la masse $m$ au bout du ressort: la force de rappel dictée par la constante de rappel du ressort~$k$, une force de frottement fluide proportionnelle à la vitesse de la masse $v(t)=\dot x(t)$, dont le coefficient de proportionnalité est l'amortissement~$c$, ainsi qu'une force extérieure~$F$.}
    \label{fig:systeme_masse_ressort}
\end{figure}
Pour mieux visualiser la signification de chacun des termes de cette équation, il est utile d'utiliser une analogie avec un système plus intuitif, le système masse-ressort illustré à la figure~\ref{fig:systeme_masse_ressort}. À partir des forces identifiées sur la figure, l'équation du mouvement pour un objet au bout d'un ressort dans un milieu visqueux est 
\begin{equation}\label{eq:masse_ressort}
    %\begin{split}
      &\sum F = m\frac{d^2x}{dt^2} = -c\frac{dx}{dt}-kx+F,% \\
     % &\Rightarrow \frac{d^2x}{dt^2}+\frac{c}{m}\frac{dx}{dt}+\frac{k}{m}x = F.
    %\end{split}
\end{equation}
avec la forme canonique 
S'il n'y a pas de force $F$ qui entretient le mouvement, l'équation \ref{eq:masse_ressort} a la même forme que l'équation pour le courant du circuit RLC avec une source constante. Cette équation, exprimée sous la forme canonique
\begin{equation}\label{eq:canonique}
    \frac{d^2I(t)}{dt^2}+2\alpha\frac{dI(t)}{dt}+\omega_0^2I(t) = 0,
\end{equation}
possède la solution générale
\begin{equation}\label{eq:sol_gen}
   I = Ae^{-\alpha t}\sin\left(\sqrt{\omega_0^2-\alpha^2}t+\phi\right), 
\end{equation}
où $A$ et $\phi$ sont des constantes déterminées par les conditions initiales du système, dans le cas où $\frac{\alpha}{\omega_0}<1$. Cette équation représente l'oscillation de la variable $I$ à la fréquence $\sqrt{\omega_0^2-\alpha^2}$. L'amplitude de l'oscillation, initialement de $A$, décroît de façon exponentielle avec le temps à un taux dicté par $\alpha$. En comparant terme à terme les équations \ref{eq:RCL}, \ref{eq:masse_ressort} et \ref{eq:canonique} avec cette solution, on observe une équivalence entre les termes résumée au tableau \ref{table-equiv_masse_ressort}.  

\begin{table}[h]
\centering
\begin{tabular}{c|ccc}
\hline
Interprétation physique & Amortissement & Oscillation & Terme d'entretien \\
Terme associé & $\frac{dI(t)}{dt}$ & $I(t)$ & \\
\hline
\textbf{Équation canonique} & $2\alpha$ & $\omega_0$ & ---\\
\textbf{Équation masse-ressort} & $\frac{c}{m}$ & $\sqrt{\frac{k}{m}}$ & $F$ \\
\textbf{Équation circuit RLC} & $\frac{R}{L}$ & $\frac{1}{\sqrt{LC}}$ & $\frac{dV_S}{dt}$ \\
\hline
\end{tabular}
\caption{\label{table-equiv_masse_ressort}Équivalences entre les termes de l'équation différentielle du système masse-ressort et celle du circuit RLC.}
\end{table}
Le tableau montre qu'on peut associer la résistance $R$ du circuit RLC au coefficient d'amortissement fluide $c$, celui-ci générant une force de frottement qui s'oppose au mouvement de la masse de façon proportionnelle à sa vitesse, semblable à la résistance qui crée une opposition au courant proportionnelle à sa dérivée temporelle. De même, les paramètres $L$ et $C$ peuvent être associés à la masse $m$ et à l'inverse de la constante de rappel du ressort $k$ respectivement. En l'absence de résistance, le système sera donc non-amorti et oscillera à la fréquence $\omega_0=\frac{1}{\sqrt{LC}}$. Si la résistance est non-nulle, la fréquence d'oscillation sera amortie et sera plutôt de $\sqrt{w_0^2-\alpha^2}=\sqrt{\frac{1}{LC}-\left(\frac{R}{L}\right)^2}$.  

La résolution de l'équation homogène décrite ici permet de trouver la fréquence de résonance d'un circuit. Pour déterminer la réponse en fréquence, c'est-à-dire la valeur du gain en fonction de la fréquence d'oscillation du signal d'entrée $V_S(t)$, il faut résoudre l'équation non-homogène. Ceci est habituellement réalisé en calculant la fonction de transfert du système directement dans l'espace des fréquences avec les expressions pour l'impédance des composants, tel que détaillé dans le complément \textit{Analyse fréquentielle}.

Noter qu'à partir de la solution dérivée pour le courant circulant dans le circuit RLC en série, on pourrait obtenir obtenir des équations pour la tension aux bornes de chacun des composants à partir des relations tension/courant. Par exemple, si on désire connecter une charge sur la bobine, on pourrait déterminer la tension à l'entrée de la charge en connaissant $I(t)$ et sachant que $V_L(t) = L\frac{dI(t)}{dt}$.

\section{Oscillateur avec la puce 555}
%Ancienne sous-section
\begin{figure}[h]
\centering
\begin{circuitikz} \draw[thick]
(0,0) to[short] (0,0.5) node[right]{4} to[short,*-*] (0,1.5) node[right]{3} to[short] (0,2.5) node[right]{2} to[short,*-*] (0,3.5) node[right]{1} to[short] (0,4) to[short] (3,4) to[short] (3,3.5) node[left]{8} to[short,*-*] (3,2.5) node[left]{7} to[short] (3,1.5) node[left]{6} to[short,*-*] (3,0.5) node[left]{5} to[short] (3,0) to[short] (0,0)
{[anchor=east] (0,3.5) node{\textbf{GND}} (0,2.5) node{\textbf{TRIG}} (0,1.5) node{\textbf{v$_{\text{out}}$}} (0,0.5) node{\textbf{RESET}}}
{[anchor=west] (3,3.5) node{\textbf{v$_{\text{cc}}$}} (3,2.5) node{\textbf{DISCH}} (3,1.5) node{\textbf{THRES}} (3,0.5) node{\textbf{CONT}}}
;\end{circuitikz}
\caption{\label{sch-555}Schéma des ports d'entrée et sortie d'une puce 555.}
\end{figure}
La puce 555 est un circuit intégré (figure \ref{sch-555}) souvent utilisé en électronique analogique pour, entre autres, bâtir des oscillateurs. Les niveaux de déclenchement (\textit{trigger}) et de seuil (\textit{threshold}) valent respectivement un tiers et deux tiers de l'alimentation \textbf{v$_{\text{cc}}$}. Lorsque l'entrée \textbf{TRIG} est inférieure au niveau de déclenchement, la sortie de la puce \textbf{v$_{\text{out}}$} vaut \textbf{v$_{\text{cc}}$}. Lorsque l'entrée \textbf{TRIG} est supérieure au niveau de déclenchement et qu'en plus l'entrée \textbf{THRES} est supérieure au niveau de seuil, alors la sortie devient nulle. Lorsque la sortie devient nulle, un court-circuit se fait entre l'entrée \textbf{DISCH} (\textit{discharge}) et la mise à la terre. La table \ref{table-555} résume ce fonctionnement.

\begin{table}[h]
\centering
\begin{tabular}{|c|c|c|}
\hline
\textbf{TRIG} & \textbf{THRES} & \textbf{OUT} \\
\hline
$<\frac{1}{3}$\textbf{v$_{\text{cc}}$} & --- & \textbf{v$_{\text{cc}}$} \\
\hline
$>\frac{1}{3}$\textbf{v$_{\text{cc}}$} & $>\frac{2}{3}$\textbf{v$_{\text{cc}}$} & 0 \\
\hline
$>\frac{1}{3}$\textbf{v$_{\text{cc}}$} & $<\frac{2}{3}$\textbf{v$_{\text{cc}}$} & Conserve l'état \\
\hline
\end{tabular}
\caption{\label{table-555}Sortie d'une puce 555 en fonction des tensions aux entrées \textbf{TRIG} et \textbf{THRES}.}
\end{table}

La figure~\ref{sch-alarme-1} illustre comment utiliser la puce 555 pour générer un signal oscillant à une fréquence précise. Initialement, lorsque le bloc d'alimentation s'allume, la tension à l'entrée \textbf{TRIG} est basse, donc la sortie est élevée ($v_{\mathrm{out}}=v_{\mathrm{s}}=\mathbf{v_{\text{cc}}}$) et le condensateur de capacité $C$ se charge avec une constante de temps $\tau=\left(R_A+R_B\right)\,C$. La tension aux bornes du condensateur, \textit{i.e.} la tension aux terminaux \textbf{TRIG} et \textbf{THRES}, augmente jusqu'à atteindre le niveau de seuil, soit $\frac{2}{3}\,v_{\mathrm{s}}$. À ce moment, la sortie devient basse ($v_{\mathrm{out}}=0$~V), le terminal \textbf{DISCH} devient mis à la terre et le condensateur se décharge avec une constante de temps plus petite $\tau=R_B\,C$. Lorsque la différence de potentiel aux bornes du condensateur a diminué jusqu'à atteindre $\frac{1}{3}\,v_{\mathrm{s}}$, alors la sortie redevient élevée, le terminal \textbf{DISCH} est déconnecté de la masse et le condensateur recommence à se charger. Ainsi, la sortie passe de 0~V à $v_{\mathrm{s}}$ puis retourne à 0~V et ainsi de suite, avec une période de $T=\mathrm{ln}\!\left(2\right)\left(R_A+2\,R_B\right)C$\label{eq:alarme}.

\begin{figure}[h]
\centering
\begin{circuitikz} \draw[thick]
(0,0) to[short,-*] (0,1) node[right]{2} to[short] (0,2) node[right]{6} to[short,*-*] (0,3) node[right]{7} to[short] (0,4) to[short] (1,4) node[below]{4} to[short,*-*] (2,4) node[below]{8} to[short] (3,4) to[short] (3,3) node[left]{3} to[short,*-*] (3,1) node[left]{5} to[short] (3,0) to[short,-*] (1.5,0) node[above]{1} to[short] (0,0)
;\draw
(3,3) to[short,-o] (3.5,3) node[right]{$v_{\mathrm{out}}$}
(1.5,0) to[short] (1.5,-0.5)
(-2,-0.5) to[C=$C$] (-2,1) to[R=$R_B$] (-2,3) to[R=$R_A$] (-2,5) to[short] (2,5)
(-2,1) to[short] (0,1)
(-1,1) to[short] (-1,2) to[short] (0,2)
(-2,3) to[short] (0,3)
(1,4) to[short] (1,5)
(2,4) to[short] (2,5)
(-4,-0.5) node[ground]{} to[V=$v_{\mathrm{s}}$] (-4,5) to[short] (-2,5)
(-4,-0.5) to[short] (1.5,-0.5)
;\end{circuitikz}
\caption{Circuit d'une puce 555 en mode astable. Les numéros des entrées de la puce sont définis à la figure~\ref{sch-555}.}
\label{sch-alarme-1}
\end{figure}

\end{document}