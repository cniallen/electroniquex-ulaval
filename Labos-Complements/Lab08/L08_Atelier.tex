%Créé par Claudine Allen en collaboration avec Jean-Raphaël Carrier
%(Élimination du labo de résistivité des matériaux (IV) à la fin de l'ère JRC + regroupement des labos VI & VII en début de pandémie COVID-19 => renumérotation X -> VIII maintenant)
%Dernière modification JRC: 2 avril 2014
%Dernière modification JG: 11 novembre 2020
%Dernière modification CA: 8 avril 2024
%**********************
%ToDo
% - pont de Wien: la branche sur l'entrée non inverseuse combine un filtre passe-haut et un filtre passe-bas pour obtenir une bande. Fixer RC pour un des filtres (voir ancien protocole) et étudier la bande passante pour différents RC de l'autre filtre. Changer la résistance 1 k\Omega pour un pot 5k afin de trouver le signal de sortie le plus sinusoïdal: plus il y a de gain d'amplification plus des fréquences de la bande passante vont s'ajouter. Noter qu'une bande passante trop large selon les valeurs de R et C choisies va aussi rajouter des fréquences et être moins sinusoïdal en sortie.
% - Oscillateur harmonique à rétroaction douteux, trouver le bobo : critère nécessaires, mais pas suffisants de Barkhausen
% - Puce 555: changer la fonctionnalité pour un calcul du temps visant à ré-intégrer la rétro-ingénierie du circuit mesurant l'accélération gravitationnelle entre cet atelier et le prochain. Passer alors le speaker pour le son sur un autre oscillateur (check anciens protocoles)
%- Section oscillateur quartz (en démo et/ou état solide!) à ajouter. ->Variance Allen: mesure de stabilité (avancé) pourrait être fait en démo qu’on garde tjrs tjrs fonctionnel et p-e même compare à une éventuelle horloge atomique :)

\RequirePackage[l2tabu, orthodox]{nag} %Check for obsolete commands
\documentclass[canadien,12pt,oneside,letterpaper]{article}
%
%-----------------------------------------------------
%Loading packages
%
\usepackage[utf8]{inputenc}
\usepackage[T1]{fontenc}
\usepackage[canadien]{babel}
\usepackage{lmodern}
\usepackage{textcomp}
\usepackage{amsmath,amssymb}
\usepackage{siunitx}
\usepackage{xcolor}
\usepackage[colorlinks=true,allcolors=blue]{hyperref}
\usepackage[all]{hypcap}
\usepackage{graphicx}
\usepackage[americanvoltages,americancurrents,siunitx]{circuitikz}
\usetikzlibrary{babel}
\usepackage{caption}
\usepackage[letterpaper,headheight=15pt]{geometry}
\usepackage{fancyhdr}
\usepackage{setspace}
\usepackage{float}
%
%----------------------------------------------------
%Other configurations and layout
%
\sisetup{separate-uncertainty}
\captionsetup{font=small,labelfont=bf,margin=0.1\textwidth}
\pagestyle{fancy}
\fancyhf{}
\lhead{\textsl{GPH-2006/PHY-2002~---~Laboratoire~VIII}}
\rhead{\textsl{Page \thepage}}
\setcounter{secnumdepth}{0}
\setlength{\parskip}{1.5ex plus0.5ex minus0.2ex}
\setlength\parindent{0pt}
%\onehalfspacing
\interfootnotelinepenalty=10000 %To avoid footnotes spreading on several pages.
%
%---------------------------------------------------
%
\title{\textbf{Atelier VIII}\\Oscillateurs électroniques\thanks{Auteurs: Claudine Allen, Samuel Nadeau, Jérémie Guilbert \& Jean-Raphaël Carrier}}
\renewcommand\footnotemark{}
\date{}

\begin{document}

\maketitle \vspace{-2.5cm}

\section{Thématique}\label{sec:objectifs}
À l’inverse de filtrer des fréquences d’un signal, cet atelier explore comment y ajouter de nouvelles fréquences. Un circuit linéaire ne peut pas modifier les fréquences à son entrée\footnote{Les modèles des circuits RLC sont des équations différentielles ordinaires d'ordre 2 où un signal en entrée est représenté par un terme non homogène. Comme son nom l'indique pour la résolution, la méthode des coefficients indéterminés démontre que la solution en signal de sortie a la même forme que l'entrée, incluant la même fréquence.}, d'où il ne peut pas générer un signal oscillant périodiquement (CA) à partir d’un courant constant (CC) qui a une fréquence nulle. Ceci nécessite alors au moins un composant non linéaire actif avec une boucle de rétroaction entre son entrée et sa sortie, d'où les conditions d'oscillation stable peuvent être limitées. Les premiers circuits vont contraster une oscillation instable entre un niveau haut et un niveau bas d'un inverseur avec celle stable et ajustable en fréquence d'une puce 555 dédiée comme minuterie. Les principes des autres oscillateurs étudiés feront intervenir la fonction de comparateur de l’amplificateur opérationnel pour contrôler le cycle de charge-décharge d’un condensateur, puis celle d’amplification avec rétroaction où la fréquence d’oscillation est sélectionnée avec un filtre passe-bande. Des fonctions d’intérêt pour les circuits oscillateurs sont d’actionner un transducteur afin d’obtenir des ondes mécaniques, ou encore de définir un étalon de temps (horloge) permettant de mesurer une durée écoulée. %La transduction sera d’abord mise en pratique avec un haut-parleur pour générer des ondes sonores, et puis avec la résonance de vibration d’un cristal de quartz.

% % Les travaux effectués aborderont les objectifs d’ensemble 1, 2, 5, 7, 11 et 12 du plan de cours et couvriront la qualité 5, utilisation d’outils d’ingénierie, prescrite dans les normes du Bureau d’agrément d’Ingénieurs Canada (BAIC).


\section{Lecture préparatoire}
\vspace{-1ex}
\begin{itemize}
\item complément \textit{Oscillateurs électroniques}
\end{itemize}

% \section{Matériel}

% La réalisation de ce laboratoire requiert l'utilisation de:
% \begin{itemize}
% \item un oscilloscope avec générateur de fonctions;
% \item un bloc d'alimentation;
% \item un multimètre;
% \item un haut-parleur;
% \item trois condensateurs de 1~$\mu$F dont un condensateur de précision ($\pm5$~\%);
% \item un peu beaucoup de résistances : 270~$\Omega$ (deux fois), 560~$\Omega$, 1~k$\Omega$, 10~k$\Omega$ (deux fois), 100~k$\Omega$ et 1~M$\Omega$;
% \item un potentiomètre linéaire de 1~k$\Omega$ (code 102);
% \item un amplificateur opérationnel;
% \item deux puces 555;
% \item une plaquette de montage.
% \end{itemize}

\section{Préparation}\label{sec:prep}

AVANT la séance d'atelier, écrivez sommairement dans votre cahier de recherche tout ce que vous prévoyez faire au laboratoire pour atteindre les objectifs. N'oubliez pas de calculer toute valeur ou modèle de référence donné en complément aux fins de comparaison à l'expérience. En plus des manipulations expérimentales, des mesures à prendre et de leur analyse, notez aussi les étapes de modélisation, conception et simulation nécessaires. Plusieurs logiciels spécialisés pour l'électronique listés sur le site de cours, dans la section \texttt{Logiciels} de l'onglet \texttt{Matériel didactique}, vous aideront dans cette préparation. En particulier, l'onglet \texttt{Circuits} de \href{https://www.falstad.com/circuit/}{l'application Web de Paul Falstad} contient pratiquement tous les circuits en atelier, montrant directement les comportements idéaux attendus\footnote{Une liste des circuits se trouve \href{https://www.falstad.com/circuit/directions.html}{ici}, mais elle n'est peut-être pas exhaustive.}. 

\section{Partie 1 --- Oscillateurs à relaxation}

    Les objectifs à atteindre avec les oscillateurs des figures~\ref{fig:relax-comparateur} et~\ref{fig:555chip} sont:
    
\begin{itemize}
    \item expliquer le fonctionnement d'un oscillateur à relaxation, %J'aimerais éventuellement arriver à faire modèliser, mais ça va prendre un retravail du complément de l'ampli-op avant. Mais surtout rajouter la version de vocademy.net pour introduire le concept de Duty cycle -> PWM & reconnecter à tension efficace du lab IV :)
    \item utiliser un oscillateur comme horloge numérique,
    \item pratiquer la conversion d'un signal électrique en un signal mécanique (sonore).
\end{itemize}

\begin{figure}[H]
\centering
\begin{circuitikz} \draw
(0,0) node[op amp](opamp){}
(opamp.+) to[short] 
(-1.2,-0.5) to[short] (-1.2,-2.2)
(opamp.-) to[short] 
(-1.2,0.5) to[short] (-1.2,2.2)
(opamp.out) to[short,-o] 
(2.1,0) node[right]{$v_{\mathrm{out}}$}
(opamp.down) ++ (0,-0.5) node[below]{$-5$~V} -- (opamp.down)
(opamp.up) ++ (0,0.5) node[above]{5~V} -- (opamp.up)
(-4,2.2) node[ground]{} to[C=1$\mu$F] 
(-1.2,2.2) to[R=560 $\Omega$] 
(1.6,2.2) to[short] 
(1.6,-2.2) to[R=10~k$\Omega$] 
(-1.2,-2.2) to[R=10~k$\Omega$] 
(-4,-2.2) node[ground]{}
;\end{circuitikz}
\caption{Oscillateur à relaxation avec ampli-op. Utilisez le condensateur de précision avec une capacité de 1 $\mu$F $\pm$ 5 \%.}
\label{fig:relax-comparateur}
\end{figure}
\vspace{-1ex}

\begin{figure}[h]
\centering
\begin{circuitikz} \draw[thick]
(0,0) to[short,-*] (0,1) node[right]{2} to[short] (0,2) node[right]{6} to[short,*-*] (0,3) node[right]{7} to[short] (0,4) to[short] (1,4) node[below]{4} to[short,*-*] (2,4) node[below]{8} to[short] (3,4) to[short] (3,3) node[left]{3} to[short,*-*] (3,1) node[left]{5} to[short] (3,0) to[short,-*] (1.5,0) node[above]{1} to[short] (0,0)
;\draw
(3,3) to[short,-o] 
(3.5,3) node[right]{$v_{\mathrm{out}}$}
(1.5,0) to[short] (1.5,-0.5)
(-2,-0.5) to[C=$C$] 
(-2,1) to[R=$R_B$] 
(-2,3) to[R=$R_A$] 
(-2,5) to[short] (2,5)
(-2,1) to[short] (0,1)
(-1,1) to[short] (-1,2) to[short] (0,2)
(-2,3) to[short] (0,3)
(1,4) to[short] (1,5)
(2,4) to[short] (2,5)
(-4,5) to[V, l_=$v_{\mathrm{s}}$] (-4,-0.5)
(-4,-0.5) node[ground]{} 
(-4,5) to[short] (-2,5)
(-4,-0.5) to[short] (1.5,-0.5)
;\end{circuitikz}
\caption{\label{fig:555chip}Oscillateur avec puce 555.}
\end{figure}

\subsection{Questions à explorer avec ces circuits}
\begin{itemize}
    \item Quelles sont les trois fonctions électroniques qui interviennent pour faire osciller le circuit de la figure~\ref{fig:relax-comparateur}? %J'aimerais éventuellement arriver à faire modèliser, mais ça va prendre un retravail du complément de l'ampli-op avant. Mais surtout rajouter la version de vocademy.net pour introduire le concept de Duty cycle -> PWM & reconnecter à tension efficace du lab IV :)
    \item Quel est la valeur de son gain en tension dicté par la branche en entrée non inverseuse?
    \item Quelle fréquence d'oscillation mesurez-vous à la sortie de ce circuit après l'avoir construit? Montrez que celui-ci est alors une horloge numérique avec une équation simple pour évaluer une durée de temps écoulé.
    \item Toujours pour le circuit de la figure \ref{fig:relax-comparateur}, si vous changez la mise à la terre pour une tension entre 0 à 2 V et que vous faites varier celle-ci, quel phénomène observez-vous?
    \item En vous référant à votre compréhension des circuits d'oscillateurs et à l'ampli-op, pouvez-vous expliquer pourquoi le phénomène a lieu?
    \item Quel serait une application de ce phénomène? (pas certain de la pertinence de cette question mais je la laisse ici pour l'instant, mais on peut les amener à parler du PWM et de l'Arduino)
    \item Quel ensemble de valeurs de $R_A$, $R_B$ et $C$ à la figure~\ref{fig:555chip} donnent un signal de sortie à une fréquence audible par l'oreille humaine? Vérifiez votre prédiction en ajoutant un haut-parleur à la sortie du circuit et en plaçant un potentiomètre pour $R_B$. %R_A=270~ohms, R_B=1~kohms, C=1~microF
    \item  Est-ce que l'ajout du haut-parleur affecte la fréquence et cause de la distorsion, c'est-à-dire une déformation du signal de sortie de l'oscillateur? Comment feriez-vous pour rapprocher le son d'une note juste, parfaite? %rajouter un filtre passe-bande actif avec le neilleur facteur de qualité possible
%    \item En ajoutant un deuxième oscillateur du même type, qu'arrive-t-il au son généré? %alarme, base de concepts de modulation pourraient être introduits, porteuse & enveloppe
%    \item Quel est le rôle de chacune des puces? Essayez différentes valeurs de résistances pour chacun des oscillateurs afin de répondre aux deux dernières questions. 
\end{itemize}
% proposer de trouver la note (table) et app pour mesurer (procède avec oscillo), j'ai envie de ramener ma symphonie! Mais probablement à l'intérieur même d'une équipe qui refait plusieurs fois le même circuit (vidéo de la fois que ça a déjà été fait). oouuuhhh ramener via programation d'un Arduino!... sauf que ça prendrait une résistance et/ou condensateur controllable par tension pour changer la férquence, hm. Old version : Dans cette partie du laboratoire, vous utiliserez deux puces 555 pour contruire un circuit constitué de deux oscillateurs. L'un produira une oscillation à une fréquence assez élevée pour être un son audible via un haut-parleur (la porteuse), alors que l'autre produira une oscillation à très basse fréquence qui modulera l'amplitude du son (l'enveloppe). Vous obtiendrez donc un son dont l'intensité oscillera dans le temps, comme une alarme!

\section{Partie 2 --- Oscillateur chaotique}

L'objectif de cette partie est de faire osciller une puce inverseuse qui convertit une tension basse en une tension haute et vice-versa, tout en décrivant son comportement.

\footnotetext{Cette puce fait partie de la famille des portes logiques auquel un prochain atelier est consacré. La tension basse est alors définie comme le niveau logique $0$ et la haute, comme le niveau logique $1$.}
\begin{figure}[h]
\centering
\begin{circuitikz} \draw
(0,0) node[not port](not){}
(not.in) to[short] (-1,0) node[left]{$v_{\mathrm{in}}$}
(not.out) to[short] (1,0) node[right]{$v_{\mathrm{out}}$}
;\end{circuitikz}
\caption{Inverseur logique sur puce aussi appelé porte NON\protect\footnotemark.}
\label{fig:astable-inverseur}
\end{figure}

% a) Un circuit logique très simple avec rétroaction est constitué uniquement d'un inverseur (porte NON) dont la sortie est retournée à l'entrée. Lorsque l'entrée est \textit{zéro}, la sortie est \textit{un}. Alors l'entrée devient \textit{un} et la sortie \textit{zéro}, et ainsi de suite. On dit que ce circuit est astable, puisqu'il ne possède aucun niveau stable : il s'agit en fait d'un oscillateur. Construisez cet oscillateur (figure~\ref{sch-osc-astable}) et mesurez au mieux, à l'aide de l'oscilloscope, la fréquence des oscillations. Soyez impressionnés de la valeur mesurée.
\subsection{Questions à explorer avec ce circuit}
\begin{itemize}
    \item En analysant à l'oscilloscope l'entrée et la sortie de l'inverseur schématisé à la figure~\ref{fig:astable-inverseur}, que se passe-t-il si vous reliez $v_{\text{out}}$ et $v_{\text{in}}$ par un simple fil? %rétroaction, oscillation astable
    \item Quel est l'ordre de grandeur de la fréquence d'oscillation? Pouvez-vous estimer la gigue temporelle \textit{(jitter)}, c'est-à-dire de combien fluctue la valeur de la période?
    \item Expérimentez avec l'alimentation de la puce, les autres inverseurs sur celle-ci et le mode horizontal d'affichage à l'oscilloscope, pouvez-vous obtenir une fonction au sens mathématique du terme, soit $v_{\text{out}} = f\left(v_{\text{in}}\right)$? Est-ce que tous les couples de coordonnées $(v_{\text{in}},v_{\text{out}})$ se produisent? %Ça prend le mode XY et j'ai vu au mieux un cycle limite, encore jamais de fonction associant une seule valeur de v_out à v_in. L'ensemble de coordonées qui se produit est cependant corrélé (pourrait faire calculer des matrices de covariance, corrélation et trucs du genre ouh!), il ne couvre pas l'infinité de toutes les valeurs.
\end{itemize}
%Ring oscillator d'intérêt (et aussi bascule asynchrone au prochain labo), en particulier pour introduire le terme "jitter": https://en.wikipedia.org/wiki/Ring_oscillator; https://www.google.com/url?sa=t&rct=j&q=&esrc=s&source=web&cd=&ved=2ahUKEwin3Lz0zK6FAxVrFmIAHRc5DzoQFnoECBMQAQ&url=https%3A%2F%2Fen.wikipedia.org%2Fwiki%2FRing_oscillator&usg=AOvVaw1HaXg4iCtqOLMUE78BBG_j&opi=89978449 -> Application "A Provably Secure True Random Number Generator with Built-in Tolerance to Active Attacks" (PDF). Archived from the original (PDF) on 2016-03-04. Retrieved 2012-05-12. ; Aussi dans un contexte de logique, On dit que ce circuit est astable, puisqu'il ne possède aucun niveau stable : il s'agit en fait d'un oscillateur.

\section{Partie 3 --- Oscillateur harmonique à rétroaction}
L'objectif de cette dernière partie est de concevoir son propre oscillateur harmonique en sélectionnant et caractérisant :

\begin{itemize}
    \item le gain en rétroaction sur l'entrée inverseuse de l'ampli-op,
    \item le filtre à l'entrée non inverseuse.
\end{itemize}

\subsection{Pistes à explorer}
\begin{itemize}
    \item Revoir le complément qui traite de gain et celui qui traite de filtrage.
    \item Rechercher des exemples de circuits d'oscillateurs harmoniques en partant du complément de cet atelier jusqu'à tout le vaste Internet, sans oublier notre \href{https://www.bibl.ulaval.ca}{bibliothèque universitaire} qui garantit la crédibilité des ressources cataloguées.
    \item Faire un diagramme schématique du circuit et les calculs théoriques associés pour le modèle idéal d'oscillateur harmonique sélectionné.
    \item Vérifier ses calculs en simulant ce circuit à l'aide de votre app préférée de la section \texttt{Matériel didactique}$\rightarrow$\texttt{Logiciels} sur le site de cours.
    \item Comparer le gain d'amplification calculé $>1$ à celui mesuré en pratique avec le ratio $\frac{v_{\text{out}}}{v_{\text{in}}}$ lorsque les résistances sont branchées à l'ampli-op avec une tension connue appliquée en entrée non inverseuse.
    \item En l'absence de l'ampli-op, caractériser la réponse complexe en fréquence du ou des filtres choisis à l'aide des diagrammes de Bode.
\end{itemize}
%Commence de qqch qui a déjà été (peut faire remarquer que mes choix de filtre la semaine dernière étaient pédagogiques, beaucoup de design plus complexes et performants existent), référer à tout mon blabla DIY à réorganiser en médiagraphie du site et présenter en début de lancement de projet de conception avec logiciels, fonctions, etc.)

%Parler des critères de Barkhausen au tableau et éventuellement mettre en complément

% Pont de Wien du complément fonctionne avec les caleurs $C=1\,\mu\mathrm{F}$ et $R=270\,\Omega$

\end{document}