%Créé par Claudine Allen en collaboration avec Jean-Raphaël Carrier
%(Élimination du labo de résistivité des matériaux (IV) à la fin de l'ère JRC + regroupement des labos VI & VII en début de pandémie COVID-19 => renumérotation X -> VIII maintenant)
%Dernière modification JRC: 2 avril 2014
%Dernière modification JG: 11 novembre 2020
%Dernière modification CA: 29 novembre 2022
%**********************
%ToDo
% - pont de Wien: la branche sur l'entrée non inverseuse combine un filtre passe-haut et un filtre passe-bas pour obtenir une bande. Fixer RC pour un des filtres (voir ancien protocole) et étudier la bande passante pour différents RC de l'autre filtre. Changer la résistance 1 k\Omega pour un pot 5k afin de trouver le signal de sortie le plus sinusoïdal: plus il y a de gain d'amplification plus des fréquences de la bande passante vont s'ajouter. Noter qu'une bande passante trop large selon les valeurs de R et C choisies va aussi rajouter des fréquences et être moins sinusoïdal en sortie.
% - Oscillateur harmonique à rétroaction douteux, trouver le bobo ou remplacer p-e avec l'oscillateur astable de la porte NON-OU qui est commented out dans le prochain atelier? consulter la bible The Art of electronics pour se faire une tête.
% - Puce 555: changer la fonctionnalité pour un calcul du temps visant à ré-intégrer la rétro-ingénierie du circuit mesurant l'accélération gravitationnelle entre cet atelier et le prochain. Passer alors le speaker pour le son sur un autre oscillateur (check anciens protocoles)
% - Raccourcir en combinant les sections de préparation et déroulement de l'atelier vu que ça répète le début du lab 7 (à recopier au cours 6 aussi). Réviser et ramener les anciens objectifs en renommant "Thématique et objectifs" en voyant si ça ne serait pas mieux de renommer les sous-objectifs en "but à atteindre ou qqch du genre.

\RequirePackage[l2tabu, orthodox]{nag} %Check for obsolete commands
\documentclass[canadien,12pt,oneside,letterpaper]{article}
%
%-----------------------------------------------------
%Loading packages
%
\usepackage[utf8]{inputenc}
\usepackage[T1]{fontenc}
\usepackage[canadien]{babel}
\usepackage{lmodern}
\usepackage{textcomp}
\usepackage{amsmath,amssymb}
\usepackage{siunitx}
\usepackage{xcolor}
\usepackage{hyperref}
\usepackage[all]{hypcap}
\usepackage{graphicx}
\usepackage[oldvoltagedirection,americanvoltages,americancurrents,siunitx]{circuitikz}
\usetikzlibrary{babel}
\usepackage{caption}
\usepackage[letterpaper,headheight=15pt]{geometry}
\usepackage{fancyhdr}
\usepackage{setspace}
\usepackage{float}
%
%----------------------------------------------------
%Other configurations and layout
%
\sisetup{separate-uncertainty}
\captionsetup{font=small,labelfont=bf,margin=0.1\textwidth}
\pagestyle{fancy}
\fancyhf{}
\lhead{\textsl{GPH-2006/PHY-2002~---~Laboratoire~VIII}}
\rhead{\textsl{Page \thepage}}
\setcounter{secnumdepth}{0}
\setlength{\parskip}{1.5ex plus0.5ex minus0.2ex}
\setlength\parindent{0pt}
%\onehalfspacing
\interfootnotelinepenalty=10000 %To avoid footnotes spreading on several pages.
%
%---------------------------------------------------
%
\title{\textbf{Atelier VIII}\\Oscillateurs électroniques\thanks{Auteurs: Claudine Allen, Jérémie Guilbert \& Jean-Raphaël Carrier}}
\renewcommand\footnotemark{}
\date{}

\begin{document}

\maketitle \vspace{-2cm}

%\section{Objectif}

% À l’inverse de filtrer des fréquences d’un signal, cet atelier explore comment ajouter de nouvelles fréquences. La génération de ces signaux oscillant périodiquement à partir d’un courant constant fera notamment intervenir la fonction de comparateur de l’amplificateur opérationnel pour contrôler le cycle de charge-décharge d’un condensateur, puis celle d’amplification avec rétroaction où la fréquence d’oscillation est sélectionnée avec un filtre passe-bande. L’étudiant.e sera aussi initié.e à la conception d’un circuit LC résonnant à une fréquence d’oscillation requise. Des fonctions d’intérêt pour les circuits oscillateurs étudiés sont d’actionner un transducteur afin d’obtenir des ondes mécaniques, ou encore de définir un étalon de temps lorsque l’oscillation est pratiquement harmonique avec une seule fréquence. %La transduction sera d’abord mise en pratique avec un haut-parleur pour générer des ondes sonores, et puis avec la résonance de vibration d’un cristal de quartz pour obtenir une oscillation quasi-monochrome qui définira la seconde.

% % Les travaux effectués aborderont les objectifs d’ensemble 1, 2, 5, 7, 11 et 12 du plan de cours et couvriront la qualité 5, utilisation d’outils d’ingénierie, prescrite dans les normes du Bureau d’agrément d’Ingénieurs Canada (BAIC).


\section{Lecture préparatoire}
\begin{itemize}
\item complément \textit{Oscillateurs électroniques}
\end{itemize}
\vspace{-1ex}
% Avant de se présenter à la séance de laboratoire, chaque étudiant.e doit:
% \begin{itemize}
% \item lire le protocole de ce laboratoire;
% \item déterminer la fréquence des oscillations que produit l'oscillateur à relaxation (figure~\ref{sch-osc-relax}) en fonction de $R$ et $C$ selon les principes de fonctionnement de l'amplificateur opérationnel.
% \end{itemize}

% \section{Matériel}

% La réalisation de ce laboratoire requiert l'utilisation de:
% \begin{itemize}
% \item un oscilloscope avec générateur de fonctions;
% \item un bloc d'alimentation;
% \item un multimètre;
% \item un haut-parleur;
% \item trois condensateurs de 1~$\mu$F dont un condensateur de précision ($\pm5$~\%);
% \item un peu beaucoup de résistances : 270~$\Omega$ (deux fois), 560~$\Omega$, 1~k$\Omega$, 10~k$\Omega$ (deux fois), 100~k$\Omega$ et 1~M$\Omega$;
% \item un potentiomètre linéaire de 1~k$\Omega$ (code 102);
% \item un amplificateur opérationnel;
% \item deux puces 555;
% \item une plaquette de montage.
% \end{itemize}
\section{Préparation}

\setlength{\parskip}{1ex plus 0.5ex minus 0.2ex}
Les ateliers se rapprochent d'un mode de travail expérimental plus professionnel, car vous devez vous-même préparer votre propre protocole avant d'arriver au laboratoire afin d'atteindre chaque sous-objectif énoncé ci-dessous. Ces sous-objectifs sont généralement accompagnés d'une figure de circuits, mais il y a toutefois progression vers une conception de plus en plus libre où vous devrez proposer vos propres design de circuits. Pour vous assister dans cette rédaction de protocole, les compléments en lecture préparatoire ci-dessus détaillent la théorie et les modèles pour comprendre le fonctionnement des circuits et les analyser expérimentalement. Concrètement dans votre cahier de recherche avec votre coéquipier.ère, écrivez des manipulations et des analyses à faire pour chaque sous-objectif afin de les mettre en commun en plus grand groupe lors de votre arrivée à la séance de laboratoire. 

\section{Déroulement de l’atelier}

Lors de la séance de laboratoire de 2h à la date spécifiée dans la section \texttt{Contenu et activités} sur le site de cours, nous nous diviserons en grands groupes, chacun animé par un membre de l’équipe d’enseignement. Une partie des sous-objectifs et circuits sera attribuée à chaque groupe dans le but d'en faire une présentation orale éducative de 5~minutes pour le reste de la classe à la fin de la séance. La méthodologie de travail en groupe vous appartient alors entièrement pour réussir la meilleure présentation possible qui sera évaluée sur une base ordinale tel qu'expliqué dans la partie synchrone du cours~6.

Nous recommandons de commencer la séance en délibérant sur l'optimisation des manipulations expérimentales et des analyses proposées dans vos cahiers de recherche respectifs avec l'assistance du membre de l'équipe d'enseignement et il vous faudra ensuite répartir les tâches dans votre groupe. Des exemples de tâches possibles sont la planification et l'organisation, la conception, la simulation, la réalisation expérimentale, la mesure, l'analyse, la communication, etc. et il peut bien sûr y avoir de la redondance pour vérifier l'exactitude des résultats de mesure et d'analyse. Notez que la fin de la section \texttt{Matériel didactique} sur le site de cours propose différents choix de logiciels et d'applications pour vous assister dans les tâches de conception et de simulation.

%La dernière activité de la séance sera de vous rendre sur le site Web du cours dans la section \texttt{Évaluations et résultats} pour l'atelier concerné et y évaluer vos pairs sur leur contribution respective au travail du groupe. Il n'y a pas d'autres remises à faire pour chaque atelier afin de vous donner le temps de travailler sur votre projet de conception final.

\setlength{\parskip}{1ex plus 0.5ex minus 0.2ex}

\subsection{Oscillateur à relaxation avec comparateur}

\begin{figure}[H]
\centering
\begin{circuitikz} \draw
(0,0) node[op amp](opamp){}
(opamp.+) to[short] (-1.2,-0.5) to[short] (-1.2,-2.2)
(opamp.-) to[short] (-1.2,0.5) to[short] (-1.2,2.2)
(opamp.out) to[short,-o] (2.1,0) node[right]{$v_{\mathrm{out}}$}
(opamp.down) ++ (0,-0.5) node[below]{$-5$~V} -- (opamp.down)
(opamp.up) ++ (0,0.5) node[above]{5~V} -- (opamp.up)
(-4,2.2) node[ground]{} to[C=$C$] (-1.2,2.2) to[R=$R$] (1.6,2.2) to[short] (1.6,-2.2) to[R=10~k$\Omega$] (-1.2,-2.2) to[R=10~k$\Omega$] (-4,-2.2) node[ground]{}
;\end{circuitikz}
\caption{\label{sch-osc-relax}\textbf{Sous-objectif:} Déterminer la forme du signal à la sortie de l'amplificateur et étudier l'impact des paramètres R et C sur celui-ci.}
\end{figure}

% a) Sur la plaquette de montage, construisez le circuit de l'oscillateur à relaxation que vous avez analysé en préparation (figure~\ref{sch-osc-relax}). Encore une fois, envoyez la sortie $v_{\mathrm{out}}$ à la fois dans le haut-parleur et dans le canal~1 de l'oscilloscope. Utilisez le condensateur de précision de 1~$\mu\mathrm{F}$~$\pm5$~\% pour la capacité $C$. Pour la résistance $R$, placez une résistance de 560~$\Omega$.

% b) À l'oscilloscope, observez la forme du signal et notez sa fréquence à l'aide des curseurs et aussi avec la FFT.

% c) Comparez vos résultats expérimentaux avec l'équation de la fréquence en fonction de $R$ et $C$ que vous aviez calculée en préparation. Pour vos calculs, utilisez la valeur nominale du condensateur et la valeur réelle de la résistance mesurée avec le multimètre.

% d) Remplacez la résistance de 560~$\Omega$ par un potentiomètre linéaire de 1~k$\Omega$. En ajustant le potentiomètre, vous serez en mesure de générer un son à une fréquence donnée. Vous pouvez ajouter une résistance en série avec le potentiomètre si vous avez besoin d'une résistance supérieure à 1~k$\Omega$. Chaque équipe devra produire une note différente (dictée par vos dictateurs). Le but ultime est, en combinant les efforts de toute la classe, de jouer une symphonie grandiose!


\subsection{Oscillateur harmonique à rétroaction - pont de Wien}

% a) En continuant sur une autre plaquette de montage, construisez l'oscillateur à pont de Wien (figure~\ref{sch-osc-Wien}). Prenez les composants $C=1\,\mu\mathrm{F}$ et $R=270\,\Omega$. Utilisez les sorties $+25$~V et $-25$~V du bloc d'alimentation pour alimenter l'amplificateur opérationnel.

\vspace{2ex}
\begin{figure}[H]
\centering
\begin{circuitikz} \draw
(0,0) node[op amp](opamp){}
(opamp.+) to[short] (-1.2,-0.5) to[short] (-1.2,-2.2)
(opamp.-) to[short] (-1.2,0.5) to[short] (-1.2,2.2)
(opamp.out) to[short,-o] (2.1,0) node[right]{$v_{\mathrm{out}}$}
(opamp.down) ++ (0,-0.5) node[below]{$-5$~V} -- (opamp.down)
(opamp.up) ++ (0,0.5) node[above]{5~V} -- (opamp.up)
(-4,2.2) node[ground]{} to[R=1~k$\Omega$] (-1.2,2.2) to[R=10~k$\Omega$] (1.6,2.2) to[short] (1.6,0) to[C=$C$] (1.6,-2.2) to[R=$R$] (-1.2,-2.2)
(-1.2,-2.2) to[C=$C$] (-3,-2.2) node[ground]{} to[R=$R$] (-3,-0.5) to[short] (-1.2,-0.5)
;\end{circuitikz}
\vspace{2ex}
\caption{\label{sch-osc-Wien}\textbf{Sous-objectif:} Puisque la section du bas de ce circuit (SANS l'ampli-op ni la section du haut) combine un filtre passe-haut avec un filtre passe-bas, caractériser le filtre passe-bande qui résulte pour différentes valeurs de R et C.} %Également, ajouter de compléter le circuit optimiser bande + gain pour meilleur sin en sortie, voir ToDo au début.
\end{figure}
%OBJECTIFS: 1) Déterminer la fréquence du signal à la sortie. 2) Déterminer l'impact de l'ajout d'une charge à la sortie (résistance ou haut parleur sur tinkercad??) sur la fréquence. Qu'est-ce qui cause la variation observée?

% b) À l'aide de l'oscilloscope, observez la forme du signal $v_{\mathrm{out}}$. Notez la fréquence principale du signal généré en mesurant la fréquence des oscillations avec les curseurs et aussi en effectuant une FFT.

% c) Branchez un haut-parleur entre la sortie $v_{\mathrm{out}}$ et la mise à la terre. Cependant, l'ajout du haut-parleur dans le circuit modifie légèrement le fonctionnement de ce dernier. En gardant affichée la FFT du signal à l'oscilloscope, mesurez la nouvelle fréquence principale.

% \subsubsection{Fonctionnement de l'oscillateur à pont de Wien}
% %SOUS-SECTION À METTRE EN COMPLÉMENT
% Le signal à l'entrée non inverseuse (\textit{a priori} ce signal n'est que du bruit) est amplifié, puis il passe au travers d'un pont de Wien qui le filtre, avant de retourner à l'entrée de l'amplificateur, pour être à nouveau amplifié et filtré, et ainsi de suite, \textit{ad infinitum}. Le pont de Wien est \textit{grosso modo} l'addition d'un filtre passe-bas et d'un filtre passe-haut, pour former un filtre passe-bande. La fréquence du signal de sortie est $f=\left(2\,\pi\,R\,C\right)^{-1}$. L'amplitude finale du signal est fixée par la source d'alimentation. % Cet oscillateur doit son oscillation au chargement/déchargement du condensateur qui relie l'entrée non inverseuse de l'amplificateur au \textit{ground}.

\vspace{ \stretch{3} }
\subsection{Conception - Oscillateur harmonique à rétroaction}

% Un autre oscillateur reposant sur le principe de filtrage et amplification en boucle du signal est l'oscillateur LC. Vous avez étudié au dernier atelier le filtre RLC passe-bande dont la fréquence de résonance correspond à une oscillation, tel qu'attendu pour tout système linéaire qui se modélise par une équation différentielle ordinaire d'ordre 2. Ce filtre peut donc jouer le rôle du pont de Wien. 
  
% Faites la conception d'un tel oscillateur en plaçant un circuit LC parallèle (\textit{tank circuit}) à l'entrée non inverseuse d'un amplificateur non inverseur. Le circuit devra respecter les spécifications suivantes à démontrer à l'oscilloscope:

\vspace{ \stretch{1} }
Faites la conception d'un circuit oscillateur basé sur l'utilisation d'un ampli-op en rétroaction. En utilisant le circuit approprié placé à l'entrée non inverseuse de l'amplificateur, vous devez pouvoir obtenir les spécifications suivantes pour le signal de sortie $v_\mathrm{out}$ :  

\vspace{ \stretch{1} }
\begin{itemize}
     \setlength{\itemsep}{2ex}
     \item gain d'au moins un facteur 10,
     \item fréquence minimale d'oscillation de 7 kHz.
 \end{itemize}
 
\newpage
\begin{figure}[h]
\centering
\begin{circuitikz} \draw
(0,0) node[op amp](opamp){}
(opamp.+) to[short] (-1.2,-0.5) to[short] (-1.2,-2.2)
(opamp.-) to[short] (-1.2,0.5) to[short] (-1.2,2.2)
(opamp.out) to[short,-o] (2.1,0) node[right]{$v_{\mathrm{out}}$}
(opamp.down) ++ (0,-0.5) node[below]{$-5$~V} -- (opamp.down)
(opamp.up) ++ (0,0.5) node[above]{5~V} -- (opamp.up)
(-4,2.2) node[ground]{} to[R=$R_1$] (-1.2,2.2) to[R=$R_2$] (1.6,2.2) to[short] (1.6,-2.2) to[R=$R_3$] (-1.2,-2.2) to[short] (-2.2,-2.2)
node[draw,minimum width=2cm,minimum height=2cm, dashed] at (-3.2,-2.2){?} (-4.2,-2.2) to[short] (-5,-2.2) node[ground]{};
\end{circuitikz}
\caption{\label{sch-osc-relax}\textbf{Sous-objectif:} Compléter le circuit avec les composants de votre choix afin d'obtenir les spécifications demandées.}
\end{figure}

%%%CORRIGÉ%%%
%  \begin{figure}[h]
% \centering
% \begin{circuitikz} \draw
% (0,0) node[op amp](opamp){}
% (opamp.+) to[short] (-1.2,-0.5) to[short] (-1.2,-2.2)
% (opamp.-) to[short] (-1.2,0.5) to[short] (-1.2,2.2)
% (opamp.out) to[short,-o] (2.1,0) node[right]{$v_{\mathrm{out}}$}
% (opamp.down) ++ (0,-0.5) node[below]{$-5$~V} -- (opamp.down)
% (opamp.up) ++ (0,0.5) node[above]{5~V} -- (opamp.up)
% (-4,2.2) node[ground]{} to[R=$1~\Omega$] (-1.2,2.2) to[R=$11~\Omega$] (1.6,2.2) to[short] (1.6,-2.2) to[R=$560~\Omega$] (-1.2,-2.2) to[short] (-2.2,-2.2);
% \draw (-2.2,-2.2) to[short] (-2.2,-1.2) to[C,l_=$1~\mathrm{nF}$] (-4.2,-1.2) to[short] (-4.2,-3.2) to[L=$1~\mathrm{mH}$] (-2.2,-3.2) -- (-2.2,-2.2);
% \draw (-4.2,-2.2) to[short] (-5,-2.2) node[ground];
% %node[draw,minimum width=2cm,minimum height=2cm, dashed] at (-3.2,-2.2){?} (-4.2,-2.2) to[short] (-5,-2.2) node[ground]{};
% \end{circuitikz}
% \caption{Oscillateur LC--Exemple fonctionnel}
% \end{figure}

 %La recherche sur Internet en particulier filtrée par images, ou oserons-nous dans la littérature en bibliothèque(!), est la meilleure amie du concepteur en herbe. Vous pouvez aussi simuler le circuit si désiré, avec <\url{http://www.falstad.com/circuit/>} par exemple.

\subsection{Oscillations avec la puce 555}

% La puce 555 est un circuit intégré qui est très souvent utilisé en électronique pour, entre autres, bâtir un oscillateur.
% \begin{figure}[h]
%  \centering
%  \begin{circuitikz} \draw[thick]
%  (0,0) to[short] (0,0.5) node[right]{4} to[short,*-*] (0,1.5) node[right]{3} to[short] (0,2.5) node[right]{2} to[short,*-*] (0,3.5) node[right]{1} to[short] (0,4) to[short] (3,4) to[short] (3,3.5) node[left]{8} to[short,*-*] (3,2.5) node[left]{7} to[short] (3,1.5) node[left]{6} to[short,*-*] (3,0.5) node[left]{5} to[short] (3,0) to[short] (0,0)
%  {[anchor=east] (0,3.5) node{\textbf{GND}} (0,2.5) node{\textbf{TRIG}} (0,1.5) node{\textbf{OUT}} (0,0.5) node{\textbf{RESET}}}
%  {[anchor=west] (3,3.5) node{\textbf{V$_{\text{CC}}$}} (3,2.5) node{\textbf{DISCH}} (3,1.5) node{\textbf{THRES}} (3,0.5) node{\textbf{CONT}}}
%  ;\end{circuitikz}
%  \caption{\label{sch-555}Schéma d'une puce 555.}
%  \end{figure}

%\subsubsection{Fonctionnement de la puce 555}

% Les niveaux de déclenchement (\textit{trigger}) et de seuil (\textit{threshold}) valent un tiers et deux tiers de l'alimentation, respectivement. Lorsque l'entrée \textbf{TRIG} est inférieure au niveau de déclenchement, la sortie de la puce (\textbf{OUT}) vaut \textbf{V$_{\text{CC}}$}. Lorsque l'entrée \textbf{TRIG} est supérieure au niveau de déclenchement et qu'en plus l'entrée \textbf{THRES} est supérieure au niveau de seuil, alors la sortie devient nulle. Lorsque la sortie devient nulle, un court-circuit se fait entre l'entrée \textbf{DISCH} (\textit{discharge}) et la mise à la terre. Mais vous comprendrez mieux en voyant un circuit complet.
% \begin{table}[h]
% \centering
% \begin{tabular}{|c|c|c|}
% \hline
% \textbf{TRIG} & \textbf{THRES} & \textbf{OUT} \\
% \hline
% $<\frac{1}{3}$\textbf{V$_{\text{CC}}$} & --- & \textbf{V$_{\text{CC}}$} \\
% \hline
% $>\frac{1}{3}$\textbf{V$_{\text{CC}}$} & $>\frac{2}{3}$\textbf{V$_{\text{CC}}$} & 0 \\
% \hline
% $>\frac{1}{3}$\textbf{V$_{\text{CC}}$} & $<\frac{2}{3}$\textbf{V$_{\text{CC}}$} & Conserve l'état \\
% \hline
% \end{tabular}
% \caption{\label{table-555}Sortie d'une puce 555 en fonction des tensions aux entrées \textbf{TRIG} et \textbf{THRES}.}
% \end{table}

% Dans cette partie du laboratoire, vous utiliserez deux puces 555 pour contruire un circuit constitué de deux oscillateurs. L'un produira une oscillation à une fréquence assez élevée pour être un son audible via un haut-parleur (la porteuse), alors que l'autre produira une oscillation à très basse fréquence qui modulera l'amplitude du son (l'enveloppe). Vous obtiendrez donc un son dont l'intensité oscillera dans le temps, comme une alarme!

% a) Construisez d'abord l'oscillateur qui permettra de produire le son. Sur la plaquette de montage, faites le circuit de la figure~\ref{sch-alarme-1}. Utilisez les composants $R_A=270~\Omega$, $R_B=1~k\Omega$ et $C=1~\mu\mathrm{F}$. Reliez le haut-parleur à la sortie de la puce.
\begin{figure}[H]
\centering
\begin{circuitikz} \draw[thick]
(0,0) to[short,-*] (0,1) node[right]{2} to[short] (0,2) node[right]{6} to[short,*-*] (0,3) node[right]{7} to[short] (0,4) to[short] (1,4) node[below]{4} to[short,*-*] (2,4) node[below]{8} to[short] (3,4) to[short] (3,3) node[left]{3} to[short,*-*] (3,1) node[left]{5} to[short] (3,0) to[short,-*] (1.5,0) node[above]{1} to[short] (0,0)
;\draw
(3,3) to[short,-o] (3.5,3) node[right]{$v_{\mathrm{out}}$}
(1.5,0) to[short] (1.5,-0.5)
(-2,-0.5) to[C=$C$] (-2,1) to[R=$R_B$] (-2,3) to[R=$R_A$] (-2,5) to[short] (2,5)
(-2,1) to[short] (0,1)
(-1,1) to[short] (-1,2) to[short] (0,2)
(-2,3) to[short] (0,3)
(1,4) to[short] (1,5)
(2,4) to[short] (2,5)
(-4,-0.5) node[ground]{} to[V=$v_{\mathrm{s}}$] (-4,5) to[short] (-2,5)
(-4,-0.5) to[short] (1.5,-0.5)
;\end{circuitikz}
\caption{\label{sch-alarme-1}\textbf{Sous-objectif:} Connecter un haut-parleur à la sortie $v_\mathrm{out}$ de la puce 555. On souhaite ainsi faire la \textit{transduction} du signal $v_\mathrm{out}$, c'est-à-dire le convertir d'un signal électrique à un signal mécanique (sonore). Déterminer un ensemble de valeurs de $R_A$, $R_B$ et $C$ qui feront en sorte que le signal résultant sera à une fréquence audible par l'oreille humaine.}
\end{figure}

\subsubsection{Optionnel: Fabrication d'une alarme}

Vous pouvez obtenir un son d'alarme rudimentaire en ajoutant un deuxième oscillateur qui viendra moduler beaucoup plus lentement le son de votre premier oscillateur. Avec le circuit de la figure~\ref{sch-alarme-2}, la sortie de la nouvelle puce (B) commande en effet directement directement l'autre (A). Notez que les deux nouvelles résistances sont nettement plus grandes, afin de fournir un signal oscillant beaucoup plus lentement.
\begin{figure}[h]
\centering
\begin{circuitikz} \draw[thick]
(0,0) to[short,-*] (0,1) node[right]{2} to[short] (0,2) node[right]{6} to[short,*-*] (0,3) node[right]{7} to[short] (0,4) to[short] (1,4) node[below]{4} to[short,*-*] (2,4) node[below]{8} to[short] (3,4) to[short] (3,3) node[left]{3} to[short,*-*] (3,1) node[left]{5} to[short] (3,0) to[short,-*] (1.5,0) node[above]{1} to[short] (0,0)
(-7,0) to[short,-*] (-7,1) node[right]{2} to[short] (-7,2) node[right]{6} to[short,*-*] (-7,3) node[right]{7} to[short] (-7,4) to[short] (-6,4) node[below]{4} to[short,*-*] (-5,4) node[below]{8} to[short] (-4,4) to[short] (-4,3) node[left]{3} to[short,*-*] (-4,1) node[left]{5} to[short] (-4,0) to[short,-*] (-5.5,0) node[above]{1} to[short] (-7,0)
;\draw
(3,3) to[short,-o] (3.5,3) node[right]{$v_{\mathrm{out}}$}
(1.5,0) to[short] (1.5,-0.5)
(-2,-0.5) to[C=1~$\mu$F] (-2,1) to[R=1~k$\Omega$] (-2,3)
(-2,1) to[short] (0,1)
(-1,1) to[short] (-1,2) to[short] (0,2)
(-2,3) to[short] (0,3)
(1,4) to[short] (1,5)
(2,4) to[short] (2,5)
(-11,-0.5) node[ground]{} to[V=4~V] (-11,5) to[short] (-2,5) to[short] (2,5)
(-4,3) to[R=270~$\Omega$] (-2,3)
(-5.5,0) to[short] (-5.5,-0.5)
(-9,-0.5) to[C=1~$\mu$F] (-9,1) to[R=1~M$\Omega$] (-9,3) to[R=100~k$\Omega$] (-9,5)
(-9,1) to[short] (-7,1)
(-8,1) to[short] (-8,2) to[short] (-7,2)
(-9,3) to[short] (-7,3)
(-6,4) to[short] (-6,5)
(-5,4) to[short] (-5,5)
(-11,-0.5) to[short] (1.5,-0.5)
(-5.5,1.7) node[above]{\textbf{B}}
(1.5,1.7) node[above]{\textbf{A}}
;\end{circuitikz}
\caption{\label{sch-alarme-2}Circuit d'une alarme.}
\end{figure}

\subsubsection{Note sur les oscillateurs et le simulateur Falstad}
Sur le simulateur de circuits électroniques de Paul Falstad, il est bien important de différencier entre les amplificateurs opérationnels «réels» et «idéaux». Les amplificateurs réels sont de mise ici, car dans certains cas, c'est la présence de bruit aux bornes d'entrée qui est à l'origine du signal mesuré à la sortie, un phénomène qui n'est pas modélisé par les amplificateurs idéaux.  

% Initialement, lorsque le bloc d'alimentation s'allume, la tension à l'entrée \textbf{TRIG} est basse, donc la sortie est élevée ($v_{\mathrm{out}}=v_{\mathrm{s}}=\mathrm{V_{CC}}$) et le condensateur $C$ se charge avec une constante de temps $\tau=\left(R_A+R_B\right)\,C$. La tension aux bornes du condensateur, \textit{i.e.} la tension aux terminaux \textbf{TRIG} et \textbf{THRES}, augmente jusqu'à atteindre le niveau de seuil, soit $\frac{2}{3}\,v_{\mathrm{s}}$. À ce moment, la sortie devient basse ($v_{\mathrm{out}}=0$~V), le terminal \textbf{DISCH} devient mis à la terre et le condensateur se décharge avec une constante de temps plus petite $\tau=R_B\,C$. Lorsque la différence de potentiel aux bornes du condensateur a diminué jusqu'à atteindre $\frac{1}{3}\,v_{\mathrm{s}}$, alors la sortie redevient élevée, le terminal \textbf{DISCH} est déconnecté de la masse et le condensateur recommence à se charger. Ainsi, la sortie passe de 0~V à $v_{\mathrm{s}}$ puis retourne à 0~V et ainsi de suite, avec une période de $T=\mathrm{ln}\!\left(2\right)\left(R_A+2\,R_B\right)C$\label{eq:alarme}.

% b) Allumez le bloc d'alimentation et fournissez une tension suffisante pour entendre un son clair (4~V devrait être une tension suffisante). Lorsque vous êtes tannés d'entendre cette note plutôt horrible, désactivez l'alimentation.

% d) Lorsque votre circuit est fin prêt, allumez de nouveau le bloc d'alimentation et écoutez la douce mélodie de votre alarme!

% e) Comparez les deux fréquences (la fréquence sonore et la fréquence associée à l'alternance son--silence) de l'alarme avec les fréquences attendues (calculées à partir de l'équation donnée en page~\pageref{eq:alarme}).

% \subsection{Partie pour terminer votre dernier laboratoire plus tôt}

% Si vous complétez vos manipulations avant la fin de la séance de laboratoire, veuillez lire la première partie du prochain et dernier laboratoire de votre session puis vous présenter au montage d'accélération gravitationnelle pour y faire votre première contribution d'une dizaine de minutes.


%Section oscillateur quartz (démo?) à ajouter. ->Variance Allen: mesure de stabilité (avancé) pourrait être fait en démo qu’on garde tjrs tjrs fonctionnel et p-e même compare à une éventuelle horloge atomique :)

\end{document}