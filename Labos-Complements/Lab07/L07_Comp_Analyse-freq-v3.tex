\documentclass[12pt,oneside,letterpaper]{article}

\usepackage[canadien]{babel}
\usepackage[utf8]{inputenc}
\usepackage[T1]{fontenc}
\usepackage{lmodern}
\usepackage{graphicx}
\usepackage[letterpaper]{geometry}
\usepackage[americanvoltages,americancurrents, siunitx]{circuitikz}
\usetikzlibrary{babel}
\usepackage{amsmath}
\usepackage{caption}
\usepackage{subfig}
\usepackage{hyperref}
\usepackage[all]{hypcap}


\captionsetup{font=small,labelfont=bf,margin=0.1\textwidth}
\pagestyle{myheadings}
\markboth{GPH-2006/PHY-2002~---~Analyse~fréquentielle}{GPH-2006/PHY-2002~---~Analyse~fréquentielle}


\begin{document}


\title{\textbf{Complément}\\Analyse fréquentielle}
\author{Jean-Raphaël Carrier \& Claudine Allen}
\date{}
\maketitle


\section{Bande passante et fréquence de coupure}

Dans plusieurs applications, il peut advenir que la réponse d'un système varie avec la fréquence. Un condensateur en est un exemple. À basses fréquences, aucun courant ne traverse le condensateur et celui-ci se comporte comme un circuit ouvert. À l'opposé, il se comporte comme un court-circuit pour les hautes fréquences, c'est-à-dire qu'il ne s'oppose plus au passage du courant. En traçant un graphique du courant ou de la tension en fonction de la fréquence, il est possible de voir une transition entre ces deux comportements. La fréquence correspondant à cette transition s'appelle la \textit{fréquence de coupure}.

La bande passante d'un système, ou d'un signal, est l'intervalle de fréquences situé entre les fréquences de coupure. En d'autres mots, la bande passante est l'intervalle de fréquences où l'intensité est supérieure à un certain seuil. L'expression \textit{bande passante} est aussi utilisée pour décrire la largeur de cet intervalle. La bande passante est alors mesurée en hertz (Hz) et est dénotée $B$ (pour \textit{bandwidth}). Si les fréquences de coupure d'un signal sont $f_{c1}$ et $f_{c2}$, alors la largeur de la bande passante vaut:
\begin{equation}
B=f_{c2}-f_{c1}.
\end{equation}

La largeur de la bande passante dépend des fréquences de coupure, qui elles-mêmes dépendent de la définition du seuil d'intensité au-delà duquel le signal est considéré comme trop faible pour faire partie de la bande passante. Généralement, les fréquences de coupure équivalent à une perte d'intensité de 3~dB, c'est-à-dire aux endroits où l'amplitude du signal (tension, courant) vaut $2^{-½}$ de son amplitude maximale. Plus rarement, la bande passante peut être définie comme l'intervalle entre les fréquences où l'intensité a chuté de 6~dB ; l'amplitude est alors égale à la moitié de sa valeur maximale.

\subsection{Décibel}

Le décibel (dB) est une unité de mesure logarithmique du rapport entre deux puissances. Ainsi, la valeur en décibels d'une puissance $P$, par rapport à une puissance de référence $P_0$, est:
\begin{equation}
X_{dB}=10 \, \mathrm{log}_{10}\left( \frac{P}{P_0} \right).
\end{equation}

Lorsque la puissance est proportionnelle au carré d'une autre grandeur (c'est le cas du courant et de la tension dans une résistance, puisque $P=R\,I^2=V^2/R$), la valeur en décibels peut être obtenue en comparant directement les deux amplitudes de cette grandeur. Par exemple, pour la tension, cela donne:
\begin{equation}
X_{dB}=10 \, \mathrm{log}_{10}\left( \frac{P}{P_0} \right)=10 \, \mathrm{log}_{10}\left( \frac{V^2}{V_0^2} \right)=20 \, \mathrm{log}_{10}\left( \frac{V}{V_0} \right).
\end{equation}


\section{Filtres}

Les filtres permettent d'atténuer certaines fréquences d'un signal tout en laissant passer les autres sans atténuation. Ils sont très utiles pour effacer certains bruits et signaux parasites, afin de ne conserver qu'un signal choisi.

Il existe différentes sortes de filtres ; leur application dépend des fréquences voulant être conservées/atténuées. Un filtre passe-bas, comme son nom l'indique, laisse passer les basses fréquences (celles qui sont inférieures à sa fréquence de coupure) et atténue significativement les fréquences plus élevées. Un filtre passe-haut agit à l'opposé.

Un filtre passe-bande est en quelque-sorte la combinaison de ces deux filtres : les fréquences se situant entre les deux fréquences de coupure du filtre passent sans atténuation significative, alors que les fréquences inférieures et supérieures sont effacées. Un filtre coupe-bande agit à l'opposé : il laisse passer toutes les fréquences sauf celles situées dans la fourchette définie par ses deux fréquences de coupure.


\subsection{Exemple --- Circuit RC en série}
\label{exemple-RC}

Soit le circuit RC illustré à la figure~\ref{circuitRC-serie}.

\begin{figure}[h]
\begin{center}
\begin{circuitikz} \draw
(0,3) to[V, l_=$V_S$~(\textsc{ac})] (0,0)
(0,3) to[C=$C$] (3,3) to[R=$R$] 
(3,0) to[short] (0,0)
;\end{circuitikz}
\end{center}
\caption{\label{circuitRC-serie}Circuit RC en série.}
\end{figure}

En considérant ce circuit comme un diviseur de tension, les tensions aux bornes du condensateur et de la résistance sont obtenues en multipliant la tension fournie par la source par la \textit{fonction de transfert} de chaque composant.
\begin{gather}
V_C=H_C\,V_S=\frac{Z_C}{Z_C+Z_R}\,V_S\\
V_R=H_R\,V_S=\frac{Z_R}{Z_C+Z_R}\,V_S
\end{gather}

Or, puisque l'impédance du condensateur dépend de la fréquence de la source, il en résulte que les deux fonctions de transfert, donc les deux tensions, varieront aussi en fonction de la fréquence. Les fonctions de transfert peuvent s'écrire:
\begin{gather}
H_C\!\left(j\,\omega\right)=\frac{1}{1+j\,\omega\,R\,C}\\
H_R\!\left(j\,\omega\right)=\frac{j\,\omega\,R\,C}{1+j\,\omega\,R\,C}.
\end{gather}

Pour un dipôle, la fonction de transfert peut être vue comme un phaseur, c'est-à-dire comme une grandeur ayant une amplitude (qu'on appelle le gain) et une phase.
\begin{equation}
H\!\left(j\,\omega\right)=G\!\left(\omega\right)\,\mathrm{e}^{j\,\varphi}
\end{equation}
Dans le cas du présent circuit RC, les gains et les phases du condensateur et de la résistance sont les suivants:
\begin{subequations}
\begin{gather}
G_C=\frac{1}{\sqrt{1+\left(\omega\,R\,C\right)^2}}\\
\varphi_C=\mathrm{arctan}\!\left(-\omega\,R\,C\right)
\end{gather}
\end{subequations}
\begin{subequations}
\begin{gather}
G_R=\frac{\omega\,R\,C}{\sqrt{1+\left(\omega\,R\,C\right)^2}}\\
\varphi_R=\mathrm{arctan}\!\left(\frac{1}{\omega\,R\,C}\right).
\end{gather}
\end{subequations}

En analysant les deux équations des gains, il est possible de constater que lorsque $\omega\,\rightarrow\,0$, $G_C\,\rightarrow\,1$ et $G_R\,\rightarrow\,0$. L'inverse survient lorsque $\omega\,\rightarrow\,\infty$. Ainsi, si la sortie du diviseur de tension est prise aux bornes du condensateur, les hautes fréquences seront atténuées mais les basses fréquences ne le seront pas : le circuit agit alors comme un filtre passe-bas. À l'opposé, si la sortie est prise aux bornes de la résistance, le diviseur de tension se comportera comme un filtre passe-haut.

La fréquence de coupure (à --3 dB) est celle où les gains valent $2^{-½}$:
\begin{equation}
G_C=G_R=\frac{1}{\sqrt{2}}.
\end{equation}
La fréquence de coupure est alors:
\begin{subequations}
\begin{gather}
\omega_c=\frac{1}{R\,C}\\
f_c=\frac{1}{2\,\pi\,R\,C}.
\end{gather}
\end{subequations}

Tous les filtres passifs peuvent être analysés de la même façon, à partir des impédances.


\subsection{Exemples de filtres passifs}

Il a été montré dans l'exemple précédent qu'un circuit RC peut être utilisé comme filtre passe-bas ou passe-haut, tout dépendamment aux bornes de quel composant la sortie est prise. Le filtrage provient directement de la réponse en fréquence du condensateur ; il ne laisse pas passer un courant continu mais va permettre le passage à un courant alternatif. C'est le même principe, mais à l'opposé, pour un circuit RL. Une bobine va laisser passer facilement un courant continu, mais elle s'opposera au passage d'un courant alternatif. Dans le cas d'un filtre RL, la fréquence de coupure est donnée par:
\begin{equation}
f_c=\frac{R}{2\,\pi\,L}.
\end{equation}

En plaçant un filtre passe-haut et un filtre passe-bas en série, cela forme un filtre passe-bande. En combinant un filtre passe-haut et passe-bas en parallèle, on obtient un filtre coupe-bande. Encore une fois, le filtrage découle directement de la réponse en fréquence des composants (C ou L).

Toutefois, il existe un autre moyen de faire un filtre passe-bande ou coupe-bande : en utilisant un oscillateur. Un circuit contenant à la fois un condensateur (ou plus) et une bobine d'inductance (ou plus) est un oscillateur. Un circuit de la sorte est appelé circuit RLC, pour résistance, inductance et capacité. Le condensateur veut favoriser les hautes fréquences alors que la bobine souhaite l'inverse, ce qui donne un comportement oscillant, similaire à celui d'un système masse-ressort. Le rôle de la résistance est de limiter ces oscillations --- elle est comparable au frottement dans le cas d'un oscillateur mécanique.

Comme tous les oscillateurs, le circuit RLC a une (ou plusieurs) fréquence naturelle. C'est près de cette fréquence naturelle que le système va entrer en résonance ; c'est donc là que la tension à la sortie sera la plus grande. À l'inverse, loin de la fréquence naturelle, le signal va être considérablement atténué.

Pour les exemples de filtres suivants, la tension de sortie est toujours prise aux bornes de la résistance. Il est à noter que les tensions aux bornes des autres composants seront complémentaires.


\subsubsection{Passe-bande}

Les circuits RLC peuvent former des filtres passe-bande. Dans ce cas, il existe deux fréquences de coupure distinctes et seules les fréquences situées entre elles ne seront pas significativement atténuées. Lorsqu'un filtre passe-bande (ou coupe-bande) provient d'un circuit RLC (i.e. un oscillateur), les deux fréquences de coupure sont situées de part et d'autre d'une fréquence centrale, $f_0$, qui correspond à la fréquence naturelle du circuit. La fréquence naturelle équivaut à la moyenne \textit{géométrique} des deux fréquences de coupure et non à leur moyenne \textit{algébrique}, c'est-à-dire:
\begin{equation}
f_0=\sqrt{f_{c1}\,f_{c2}}.
\end{equation}

Les figures~\ref{RLC-passe-bande1} et \ref{RLC-passe-bande2} montrent deux configurations d'un filtre passe-bande. Dans les deux cas, la fréquence naturelle de la bande passante est:
\begin{equation}
\label{eq-passe-bande1}
f_0=\frac{1}{2\,\pi\,\sqrt{L\,C}}.
\end{equation}
La largeur de la bande passante est aussi la même dans les deux cas:
\begin{equation}
\label{eq-passe-bande2}
B=\Delta f=\frac{R}{2\,\pi\,L}.
\end{equation}

\begin{figure}[h]
\begin{center}
\begin{circuitikz} \draw
(0,0) to[short,o-o] (6,0)
(0,2) to[short,o-] (1,2) to[L=$L$] (3,2) to[C=$C$] (5,2) to[short,-o] (6,2)
(5,2) to[R=$R$] (5,0) node[ground]{}
{[anchor=east] (0,2) node{$V_{\mathrm{in}}$~+} (0,0) node{--}}
{[anchor=west] (6,2) node{+~$V_{\mathrm{out}}$} (6,0) node{--}}
;\end{circuitikz}
\end{center}
\caption{\label{RLC-passe-bande1}Circuit RLC utilisé comme filtre passe-bande.}
\end{figure}

\begin{figure}[h]
\begin{center}
\begin{circuitikz} \draw
(0,0) to[short,o-o] (6,0)
(0,2) to[short,o-o] (6,2)
(1,2) to[L=$L$] (1,0) node[ground]{}
(3,2) to[C=$C$] (3,0)
(5,2) to[R=$R$] (5,0)
{[anchor=east] (0,2) node{$V_{\mathrm{in}}$~+} (0,0) node{--}}
{[anchor=west] (6,2) node{+~$V_{\mathrm{out}}$} (6,0) node{--}}
;\end{circuitikz}
\end{center}
\caption{\label{RLC-passe-bande2}Autre circuit RLC utilisé comme filtre passe-bande.}
\end{figure}


\subsubsection{Coupe-bande}

Il est aussi possible, avec un circuit RLC, d'obtenir un filtre coupe-bande. Les figures~\ref{RLC-coupe-bande1} et \ref{RLC-coupe-bande2} en illustrent deux exemples. Pour ces deux circuits, la fréquence naturelle et la largeur de la bande passante sont les mêmes que pour les deux filtres passe-bande montrés dans la section précédente (équations~\ref{eq-passe-bande1} et \ref{eq-passe-bande2}). La seule différence réside dans le fait que la fréquence centrale $f_0$ correspond cette fois à une \textit{anti-résonance}, puisque c'est à cette fréquence que le signal sera le plus atténué.

\begin{figure}[h]
\begin{center}
\begin{circuitikz} \draw
(0,0) to[short,o-o] (4,0)
(0,3) to[short,o-o] (4,3)
(1,3) to[L=$L$] (1,1.2) to[C=$C$] (1,0) node[ground]{}
(3,3) to[R=$R$] (3,0)
{[anchor=east] (0,3) node{$V_{\mathrm{in}}$~+} (0,0) node{--}}
{[anchor=west] (4,3) node{+~$V_{\mathrm{out}}$} (4,0) node{--}}
;\end{circuitikz}
\end{center}
\caption{\label{RLC-coupe-bande1}Circuit RLC utilisé comme filtre coupe-bande.}
\end{figure}

\begin{figure}[h]
\begin{center}
\begin{circuitikz} \draw
(0,0) to[short,o-o] (5,0)
(0,2) to[short,o-] (1,2) to[short] (1,2.5) to[L=$L$] (3,2.5) to[short] (3,1.5) to[C=$C$] (1,1.5) to[short] (1,2)
(3,2) to[short,-o] (5,2)
(4,2) to[R=$R$] (4,0) node[ground]{}
{[anchor=east] (0,2) node{$V_{\mathrm{in}}$~+} (0,0) node{--}}
{[anchor=west] (5,2) node{+~$V_{\mathrm{out}}$} (5,0) node{--}}
;\end{circuitikz}
\end{center}
\caption{\label{RLC-coupe-bande2}Autre circuit RLC utilisé comme filtre coupe-bande.}
\end{figure}


\end{document}

Écrit par Jean-Raphaël Carrier
Dernière modification : 12 janvier 2014