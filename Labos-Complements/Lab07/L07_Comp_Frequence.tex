%Créé par Claudine Allen en collaboration avec Jean-Raphaël Carrier
%Dernière modification JRC: 12 janvier 2014
%Dernière modification CA: 15 novembre 2020
%***ToDo***
%Mieux connecter au cours, tourner en notes de cours avec les 2 compléments d'analyse temporelle et fréquentielle.
%Inverser intro aux dB (aussi échelle logarithmique en général, quoique ce serait bien avant le lab III pour cette analyse) et Bande passante-fréquence de coupure pour clarifier connexion -3 dB et $2^{-1/2}$ linéaire en amplitude. Peut-être même la notion de filtres avant ça... en fait revoir toute la structure pour mieux fitter avec le cours! P.S. Ça $\frac{1}{\sqrt{2}}=2^{-1/2}$ c'est pas beau "inline" dans le texte, garder seulement en "displaystyle" et/ou dans les diapos.
%Mieux connecter phaseurs au vocabulaire de PMII en notation polaire de nombres complexes: faire un tableau. Bienvenue dans la multidisciplinarité inhérente à la physique!
%Rattacher un bout de texte avec le travail expérimental sur oscilloscope qui a été fait avec la FFT? Bout de blabla déjà commencé à ce sujet: "un signal réel mesuré dans le temps, par exemple $v(t)$, devient...FFT oscilloscope similaire, mettre photo de mesure actuelle (attention illustration pédagogique, pas analyse!)... distinguer circuit système (RÉPONSE en fréquence fct de transfert Vout/Vin) vs signal (Vout pour un Vin spécifique)." See wiki Bode plots, spectral density,etc. et surtout retrouver les notes de systèmes et mesures expériementales.
%Rajouter illustration (geogebra->TikZ) des cas idéaux de chaque type de filtres (transitions verticales) dans le coin des diagrammes de Bode puis introduire déviation en pente dB/octave ou dB/décade (https://en.wikipedia.org/wiki/Roll-off , https://learnabout-electronics.org/ac_theory/filters83.php), ça va mieux connecter avec le dernier sous-objectif de l'atelier.
%
\RequirePackage[l2tabu, orthodox]{nag} %Check for obsolete commands
\documentclass[canadien,12pt,oneside,letterpaper]{article}
%
%-----------------------------------------------------
%Loading packages
%
\usepackage[utf8]{inputenc}
\usepackage[T1]{fontenc}
\usepackage[canadien]{babel}
\usepackage{lmodern}
\usepackage{textcomp}
\usepackage{amsmath,amssymb}
\usepackage{siunitx}
\usepackage{xcolor}
\usepackage{hyperref}
\usepackage[all]{hypcap}
\usepackage{graphicx}
\usepackage[oldvoltagedirection,americanvoltages,americancurrents,siunitx]{circuitikz}
\usetikzlibrary{babel}
\usepackage{pgfplots}
\pgfplotsset{compat=1.15}
\usepackage{mathrsfs}
\usetikzlibrary{arrows}
\usepackage{caption}
%\usepackage{subcaption}
\usepackage{subfig}
\usepackage[letterpaper,headheight=15pt]{geometry}
\usepackage{fancyhdr}
\usepackage{setspace}
%
\captionsetup{font=small,labelfont=bf,margin=0.1\textwidth}
\pagestyle{myheadings}
\markboth{PHY-2002/GPH-2006~---~Analyse~fréquentielle}{PHY-2002/GPH-2006~---~Analyse~fréquentielle}

\begin{document}
\definecolor{zzttff}{rgb}{0.6,0.2,1}

\title{\textbf{Complément}\\Analyse fréquentielle}
\author{Jean-Raphaël Carrier \& Claudine Allen}
\date{}
\maketitle


\section{Bande passante et fréquence de coupure}

Dans plusieurs applications, il peut advenir que la réponse d'un système varie avec la fréquence. Un condensateur en est un exemple. À basses fréquences, aucun courant ne traverse le condensateur et celui-ci se comporte comme un circuit ouvert. À l'opposé, il se comporte comme un court-circuit pour les hautes fréquences, c'est-à-dire qu'il ne s'oppose plus au passage du courant. En traçant un graphique du courant ou de la tension en fonction de la fréquence, il est possible de voir une transition entre ces deux comportements. La fréquence correspondant à cette transition s'appelle la \textit{fréquence de coupure} $f_c$.

La bande passante d'un système, ou d'un signal, est l'intervalle de fréquences situé entre les fréquences de coupure. En d'autres mots, la bande passante est l'intervalle de fréquences où l'intensité est supérieure à un certain seuil. L'expression \textit{bande passante} est aussi utilisée pour décrire la largeur de cet intervalle. La bande passante est alors mesurée en hertz (Hz) et est dénotée $B$ (pour \textit{bandwidth}). Si les fréquences de coupure d'un signal sont $f_{c1}$ et $f_{c2}$, alors la largeur de la bande passante vaut:
\begin{equation}
B=f_{c2}-f_{c1}.
\end{equation}
%
La largeur de la bande passante dépend des fréquences de coupure, qui elles-mêmes dépendent de la définition du seuil d'intensité au-delà duquel le signal est considéré comme trop faible pour faire partie de la bande passante. Généralement, les fréquences de coupure équivalent à une perte d'intensité de 3~dB, c'est-à-dire aux endroits où l'amplitude du signal (tension, courant) vaut $2^{\,\text{-\textonehalf}}$ de son amplitude maximale\footnote{Plus rarement, la bande passante peut être définie comme l'intervalle entre les fréquences où l'intensité a chuté de 6~dB ; l'amplitude est alors égale à la moitié de sa valeur maximale.}.

\subsection{Décibel}

Le décibel (dB) est une unité de mesure logarithmique du rapport entre deux puissances. Ainsi, la valeur en décibels d'une puissance $P$, par rapport à une puissance de référence $P_0$, est:
\begin{equation}
X_{dB}=10 \, \mathrm{log}_{10}\left( \frac{P}{P_0} \right).
\end{equation}

Lorsque la puissance est proportionnelle au carré d'une autre grandeur (c'est le cas du courant et de la tension dans une résistance, puisque $P=R\,I^2=V^2/R$), la valeur en décibels peut être obtenue en comparant directement les deux amplitudes de cette grandeur. Par exemple, pour la tension, cela donne:
\begin{equation}
X_{dB}=10 \, \mathrm{log}_{10}\left( \frac{P}{P_0} \right)=10 \, \mathrm{log}_{10}\left( \frac{V^2}{V_0^2} \right)=20 \, \mathrm{log}_{10}\left( \frac{V}{V_0} \right).
\end{equation}


\section{Filtres}

Les filtres permettent d'atténuer certaines fréquences d'un signal tout en laissant passer les autres sans atténuation. Ils sont très utiles pour effacer certains bruits et signaux parasites, afin de ne conserver qu'un signal choisi.

Il existe différentes sortes de filtres ; leur application dépend des fréquences voulant être conservées/atténuées. Un filtre passe-bas, comme son nom l'indique, laisse passer les basses fréquences (celles qui sont inférieures à sa fréquence de coupure $f_c$) et atténue significativement les fréquences plus élevées. Un filtre passe-haut agit à l'opposé. Un filtre passe-bande est en quelque-sorte la combinaison de ces deux filtres : les fréquences se situant entre les deux fréquences de coupure du filtre passent sans atténuation significative, alors que les fréquences inférieures et supérieures sont effacées. Un filtre coupe-bande agit à l'opposé : il laisse passer toutes les fréquences sauf celles situées dans la fourchette définie par ses deux fréquences de coupure.

%%DÉBUT RÉVISION JÉRÉMIE%%
Ces fonctions de filtrage avec les circuits se visualisent clairement à l'aide de spectres, c'est-à-dire de graphiques en fonction de la fréquence\footnote{Le terme fréquence sera souvent utilisé pour désigner la fréquence angulaire ou pulsation $\omega$. Lorsqu'il y a lieu de référer à la fréquence étalonnée en \si{\hertz}, le symbole $f$ est indiqué.} $\omega=2\pi f$ (et par extension de l'énergie dans d'autres domaines de la physique). Il faut donc passer de la variable indépendante du temps (domaine temporel) au domaine fréquentiel avec une fonction complexe de la fréquence $\omega$. L'outil théorique pour ce faire est la transformée de Fourier, alors qu'expérimentalement, un signal sinusoïdal de fréquence, amplitude et phase connues, par exemple $v_{\mathrm{in}}(t)$, peut être directement envoyé dans un système pour ensuite mesurer le changement d'amplitude et le déphasage\footnote{Éléments de réflexion: la fréquence n'est pas indiquée dans les mesures à la sortie, pourquoi cela n'est pas nécessaire? Est-ce que la fréquence d'un signal, d'une onde peut changer dans un système linéaire indépendant du temps?} à une sortie, par exemple $v_{\mathrm{out}}(t)$. Ces deux grandeurs constituent alors un nombre complexe en notation polaire et deviennent des variables dépendantes lorsque la fréquence du signal est balayée pour en faire la variable indépendante de cette fonction complexe dans le domaine de Fourier.

Pour des circuits électriques et systèmes de contrôle, ces spectres de réponse en fréquence portent le nom de diagrammes de Bode. Spécifiquement, il s'agit de deux graphes, celui du gain $G(\omega)=\left|H\!\left(\omega\right)\right|$ et celui de la phase $\varphi(\omega)=\arg\left( H\!\left(\omega\right)\right)$ tels qu'introduits dans le cours à partir de la fonction de transfert en régime permanent, soit 
\begin{equation} \label{eq:GainPhase}
H\!\left(\omega\right)=G\!\left(\omega\right)\,\mathrm{e}^{j\,\varphi(\omega)}\, .
\end{equation}
Expérimentalement, l'axe du gain en amplitude est souvent gradué en décibels selon \[G\!\left(f\right)=20\,\textrm{log}_{10}\left(\frac{V_{\mathrm{out}}(f)}{V_{\mathrm{in}}(f)}\right)\]
avec une abscisse logarithmique afin de couvrir un large intervalle en fréquence $f$. Particulièrement dans les domaines acoustique et radio, ce logarithme peut être en base 2 pour graduer l'axe à chaque octave en représentant des doublements de la fréquence fondamentale. Pour le diagramme de Bode de la phase $\phi\!\left(f\right)$, le déphasage est généralement calculé par rapport à la source qui envoie le signal d'entrée $V_{\mathrm{in}}(f)$.

Le diagramme de Bode du gain d'un filtre passe-bas idéal, \textit{i.e.} où la transition de passer à couper un signal à la fréquence de coupure $f_c$ est verticale (pente $\infty$), s'illustre donc ainsi:
\begin{figure}[h]
\centering
\begin{tikzpicture}[line cap=round,line join=round,>=triangle 45,x=1cm,y=1.5cm]
\begin{axis}[
x=1cm,y=1.5cm,
axis lines=middle,
xmin=0,%-0.7345564875119025
xmax=6, %6.7072821881861815
ymin=-0.05, %-0.11784989429436837
ymax=1.4, %1.1536179602953642
xtick={0,1,...,6},
xticklabels={,,}
ytick={0,0.2,...,1},
yticklabels={,,},
xlabel={Fréquence [Hz]},
ylabel={Gain [dB]}
]
\clip(-0.7345564875119025,-0.11784989429436837) rectangle (6.7072821881861815,1.1536179602953642);
\draw[line width=4pt,color=zzttff] (-0.7345564875119025,1) -- (-0.7345564875119025,1);
\draw[line width=4pt,color=zzttff] (-0.7345564875119025,1) -- (-0.7159518908226573,1);
\draw[line width=4pt,color=zzttff] (-0.7159518908226573,1) -- (-0.6973472941334121,1);
\draw[line width=4pt,color=zzttff] (-0.6973472941334121,1) -- (-0.6787426974441669,1);
\draw[line width=4pt,color=zzttff] (-0.6787426974441669,1) -- (-0.6601381007549217,1);
\draw[line width=4pt,color=zzttff] (-0.6601381007549217,1) -- (-0.6415335040656766,1);
\draw[line width=4pt,color=zzttff] (-0.6415335040656766,1) -- (-0.6229289073764314,1);
\draw[line width=4pt,color=zzttff] (-0.6229289073764314,1) -- (-0.6043243106871862,1);
\draw[line width=4pt,color=zzttff] (-0.6043243106871862,1) -- (-0.585719713997941,1);
\draw[line width=4pt,color=zzttff] (-0.585719713997941,1) -- (-0.5671151173086958,1);
\draw[line width=4pt,color=zzttff] (-0.5671151173086958,1) -- (-0.5485105206194506,1);
\draw[line width=4pt,color=zzttff] (-0.5485105206194506,1) -- (-0.5299059239302054,1);
\draw[line width=4pt,color=zzttff] (-0.5299059239302054,1) -- (-0.5113013272409602,1);
\draw[line width=4pt,color=zzttff] (-0.5113013272409602,1) -- (-0.49269673055171503,1);
\draw[line width=4pt,color=zzttff] (-0.49269673055171503,1) -- (-0.47409213386246984,1);
\draw[line width=4pt,color=zzttff] (-0.47409213386246984,1) -- (-0.45548753717322465,1);
\draw[line width=4pt,color=zzttff] (-0.45548753717322465,1) -- (-0.43688294048397947,1);
\draw[line width=4pt,color=zzttff] (-0.43688294048397947,1) -- (-0.4182783437947343,1);
\draw[line width=4pt,color=zzttff] (-0.4182783437947343,1) -- (-0.3996737471054891,1);
\draw[line width=4pt,color=zzttff] (-0.3996737471054891,1) -- (-0.3810691504162439,1);
\draw[line width=4pt,color=zzttff] (-0.3810691504162439,1) -- (-0.3624645537269987,1);
\draw[line width=4pt,color=zzttff] (-0.3624645537269987,1) -- (-0.3438599570377535,1);
\draw[line width=4pt,color=zzttff] (-0.3438599570377535,1) -- (-0.3252553603485083,1);
\draw[line width=4pt,color=zzttff] (-0.3252553603485083,1) -- (-0.30665076365926314,1);
\draw[line width=4pt,color=zzttff] (-0.30665076365926314,1) -- (-0.28804616697001795,1);
\draw[line width=4pt,color=zzttff] (-0.28804616697001795,1) -- (-0.26944157028077276,1);
\draw[line width=4pt,color=zzttff] (-0.26944157028077276,1) -- (-0.25083697359152757,1);
\draw[line width=4pt,color=zzttff] (-0.25083697359152757,1) -- (-0.23223237690228235,1);
\draw[line width=4pt,color=zzttff] (-0.23223237690228235,1) -- (-0.21362778021303713,1);
\draw[line width=4pt,color=zzttff] (-0.21362778021303713,1) -- (-0.19502318352379192,1);
\draw[line width=4pt,color=zzttff] (-0.19502318352379192,1) -- (-0.1764185868345467,1);
\draw[line width=4pt,color=zzttff] (-0.1764185868345467,1) -- (-0.15781399014530148,1);
\draw[line width=4pt,color=zzttff] (-0.15781399014530148,1) -- (-0.13920939345605626,1);
\draw[line width=4pt,color=zzttff] (-0.13920939345605626,1) -- (-0.12060479676681105,1);
\draw[line width=4pt,color=zzttff] (-0.12060479676681105,1) -- (-0.10200020007756583,1);
\draw[line width=4pt,color=zzttff] (-0.10200020007756583,1) -- (-0.08339560338832061,1);
\draw[line width=4pt,color=zzttff] (-0.08339560338832061,1) -- (-0.06479100669907539,1);
\draw[line width=4pt,color=zzttff] (-0.06479100669907539,1) -- (-0.04618641000983018,1);
\draw[line width=4pt,color=zzttff] (-0.04618641000983018,1) -- (-0.027581813320584972,1);
\draw[line width=4pt,color=zzttff] (-0.027581813320584972,1) -- (-0.008977216631339761,1);
\draw[line width=4pt,color=zzttff] (-0.008977216631339761,1) -- (0.00962738005790545,1);
\draw[line width=4pt,color=zzttff] (0.00962738005790545,1) -- (0.02823197674715066,1);
\draw[line width=4pt,color=zzttff] (0.02823197674715066,1) -- (0.04683657343639587,1);
\draw[line width=4pt,color=zzttff] (0.04683657343639587,1) -- (0.06544117012564107,1);
\draw[line width=4pt,color=zzttff] (0.06544117012564107,1) -- (0.08404576681488629,1);
\draw[line width=4pt,color=zzttff] (0.08404576681488629,1) -- (0.10265036350413151,1);
\draw[line width=4pt,color=zzttff] (0.10265036350413151,1) -- (0.12125496019337673,1);
\draw[line width=4pt,color=zzttff] (0.12125496019337673,1) -- (0.13985955688262194,1);
\draw[line width=4pt,color=zzttff] (0.13985955688262194,1) -- (0.15846415357186716,1);
\draw[line width=4pt,color=zzttff] (0.15846415357186716,1) -- (0.17706875026111238,1);
\draw[line width=4pt,color=zzttff] (0.17706875026111238,1) -- (0.1956733469503576,1);
\draw[line width=4pt,color=zzttff] (0.1956733469503576,1) -- (0.2142779436396028,1);
\draw[line width=4pt,color=zzttff] (0.2142779436396028,1) -- (0.23288254032884803,1);
\draw[line width=4pt,color=zzttff] (0.23288254032884803,1) -- (0.2514871370180932,1);
\draw[line width=4pt,color=zzttff] (0.2514871370180932,1) -- (0.2700917337073384,1);
\draw[line width=4pt,color=zzttff] (0.2700917337073384,1) -- (0.2886963303965836,1);
\draw[line width=4pt,color=zzttff] (0.2886963303965836,1) -- (0.3073009270858288,1);
\draw[line width=4pt,color=zzttff] (0.3073009270858288,1) -- (0.325905523775074,1);
\draw[line width=4pt,color=zzttff] (0.325905523775074,1) -- (0.34451012046431917,1);
\draw[line width=4pt,color=zzttff] (0.34451012046431917,1) -- (0.36311471715356436,1);
\draw[line width=4pt,color=zzttff] (0.36311471715356436,1) -- (0.38171931384280955,1);
\draw[line width=4pt,color=zzttff] (0.38171931384280955,1) -- (0.40032391053205474,1);
\draw[line width=4pt,color=zzttff] (0.40032391053205474,1) -- (0.41892850722129993,1);
\draw[line width=4pt,color=zzttff] (0.41892850722129993,1) -- (0.4375331039105451,1);
\draw[line width=4pt,color=zzttff] (0.4375331039105451,1) -- (0.4561377005997903,1);
\draw[line width=4pt,color=zzttff] (0.4561377005997903,1) -- (0.4747422972890355,1);
\draw[line width=4pt,color=zzttff] (0.4747422972890355,1) -- (0.4933468939782807,1);
\draw[line width=4pt,color=zzttff] (0.4933468939782807,1) -- (0.5119514906675259,1);
\draw[line width=4pt,color=zzttff] (0.5119514906675259,1) -- (0.5305560873567711,1);
\draw[line width=4pt,color=zzttff] (0.5305560873567711,1) -- (0.5491606840460163,1);
\draw[line width=4pt,color=zzttff] (0.5491606840460163,1) -- (0.5677652807352614,1);
\draw[line width=4pt,color=zzttff] (0.5677652807352614,1) -- (0.5863698774245066,1);
\draw[line width=4pt,color=zzttff] (0.5863698774245066,1) -- (0.6049744741137518,1);
\draw[line width=4pt,color=zzttff] (0.6049744741137518,1) -- (0.623579070802997,1);
\draw[line width=4pt,color=zzttff] (0.623579070802997,1) -- (0.6421836674922422,1);
\draw[line width=4pt,color=zzttff] (0.6421836674922422,1) -- (0.6607882641814874,1);
\draw[line width=4pt,color=zzttff] (0.6607882641814874,1) -- (0.6793928608707326,1);
\draw[line width=4pt,color=zzttff] (0.6793928608707326,1) -- (0.6979974575599778,1);
\draw[line width=4pt,color=zzttff] (0.6979974575599778,1) -- (0.716602054249223,1);
\draw[line width=4pt,color=zzttff] (0.716602054249223,1) -- (0.7352066509384682,1);
\draw[line width=4pt,color=zzttff] (0.7352066509384682,1) -- (0.7538112476277133,1);
\draw[line width=4pt,color=zzttff] (0.7538112476277133,1) -- (0.7724158443169585,1);
\draw[line width=4pt,color=zzttff] (0.7724158443169585,1) -- (0.7910204410062037,1);
\draw[line width=4pt,color=zzttff] (0.7910204410062037,1) -- (0.8096250376954489,1);
\draw[line width=4pt,color=zzttff] (0.8096250376954489,1) -- (0.8282296343846941,1);
\draw[line width=4pt,color=zzttff] (0.8282296343846941,1) -- (0.8468342310739393,1);
\draw[line width=4pt,color=zzttff] (0.8468342310739393,1) -- (0.8654388277631845,1);
\draw[line width=4pt,color=zzttff] (0.8654388277631845,1) -- (0.8840434244524297,1);
\draw[line width=4pt,color=zzttff] (0.8840434244524297,1) -- (0.9026480211416749,1);
\draw[line width=4pt,color=zzttff] (0.9026480211416749,1) -- (0.92125261783092,1);
\draw[line width=4pt,color=zzttff] (0.92125261783092,1) -- (0.9398572145201652,1);
\draw[line width=4pt,color=zzttff] (0.9398572145201652,1) -- (0.9584618112094104,1);
\draw[line width=4pt,color=zzttff] (0.9584618112094104,1) -- (0.9770664078986556,1);
\draw[line width=4pt,color=zzttff] (0.9770664078986556,1) -- (0.9956710045879008,1);
\draw[line width=4pt,color=zzttff] (0.9956710045879008,1) -- (1.014275601277146,1);
\draw[line width=4pt,color=zzttff] (1.014275601277146,1) -- (1.0328801979663913,1);
\draw[line width=4pt,color=zzttff] (1.0328801979663913,1) -- (1.0514847946556365,1);
\draw[line width=4pt,color=zzttff] (1.0514847946556365,1) -- (1.0700893913448817,1);
\draw[line width=4pt,color=zzttff] (1.0700893913448817,1) -- (1.0886939880341269,1);
\draw[line width=4pt,color=zzttff] (1.0886939880341269,1) -- (1.107298584723372,1);
\draw[line width=4pt,color=zzttff] (1.107298584723372,1) -- (1.1259031814126172,1);
\draw[line width=4pt,color=zzttff] (1.1259031814126172,1) -- (1.1445077781018624,1);
\draw[line width=4pt,color=zzttff] (1.1445077781018624,1) -- (1.1631123747911076,1);
\draw[line width=4pt,color=zzttff] (1.1631123747911076,1) -- (1.1817169714803528,1);
\draw[line width=4pt,color=zzttff] (1.1817169714803528,1) -- (1.200321568169598,1);
\draw[line width=4pt,color=zzttff] (1.200321568169598,1) -- (1.2189261648588432,1);
\draw[line width=4pt,color=zzttff] (1.2189261648588432,1) -- (1.2375307615480884,1);
\draw[line width=4pt,color=zzttff] (1.2375307615480884,1) -- (1.2561353582373336,1);
\draw[line width=4pt,color=zzttff] (1.2561353582373336,1) -- (1.2747399549265788,1);
\draw[line width=4pt,color=zzttff] (1.2747399549265788,1) -- (1.293344551615824,1);
\draw[line width=4pt,color=zzttff] (1.293344551615824,1) -- (1.3119491483050691,1);
\draw[line width=4pt,color=zzttff] (1.3119491483050691,1) -- (1.3305537449943143,1);
\draw[line width=4pt,color=zzttff] (1.3305537449943143,1) -- (1.3491583416835595,1);
\draw[line width=4pt,color=zzttff] (1.3491583416835595,1) -- (1.3677629383728047,1);
\draw[line width=4pt,color=zzttff] (1.3677629383728047,1) -- (1.38636753506205,1);
\draw[line width=4pt,color=zzttff] (1.38636753506205,1) -- (1.404972131751295,1);
\draw[line width=4pt,color=zzttff] (1.404972131751295,1) -- (1.4235767284405403,1);
\draw[line width=4pt,color=zzttff] (1.4235767284405403,1) -- (1.4421813251297855,1);
\draw[line width=4pt,color=zzttff] (1.4421813251297855,1) -- (1.4607859218190307,1);
\draw[line width=4pt,color=zzttff] (1.4607859218190307,1) -- (1.4793905185082759,1);
\draw[line width=4pt,color=zzttff] (1.4793905185082759,1) -- (1.497995115197521,1);
\draw[line width=4pt,color=zzttff] (1.497995115197521,1) -- (1.5165997118867662,1);
\draw[line width=4pt,color=zzttff] (1.5165997118867662,1) -- (1.5352043085760114,1);
\draw[line width=4pt,color=zzttff] (1.5352043085760114,1) -- (1.5538089052652566,1);
\draw[line width=4pt,color=zzttff] (1.5538089052652566,1) -- (1.5724135019545018,1);
\draw[line width=4pt,color=zzttff] (1.5724135019545018,1) -- (1.591018098643747,1);
\draw[line width=4pt,color=zzttff] (1.591018098643747,1) -- (1.6096226953329922,1);
\draw[line width=4pt,color=zzttff] (1.6096226953329922,1) -- (1.6282272920222374,1);
\draw[line width=4pt,color=zzttff] (1.6282272920222374,1) -- (1.6468318887114826,1);
\draw[line width=4pt,color=zzttff] (1.6468318887114826,1) -- (1.6654364854007278,1);
\draw[line width=4pt,color=zzttff] (1.6654364854007278,1) -- (1.684041082089973,1);
\draw[line width=4pt,color=zzttff] (1.684041082089973,1) -- (1.7026456787792181,1);
\draw[line width=4pt,color=zzttff] (1.7026456787792181,1) -- (1.7212502754684633,1);
\draw[line width=4pt,color=zzttff] (1.7212502754684633,1) -- (1.7398548721577085,1);
\draw[line width=4pt,color=zzttff] (1.7398548721577085,1) -- (1.7584594688469537,1);
\draw[line width=4pt,color=zzttff] (1.7584594688469537,1) -- (1.777064065536199,1);
\draw[line width=4pt,color=zzttff] (1.777064065536199,1) -- (1.795668662225444,1);
\draw[line width=4pt,color=zzttff] (1.795668662225444,1) -- (1.8142732589146893,1);
\draw[line width=4pt,color=zzttff] (1.8142732589146893,1) -- (1.8328778556039345,1);
\draw[line width=4pt,color=zzttff] (1.8328778556039345,1) -- (1.8514824522931796,1);
\draw[line width=4pt,color=zzttff] (1.8514824522931796,1) -- (1.8700870489824248,1);
\draw[line width=4pt,color=zzttff] (1.8700870489824248,1) -- (1.88869164567167,1);
\draw[line width=4pt,color=zzttff] (1.88869164567167,1) -- (1.9072962423609152,1);
\draw[line width=4pt,color=zzttff] (1.9072962423609152,1) -- (1.9259008390501604,1);
\draw[line width=4pt,color=zzttff] (1.9259008390501604,1) -- (1.9445054357394056,1);
\draw[line width=4pt,color=zzttff] (1.9445054357394056,1) -- (1.9631100324286508,1);
\draw[line width=4pt,color=zzttff] (1.9631100324286508,1) -- (1.981714629117896,1);
\draw[line width=4pt,color=zzttff] (1.981714629117896,1) -- (2.000319225807141,1);
\draw[line width=4pt,color=zzttff] (2.000319225807141,1) -- (2.0189238224963866,1);
\draw[line width=4pt,color=zzttff] (2.0189238224963866,1) -- (2.037528419185632,1);
\draw[line width=4pt,color=zzttff] (2.037528419185632,1) -- (2.0561330158748774,1);
\draw[line width=4pt,color=zzttff] (2.0561330158748774,1) -- (2.074737612564123,1);
\draw[line width=4pt,color=zzttff] (2.074737612564123,1) -- (2.0933422092533682,1);
\draw[line width=4pt,color=zzttff] (2.0933422092533682,1) -- (2.1119468059426136,1);
\draw[line width=4pt,color=zzttff] (2.1119468059426136,1) -- (2.130551402631859,1);
\draw[line width=4pt,color=zzttff] (2.130551402631859,1) -- (2.1491559993211045,1);
\draw[line width=4pt,color=zzttff] (2.1491559993211045,1) -- (2.16776059601035,1);
\draw[line width=4pt,color=zzttff] (2.16776059601035,1) -- (2.1863651926995953,1);
\draw[line width=4pt,color=zzttff] (2.1863651926995953,1) -- (2.2049697893888407,1);
\draw[line width=4pt,color=zzttff] (2.2049697893888407,1) -- (2.223574386078086,1);
\draw[line width=4pt,color=zzttff] (2.223574386078086,1) -- (2.2421789827673315,1);
\draw[line width=4pt,color=zzttff] (2.2421789827673315,1) -- (2.260783579456577,1);
\draw[line width=4pt,color=zzttff] (2.260783579456577,1) -- (2.2793881761458223,1);
\draw[line width=4pt,color=zzttff] (2.2793881761458223,1) -- (2.2979927728350678,1);
\draw[line width=4pt,color=zzttff] (2.2979927728350678,1) -- (2.316597369524313,1);
\draw[line width=4pt,color=zzttff] (2.316597369524313,1) -- (2.3352019662135586,1);
\draw[line width=4pt,color=zzttff] (2.3352019662135586,1) -- (2.353806562902804,1);
\draw[line width=4pt,color=zzttff] (2.353806562902804,1) -- (2.3724111595920494,1);
\draw[line width=4pt,color=zzttff] (2.3724111595920494,1) -- (2.391015756281295,1);
\draw[line width=4pt,color=zzttff] (2.391015756281295,1) -- (2.4096203529705402,1);
\draw[line width=4pt,color=zzttff] (2.4096203529705402,1) -- (2.4282249496597856,1);
\draw[line width=4pt,color=zzttff] (2.4282249496597856,1) -- (2.446829546349031,1);
\draw[line width=4pt,color=zzttff] (2.446829546349031,1) -- (2.4654341430382765,1);
\draw[line width=4pt,color=zzttff] (2.4654341430382765,1) -- (2.484038739727522,1);
\draw[line width=4pt,color=zzttff] (2.484038739727522,1) -- (2.5026433364167673,1);
\draw[line width=4pt,color=zzttff] (2.5026433364167673,1) -- (2.5212479331060127,1);
\draw[line width=4pt,color=zzttff] (2.5212479331060127,1) -- (2.539852529795258,1);
\draw[line width=4pt,color=zzttff] (2.539852529795258,1) -- (2.5584571264845035,1);
\draw[line width=4pt,color=zzttff] (2.5584571264845035,1) -- (2.577061723173749,1);
\draw[line width=4pt,color=zzttff] (2.577061723173749,1) -- (2.5956663198629943,1);
\draw[line width=4pt,color=zzttff] (2.5956663198629943,1) -- (2.6142709165522398,1);
\draw[line width=4pt,color=zzttff] (2.6142709165522398,1) -- (2.632875513241485,1);
\draw[line width=4pt,color=zzttff] (2.632875513241485,1) -- (2.6514801099307306,1);
\draw[line width=4pt,color=zzttff] (2.6514801099307306,1) -- (2.670084706619976,1);
\draw[line width=4pt,color=zzttff] (2.670084706619976,1) -- (2.6886893033092214,1);
\draw[line width=4pt,color=zzttff] (2.6886893033092214,1) -- (2.707293899998467,1);
\draw[line width=4pt,color=zzttff] (2.707293899998467,1) -- (2.7258984966877122,1);
\draw[line width=4pt,color=zzttff] (2.7258984966877122,1) -- (2.7445030933769576,1);
\draw[line width=4pt,color=zzttff] (2.7445030933769576,1) -- (2.763107690066203,1);
\draw[line width=4pt,color=zzttff] (2.763107690066203,1) -- (2.7817122867554485,1);
\draw[line width=4pt,color=zzttff] (2.7817122867554485,1) -- (2.800316883444694,1);
\draw[line width=4pt,color=zzttff] (2.800316883444694,1) -- (2.8189214801339393,1);
\draw[line width=4pt,color=zzttff] (2.8189214801339393,1) -- (2.8375260768231847,1);
\draw[line width=4pt,color=zzttff] (2.8375260768231847,1) -- (2.85613067351243,1);
\draw[line width=4pt,color=zzttff] (2.85613067351243,1) -- (2.8747352702016755,1);
\draw[line width=4pt,color=zzttff] (2.8747352702016755,1) -- (2.893339866890921,1);
\draw[line width=4pt,color=zzttff] (2.893339866890921,1) -- (2.9119444635801663,1);
\draw[line width=4pt,color=zzttff] (2.9119444635801663,1) -- (2.9305490602694118,1);
\draw[line width=4pt,color=zzttff] (2.9305490602694118,1) -- (2.949153656958657,1);
\draw[line width=4pt,color=zzttff] (2.949153656958657,1) -- (2.9677582536479026,1);
\draw[line width=4pt,color=zzttff] (2.9677582536479026,1) -- (2.986362850337148,1);
\draw[line width=4pt,color=zzttff] (3.0049674470263934,0) -- (3.023572043715639,0);
\draw[line width=4pt,color=zzttff] (3.023572043715639,0) -- (3.0421766404048842,0);
\draw[line width=4pt,color=zzttff] (3.0421766404048842,0) -- (3.0607812370941296,0);
\draw[line width=4pt,color=zzttff] (3.0607812370941296,0) -- (3.079385833783375,0);
\draw[line width=4pt,color=zzttff] (3.079385833783375,0) -- (3.0979904304726205,0);
\draw[line width=4pt,color=zzttff] (3.0979904304726205,0) -- (3.116595027161866,0);
\draw[line width=4pt,color=zzttff] (3.116595027161866,0) -- (3.1351996238511113,0);
\draw[line width=4pt,color=zzttff] (3.1351996238511113,0) -- (3.1538042205403567,0);
\draw[line width=4pt,color=zzttff] (3.1538042205403567,0) -- (3.172408817229602,0);
\draw[line width=4pt,color=zzttff] (3.172408817229602,0) -- (3.1910134139188475,0);
\draw[line width=4pt,color=zzttff] (3.1910134139188475,0) -- (3.209618010608093,0);
\draw[line width=4pt,color=zzttff] (3.209618010608093,0) -- (3.2282226072973383,0);
\draw[line width=4pt,color=zzttff] (3.2282226072973383,0) -- (3.2468272039865838,0);
\draw[line width=4pt,color=zzttff] (3.2468272039865838,0) -- (3.265431800675829,0);
\draw[line width=4pt,color=zzttff] (3.265431800675829,0) -- (3.2840363973650746,0);
\draw[line width=4pt,color=zzttff] (3.2840363973650746,0) -- (3.30264099405432,0);
\draw[line width=4pt,color=zzttff] (3.30264099405432,0) -- (3.3212455907435654,0);
\draw[line width=4pt,color=zzttff] (3.3212455907435654,0) -- (3.339850187432811,0);
\draw[line width=4pt,color=zzttff] (3.339850187432811,0) -- (3.3584547841220562,0);
\draw[line width=4pt,color=zzttff] (3.3584547841220562,0) -- (3.3770593808113016,0);
\draw[line width=4pt,color=zzttff] (3.3770593808113016,0) -- (3.395663977500547,0);
\draw[line width=4pt,color=zzttff] (3.395663977500547,0) -- (3.4142685741897925,0);
\draw[line width=4pt,color=zzttff] (3.4142685741897925,0) -- (3.432873170879038,0);
\draw[line width=4pt,color=zzttff] (3.432873170879038,0) -- (3.4514777675682833,0);
\draw[line width=4pt,color=zzttff] (3.4514777675682833,0) -- (3.4700823642575287,0);
\draw[line width=4pt,color=zzttff] (3.4700823642575287,0) -- (3.488686960946774,0);
\draw[line width=4pt,color=zzttff] (3.488686960946774,0) -- (3.5072915576360195,0);
\draw[line width=4pt,color=zzttff] (3.5072915576360195,0) -- (3.525896154325265,0);
\draw[line width=4pt,color=zzttff] (3.525896154325265,0) -- (3.5445007510145103,0);
\draw[line width=4pt,color=zzttff] (3.5445007510145103,0) -- (3.5631053477037558,0);
\draw[line width=4pt,color=zzttff] (3.5631053477037558,0) -- (3.581709944393001,0);
\draw[line width=4pt,color=zzttff] (3.581709944393001,0) -- (3.6003145410822466,0);
\draw[line width=4pt,color=zzttff] (3.6003145410822466,0) -- (3.618919137771492,0);
\draw[line width=4pt,color=zzttff] (3.618919137771492,0) -- (3.6375237344607374,0);
\draw[line width=4pt,color=zzttff] (3.6375237344607374,0) -- (3.656128331149983,0);
\draw[line width=4pt,color=zzttff] (3.656128331149983,0) -- (3.6747329278392282,0);
\draw[line width=4pt,color=zzttff] (3.6747329278392282,0) -- (3.6933375245284736,0);
\draw[line width=4pt,color=zzttff] (3.6933375245284736,0) -- (3.711942121217719,0);
\draw[line width=4pt,color=zzttff] (3.711942121217719,0) -- (3.7305467179069645,0);
\draw[line width=4pt,color=zzttff] (3.7305467179069645,0) -- (3.74915131459621,0);
\draw[line width=4pt,color=zzttff] (3.74915131459621,0) -- (3.7677559112854553,0);
\draw[line width=4pt,color=zzttff] (3.7677559112854553,0) -- (3.7863605079747007,0);
\draw[line width=4pt,color=zzttff] (3.7863605079747007,0) -- (3.804965104663946,0);
\draw[line width=4pt,color=zzttff] (3.804965104663946,0) -- (3.8235697013531915,0);
\draw[line width=4pt,color=zzttff] (3.8235697013531915,0) -- (3.842174298042437,0);
\draw[line width=4pt,color=zzttff] (3.842174298042437,0) -- (3.8607788947316823,0);
\draw[line width=4pt,color=zzttff] (3.8607788947316823,0) -- (3.8793834914209278,0);
\draw[line width=4pt,color=zzttff] (3.8793834914209278,0) -- (3.897988088110173,0);
\draw[line width=4pt,color=zzttff] (3.897988088110173,0) -- (3.9165926847994186,0);
\draw[line width=4pt,color=zzttff] (3.9165926847994186,0) -- (3.935197281488664,0);
\draw[line width=4pt,color=zzttff] (3.935197281488664,0) -- (3.9538018781779094,0);
\draw[line width=4pt,color=zzttff] (3.9538018781779094,0) -- (3.972406474867155,0);
\draw[line width=4pt,color=zzttff] (3.972406474867155,0) -- (3.9910110715564002,0);
\draw[line width=4pt,color=zzttff] (3.9910110715564002,0) -- (4.009615668245646,0);
\draw[line width=4pt,color=zzttff] (4.009615668245646,0) -- (4.028220264934891,0);
\draw[line width=4pt,color=zzttff] (4.028220264934891,0) -- (4.0468248616241365,0);
\draw[line width=4pt,color=zzttff] (4.0468248616241365,0) -- (4.065429458313382,0);
\draw[line width=4pt,color=zzttff] (4.065429458313382,0) -- (4.084034055002627,0);
\draw[line width=4pt,color=zzttff] (4.084034055002627,0) -- (4.102638651691873,0);
\draw[line width=4pt,color=zzttff] (4.102638651691873,0) -- (4.121243248381118,0);
\draw[line width=4pt,color=zzttff] (4.121243248381118,0) -- (4.1398478450703635,0);
\draw[line width=4pt,color=zzttff] (4.1398478450703635,0) -- (4.158452441759609,0);
\draw[line width=4pt,color=zzttff] (4.158452441759609,0) -- (4.177057038448854,0);
\draw[line width=4pt,color=zzttff] (4.177057038448854,0) -- (4.1956616351381,0);
\draw[line width=4pt,color=zzttff] (4.1956616351381,0) -- (4.214266231827345,0);
\draw[line width=4pt,color=zzttff] (4.214266231827345,0) -- (4.232870828516591,0);
\draw[line width=4pt,color=zzttff] (4.232870828516591,0) -- (4.251475425205836,0);
\draw[line width=4pt,color=zzttff] (4.251475425205836,0) -- (4.270080021895081,0);
\draw[line width=4pt,color=zzttff] (4.270080021895081,0) -- (4.288684618584327,0);
\draw[line width=4pt,color=zzttff] (4.288684618584327,0) -- (4.307289215273572,0);
\draw[line width=4pt,color=zzttff] (4.307289215273572,0) -- (4.325893811962818,0);
\draw[line width=4pt,color=zzttff] (4.325893811962818,0) -- (4.344498408652063,0);
\draw[line width=4pt,color=zzttff] (4.344498408652063,0) -- (4.3631030053413085,0);
\draw[line width=4pt,color=zzttff] (4.3631030053413085,0) -- (4.381707602030554,0);
\draw[line width=4pt,color=zzttff] (4.381707602030554,0) -- (4.400312198719799,0);
\draw[line width=4pt,color=zzttff] (4.400312198719799,0) -- (4.418916795409045,0);
\draw[line width=4pt,color=zzttff] (4.418916795409045,0) -- (4.43752139209829,0);
\draw[line width=4pt,color=zzttff] (4.43752139209829,0) -- (4.4561259887875355,0);
\draw[line width=4pt,color=zzttff] (4.4561259887875355,0) -- (4.474730585476781,0);
\draw[line width=4pt,color=zzttff] (4.474730585476781,0) -- (4.493335182166026,0);
\draw[line width=4pt,color=zzttff] (4.493335182166026,0) -- (4.511939778855272,0);
\draw[line width=4pt,color=zzttff] (4.511939778855272,0) -- (4.530544375544517,0);
\draw[line width=4pt,color=zzttff] (4.530544375544517,0) -- (4.549148972233763,0);
\draw[line width=4pt,color=zzttff] (4.549148972233763,0) -- (4.567753568923008,0);
\draw[line width=4pt,color=zzttff] (4.567753568923008,0) -- (4.586358165612253,0);
\draw[line width=4pt,color=zzttff] (4.586358165612253,0) -- (4.604962762301499,0);
\draw[line width=4pt,color=zzttff] (4.604962762301499,0) -- (4.623567358990744,0);
\draw[line width=4pt,color=zzttff] (4.623567358990744,0) -- (4.64217195567999,0);
\draw[line width=4pt,color=zzttff] (4.64217195567999,0) -- (4.660776552369235,0);
\draw[line width=4pt,color=zzttff] (4.660776552369235,0) -- (4.6793811490584805,0);
\draw[line width=4pt,color=zzttff] (4.6793811490584805,0) -- (4.697985745747726,0);
\draw[line width=4pt,color=zzttff] (4.697985745747726,0) -- (4.716590342436971,0);
\draw[line width=4pt,color=zzttff] (4.716590342436971,0) -- (4.735194939126217,0);
\draw[line width=4pt,color=zzttff] (4.735194939126217,0) -- (4.753799535815462,0);
\draw[line width=4pt,color=zzttff] (4.753799535815462,0) -- (4.7724041325047075,0);
\draw[line width=4pt,color=zzttff] (4.7724041325047075,0) -- (4.791008729193953,0);
\draw[line width=4pt,color=zzttff] (4.791008729193953,0) -- (4.809613325883198,0);
\draw[line width=4pt,color=zzttff] (4.809613325883198,0) -- (4.828217922572444,0);
\draw[line width=4pt,color=zzttff] (4.828217922572444,0) -- (4.846822519261689,0);
\draw[line width=4pt,color=zzttff] (4.846822519261689,0) -- (4.865427115950935,0);
\draw[line width=4pt,color=zzttff] (4.865427115950935,0) -- (4.88403171264018,0);
\draw[line width=4pt,color=zzttff] (4.88403171264018,0) -- (4.902636309329425,0);
\draw[line width=4pt,color=zzttff] (4.902636309329425,0) -- (4.921240906018671,0);
\draw[line width=4pt,color=zzttff] (4.921240906018671,0) -- (4.939845502707916,0);
\draw[line width=4pt,color=zzttff] (4.939845502707916,0) -- (4.958450099397162,0);
\draw[line width=4pt,color=zzttff] (4.958450099397162,0) -- (4.977054696086407,0);
\draw[line width=4pt,color=zzttff] (4.977054696086407,0) -- (4.9956592927756525,0);
\draw[line width=4pt,color=zzttff] (4.9956592927756525,0) -- (5.014263889464898,0);
\draw[line width=4pt,color=zzttff] (5.014263889464898,0) -- (5.032868486154143,0);
\draw[line width=4pt,color=zzttff] (5.032868486154143,0) -- (5.051473082843389,0);
\draw[line width=4pt,color=zzttff] (5.051473082843389,0) -- (5.070077679532634,0);
\draw[line width=4pt,color=zzttff] (5.070077679532634,0) -- (5.0886822762218795,0);
\draw[line width=4pt,color=zzttff] (5.0886822762218795,0) -- (5.107286872911125,0);
\draw[line width=4pt,color=zzttff] (5.107286872911125,0) -- (5.12589146960037,0);
\draw[line width=4pt,color=zzttff] (5.12589146960037,0) -- (5.144496066289616,0);
\draw[line width=4pt,color=zzttff] (5.144496066289616,0) -- (5.163100662978861,0);
\draw[line width=4pt,color=zzttff] (5.163100662978861,0) -- (5.181705259668107,0);
\draw[line width=4pt,color=zzttff] (5.181705259668107,0) -- (5.200309856357352,0);
\draw[line width=4pt,color=zzttff] (5.200309856357352,0) -- (5.218914453046597,0);
\draw[line width=4pt,color=zzttff] (5.218914453046597,0) -- (5.237519049735843,0);
\draw[line width=4pt,color=zzttff] (5.237519049735843,0) -- (5.256123646425088,0);
\draw[line width=4pt,color=zzttff] (5.256123646425088,0) -- (5.274728243114334,0);
\draw[line width=4pt,color=zzttff] (5.274728243114334,0) -- (5.293332839803579,0);
\draw[line width=4pt,color=zzttff] (5.293332839803579,0) -- (5.3119374364928245,0);
\draw[line width=4pt,color=zzttff] (5.3119374364928245,0) -- (5.33054203318207,0);
\draw[line width=4pt,color=zzttff] (5.33054203318207,0) -- (5.349146629871315,0);
\draw[line width=4pt,color=zzttff] (5.349146629871315,0) -- (5.367751226560561,0);
\draw[line width=4pt,color=zzttff] (5.367751226560561,0) -- (5.386355823249806,0);
\draw[line width=4pt,color=zzttff] (5.386355823249806,0) -- (5.4049604199390515,0);
\draw[line width=4pt,color=zzttff] (5.4049604199390515,0) -- (5.423565016628297,0);
\draw[line width=4pt,color=zzttff] (5.423565016628297,0) -- (5.442169613317542,0);
\draw[line width=4pt,color=zzttff] (5.442169613317542,0) -- (5.460774210006788,0);
\draw[line width=4pt,color=zzttff] (5.460774210006788,0) -- (5.479378806696033,0);
\draw[line width=4pt,color=zzttff] (5.479378806696033,0) -- (5.497983403385279,0);
\draw[line width=4pt,color=zzttff] (5.497983403385279,0) -- (5.516588000074524,0);
\draw[line width=4pt,color=zzttff] (5.516588000074524,0) -- (5.535192596763769,0);
\draw[line width=4pt,color=zzttff] (5.535192596763769,0) -- (5.553797193453015,0);
\draw[line width=4pt,color=zzttff] (5.553797193453015,0) -- (5.57240179014226,0);
\draw[line width=4pt,color=zzttff] (5.57240179014226,0) -- (5.591006386831506,0);
\draw[line width=4pt,color=zzttff] (5.591006386831506,0) -- (5.609610983520751,0);
\draw[line width=4pt,color=zzttff] (5.609610983520751,0) -- (5.6282155802099965,0);
\draw[line width=4pt,color=zzttff] (5.6282155802099965,0) -- (5.646820176899242,0);
\draw[line width=4pt,color=zzttff] (5.646820176899242,0) -- (5.665424773588487,0);
\draw[line width=4pt,color=zzttff] (5.665424773588487,0) -- (5.684029370277733,0);
\draw[line width=4pt,color=zzttff] (5.684029370277733,0) -- (5.702633966966978,0);
\draw[line width=4pt,color=zzttff] (5.702633966966978,0) -- (5.7212385636562235,0);
\draw[line width=4pt,color=zzttff] (5.7212385636562235,0) -- (5.739843160345469,0);
\draw[line width=4pt,color=zzttff] (5.739843160345469,0) -- (5.758447757034714,0);
\draw[line width=4pt,color=zzttff] (5.758447757034714,0) -- (5.77705235372396,0);
\draw[line width=4pt,color=zzttff] (5.77705235372396,0) -- (5.795656950413205,0);
\draw[line width=4pt,color=zzttff] (5.795656950413205,0) -- (5.814261547102451,0);
\draw[line width=4pt,color=zzttff] (5.814261547102451,0) -- (5.832866143791696,0);
\draw[line width=4pt,color=zzttff] (5.832866143791696,0) -- (5.851470740480941,0);
\draw[line width=4pt,color=zzttff] (5.851470740480941,0) -- (5.870075337170187,0);
\draw[line width=4pt,color=zzttff] (5.870075337170187,0) -- (5.888679933859432,0);
\draw[line width=4pt,color=zzttff] (5.888679933859432,0) -- (5.907284530548678,0);
\draw[line width=4pt,color=zzttff] (5.907284530548678,0) -- (5.925889127237923,0);
\draw[line width=4pt,color=zzttff] (5.925889127237923,0) -- (5.9444937239271685,0);
\draw[line width=4pt,color=zzttff] (5.9444937239271685,0) -- (5.963098320616414,0);
\draw[line width=4pt,color=zzttff] (5.963098320616414,0) -- (5.981702917305659,0);
\draw[line width=4pt,color=zzttff] (5.981702917305659,0) -- (6.000307513994905,0);
\draw[line width=4pt,color=zzttff] (6.000307513994905,0) -- (6.01891211068415,0);
\draw[line width=4pt,color=zzttff] (6.01891211068415,0) -- (6.0375167073733955,0);
\draw[line width=4pt,color=zzttff] (6.0375167073733955,0) -- (6.056121304062641,0);
\draw[line width=4pt,color=zzttff] (6.056121304062641,0) -- (6.074725900751886,0);
\draw[line width=4pt,color=zzttff] (6.074725900751886,0) -- (6.093330497441132,0);
\draw[line width=4pt,color=zzttff] (6.093330497441132,0) -- (6.111935094130377,0);
\draw[line width=4pt,color=zzttff] (6.111935094130377,0) -- (6.130539690819623,0);
\draw[line width=4pt,color=zzttff] (6.130539690819623,0) -- (6.149144287508868,0);
\draw[line width=4pt,color=zzttff] (6.149144287508868,0) -- (6.167748884198113,0);
\draw[line width=4pt,color=zzttff] (6.167748884198113,0) -- (6.186353480887359,0);
\draw[line width=4pt,color=zzttff] (6.186353480887359,0) -- (6.204958077576604,0);
\draw[line width=4pt,color=zzttff] (6.204958077576604,0) -- (6.22356267426585,0);
\draw[line width=4pt,color=zzttff] (6.22356267426585,0) -- (6.242167270955095,0);
\draw[line width=4pt,color=zzttff] (6.242167270955095,0) -- (6.2607718676443405,0);
\draw[line width=4pt,color=zzttff] (6.2607718676443405,0) -- (6.279376464333586,0);
\draw[line width=4pt,color=zzttff] (6.279376464333586,0) -- (6.297981061022831,0);
\draw[line width=4pt,color=zzttff] (6.297981061022831,0) -- (6.316585657712077,0);
\draw[line width=4pt,color=zzttff] (6.316585657712077,0) -- (6.335190254401322,0);
\draw[line width=4pt,color=zzttff] (6.335190254401322,0) -- (6.3537948510905675,0);
\draw[line width=4pt,color=zzttff] (6.3537948510905675,0) -- (6.372399447779813,0);
\draw[line width=4pt,color=zzttff] (6.372399447779813,0) -- (6.391004044469058,0);
\draw[line width=4pt,color=zzttff] (6.391004044469058,0) -- (6.409608641158304,0);
\draw[line width=4pt,color=zzttff] (6.409608641158304,0) -- (6.428213237847549,0);
\draw[line width=4pt,color=zzttff] (6.428213237847549,0) -- (6.446817834536795,0);
\draw[line width=4pt,color=zzttff] (6.446817834536795,0) -- (6.46542243122604,0);
\draw[line width=4pt,color=zzttff] (6.46542243122604,0) -- (6.484027027915285,0);
\draw[line width=4pt,color=zzttff] (6.484027027915285,0) -- (6.502631624604531,0);
\draw[line width=4pt,color=zzttff] (6.502631624604531,0) -- (6.521236221293776,0);
\draw[line width=4pt,color=zzttff] (6.521236221293776,0) -- (6.539840817983022,0);
\draw[line width=4pt,color=zzttff] (6.539840817983022,0) -- (6.558445414672267,0);
\draw[line width=4pt,color=zzttff] (6.558445414672267,0) -- (6.5770500113615125,0);
\draw[line width=4pt,color=zzttff] (6.5770500113615125,0) -- (6.595654608050758,0);
\draw[line width=4pt,color=zzttff] (6.595654608050758,0) -- (6.614259204740003,0);
\draw[line width=4pt,color=zzttff] (6.614259204740003,0) -- (6.632863801429249,0);
\draw[line width=4pt,color=zzttff] (6.632863801429249,0) -- (6.651468398118494,0);
\draw[line width=4pt,color=zzttff] (6.651468398118494,0) -- (6.6700729948077395,0);
\draw[line width=4pt,color=zzttff] (6.6700729948077395,0) -- (6.688677591496985,0);
\draw (2.847981552648539,0.11752165047541878) node[anchor=north west] {$f_c$};
\end{axis}
\end{tikzpicture}
    \caption{Gain d'un filtre passe-bas idéal laissant parfaitement passer toutes les fréquences inférieures à la fréquence de coupure et rejetant les autres avec un gain nul au-dessus de $f_c$.}
    \label{fig:passe-bas}
\end{figure}
%%FIN RÉVISION JÉRÉMIE%%

%merge github jeremie
% ctrl+f \begin center
\subsection{Exemple --- Circuit RC en série}
\label{exemple-RC}

Soit le circuit RC illustré à la figure~\ref{circuitRC-serie}.

\begin{figure}[h]
\centering
\begin{circuitikz} \draw
(0,0) to[V=$V_S$~(\textsc{ac})] (0,3) to[C=$C$] (3,3) to[R=$R$] (3,0) to[short] (0,0)
;\end{circuitikz}
\caption{\label{circuitRC-serie}Circuit RC en série.}
\end{figure}

En considérant ce circuit comme un diviseur de tension, les tensions aux bornes du condensateur et de la résistance sont obtenues en multipliant la tension fournie par la source par la \textit{fonction de transfert} de chaque composant.
\begin{gather}
V_C=H_C\,V_S=\frac{Z_C}{Z_C+Z_R}\,V_S\\
V_R=H_R\,V_S=\frac{Z_R}{Z_C+Z_R}\,V_S
\end{gather}

Or, puisque l'impédance du condensateur dépend de la fréquence de la source, il en résulte que les deux fonctions de transfert, donc les deux tensions, varieront aussi en fonction de la fréquence. Les fonctions de transfert en régime permanent peuvent s'écrire:
\begin{gather}
H_C\!\left(\omega\right)=\frac{1}{1+j\,\omega\,R\,C}\\
H_R\!\left(\omega\right)=\frac{j\,\omega\,R\,C}{1+j\,\omega\,R\,C}.
\end{gather}

Selon l'équation~\eqref{eq:GainPhase}, les diagrammes de Bode de ce circuit RC sont les graphes des gains et des phases obtenus à l'aide de la tension de sortie aux bornes du condensateur ou de la résistance, soit les fonctions
\begin{subequations}
\begin{gather}
G_C=\frac{1}{\sqrt{1+\left(\omega\,R\,C\right)^2}}\,,\\
\varphi_C=\mathrm{arctan}\!\left(-\omega\,R\,C\right)\,,
\end{gather}
\end{subequations}
\begin{subequations}
\begin{gather}
G_R=\frac{\omega\,R\,C}{\sqrt{1+\left(\omega\,R\,C\right)^2}}\,,\\
\varphi_R=\mathrm{arctan}\!\left(\frac{1}{\omega\,R\,C}\right)\,.
\end{gather}
\end{subequations}
En analysant les deux fonctions des gains lorsque $\omega\,\rightarrow\,0$, on remarque que la limite $G_C\,\rightarrow\,1$ et $G_R\,\rightarrow\,0$. L'inverse survient lorsque $\omega\,\rightarrow\,\infty$. Ainsi, si la sortie du diviseur de tension est prise aux bornes du condensateur, les hautes fréquences seront atténuées, mais les basses fréquences ne le seront pas : le circuit agit alors comme un filtre passe-bas. À l'opposé, si la sortie est prise aux bornes de la résistance, le diviseur de tension se comportera comme un filtre passe-haut.

La fréquence de coupure (à -3 dB) est celle où les gains valent $2^{\,\text{-\textonehalf}}$:
\begin{equation}
G_C=G_R=\frac{1}{\sqrt{2}}.
\end{equation}
La fréquence de coupure est alors:
\begin{subequations}
\begin{gather}
\omega_c=\frac{1}{R\,C}\\
f_c=\frac{1}{2\,\pi\,R\,C}.
\end{gather}
\end{subequations}

Tous les filtres passifs peuvent être analysés de la même façon, à partir des impédances.


\subsection{Exemples de filtres passifs}

Il a été montré dans l'exemple précédent qu'un circuit RC peut être utilisé comme filtre passe-bas ou passe-haut, tout dépendamment aux bornes de quel composant la sortie est prise. Le filtrage provient directement de la réponse en fréquence du condensateur ; il ne laisse pas passer un courant continu mais va permettre le passage à un courant alternatif. C'est le même principe, mais à l'opposé, pour un circuit RL. Une bobine va laisser passer facilement un courant continu, mais elle s'opposera au passage d'un courant alternatif. Dans le cas d'un filtre RL, la fréquence de coupure est donnée par:
\begin{equation}
f_c=\frac{R}{2\,\pi\,L}.
\end{equation}

En plaçant un filtre passe-haut et un filtre passe-bas en série, cela forme un filtre passe-bande. En combinant un filtre passe-haut et passe-bas en parallèle, on obtient un filtre coupe-bande. Encore une fois, le filtrage découle directement de la réponse en fréquence des composants (C ou L).

Toutefois, il existe un autre moyen de faire un filtre passe-bande ou coupe-bande : en utilisant un oscillateur. Un circuit contenant à la fois un condensateur (ou plus) et une bobine d'inductance (ou plus) est un oscillateur. Un circuit de la sorte est appelé circuit RLC, pour résistance, inductance et capacité. Le condensateur veut favoriser les hautes fréquences alors que la bobine souhaite l'inverse, ce qui donne un comportement oscillant, similaire à celui d'un système masse-ressort. Le rôle de la résistance est de limiter ces oscillations --- elle est comparable au frottement dans le cas d'un oscillateur mécanique.

Comme tous les oscillateurs, le circuit RLC a une (ou plusieurs) fréquence naturelle. C'est près de cette fréquence naturelle que le système va entrer en résonance ; c'est donc là que la tension à la sortie sera la plus grande. À l'inverse, loin de la fréquence naturelle, le signal va être considérablement atténué.

Pour les exemples de filtres suivants, la tension de sortie est toujours prise aux bornes de la résistance. Il est à noter que les tensions aux bornes des autres composants seront complémentaires.


\subsubsection{Passe-bande}

Les circuits RLC peuvent former des filtres passe-bande. Dans ce cas, il existe deux fréquences de coupure distinctes et seules les fréquences situées entre elles ne seront pas significativement atténuées. Lorsqu'un filtre passe-bande (ou coupe-bande) provient d'un circuit RLC (i.e. un oscillateur), les deux fréquences de coupure sont situées de part et d'autre d'une fréquence centrale, $f_0$, qui correspond à la fréquence naturelle du circuit. La fréquence naturelle équivaut à la moyenne \textit{géométrique} des deux fréquences de coupure et non à leur moyenne \textit{algébrique}, c'est-à-dire:
\begin{equation}
f_0=\sqrt{f_{c1}\,f_{c2}}.
\end{equation}

Les figures~\ref{RLC-passe-bande1} et \ref{RLC-passe-bande2} montrent deux configurations d'un filtre passe-bande. Dans les deux cas, la fréquence naturelle de la bande passante est:
\begin{equation}
\label{eq-passe-bande1}
f_0=\frac{1}{2\,\pi\,\sqrt{L\,C}}.
\end{equation}
La largeur de la bande passante est aussi la même dans les deux cas:
\begin{equation}
\label{eq-passe-bande2}
B=\Delta f=\frac{R}{2\,\pi\,L}.
\end{equation}

\begin{figure}[h]
\begin{center}
\begin{circuitikz} \draw
(0,0) to[short,o-o] (6,0)
(0,2) to[short,o-] (1,2) to[L=$L$] (3,2) to[C=$C$] (5,2) to[short,-o] (6,2)
(5,2) to[R=$R$] (5,0) node[ground]{}
{[anchor=east] (0,2) node{$V_{\mathrm{in}}$~+} (0,0) node{--}}
{[anchor=west] (6,2) node{+~$V_{\mathrm{out}}$} (6,0) node{--}}
;\end{circuitikz}
\end{center}
\caption{\label{RLC-passe-bande1}Circuit RLC utilisé comme filtre passe-bande.}
\end{figure}

\begin{figure}[h]
\begin{center}
\begin{circuitikz} \draw
(0,0) to[short,o-o] (6,0)
(0,2) to[short,o-o] (6,2)
(1,2) to[L=$L$] (1,0) node[ground]{}
(3,2) to[C=$C$] (3,0)
(5,2) to[R=$R$] (5,0)
{[anchor=east] (0,2) node{$V_{\mathrm{in}}$~+} (0,0) node{--}}
{[anchor=west] (6,2) node{+~$V_{\mathrm{out}}$} (6,0) node{--}}
;\end{circuitikz}
\end{center}
\caption{\label{RLC-passe-bande2}Autre circuit RLC utilisé comme filtre passe-bande.}
\end{figure}


\subsubsection{Coupe-bande}

Il est aussi possible, avec un circuit RLC, d'obtenir un filtre coupe-bande. Les figures~\ref{RLC-coupe-bande1} et \ref{RLC-coupe-bande2} en illustrent deux exemples. Pour ces deux circuits, la fréquence naturelle et la largeur de la bande passante sont les mêmes que pour les deux filtres passe-bande montrés dans la section précédente (équations~\ref{eq-passe-bande1} et \ref{eq-passe-bande2}). La seule différence réside dans le fait que la fréquence centrale $f_0$ correspond cette fois à une \textit{anti-résonance}, puisque c'est à cette fréquence que le signal sera le plus atténué.

\begin{figure}[h]
\begin{center}
\begin{circuitikz} \draw
(0,0) to[short,o-o] (4,0)
(0,3) to[short,o-o] (4,3)
(1,3) to[L=$L$] (1,1.2) to[C=$C$] (1,0) node[ground]{}
(3,3) to[R=$R$] (3,0)
{[anchor=east] (0,3) node{$V_{\mathrm{in}}$~+} (0,0) node{--}}
{[anchor=west] (4,3) node{+~$V_{\mathrm{out}}$} (4,0) node{--}}
;\end{circuitikz}
\end{center}
\caption{\label{RLC-coupe-bande1}Circuit RLC utilisé comme filtre coupe-bande.}
\end{figure}

\begin{figure}[h]
\begin{center}
\begin{circuitikz} \draw
(0,0) to[short,o-o] (5,0)
(0,2) to[short,o-] (1,2) to[short] (1,2.5) to[L=$L$] (3,2.5) to[short] (3,1.5) to[C=$C$] (1,1.5) to[short] (1,2)
(3,2) to[short,-o] (5,2)
(4,2) to[R=$R$] (4,0) node[ground]{}
{[anchor=east] (0,2) node{$V_{\mathrm{in}}$~+} (0,0) node{--}}
{[anchor=west] (5,2) node{+~$V_{\mathrm{out}}$} (5,0) node{--}}
;\end{circuitikz}
\end{center}
\caption{\label{RLC-coupe-bande2}Autre circuit RLC utilisé comme filtre coupe-bande.}
\end{figure}


\end{document}

Écrit par Jean-Raphaël Carrier
Dernière modification : 12 janvier 2014