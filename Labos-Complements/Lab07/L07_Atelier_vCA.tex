%Créé par Claudine Allen en collaboration avec Jean-Raphaël Carrier
%Dernière modification JRC: 13 janvier 2014
%Élimination du labo de résistivité des matériaux (IV) à la fin de l'ère JRC + regroupement des labos VI & VII en début de pandémie COVID-19 => renumérotation IX -> VII maintenant
%Dernière modification CA: 14 novembre 2023
%Dernière modification JG:
%***ToDo***
% - préparer un signal de bruit pour montrer comment on se rapproche de sinus/cosinus avec Q qui augmente dans un filtre passe-bande,... voir version JG.
% -  recopier préparation & déroulement de l'atelier en ordre inverse en ligne dans le cours 6.

\RequirePackage[l2tabu, orthodox]{nag} %Check for obsolete commands
\documentclass[canadien,12pt,oneside,letterpaper]{article}
%
%-----------------------------------------------------
%Loading packages
%
\usepackage[utf8]{inputenc}
\usepackage[T1]{fontenc}
\usepackage[canadien]{babel}
\usepackage{lmodern}
\usepackage{textcomp}
\usepackage{amsmath,amssymb}
\usepackage{siunitx}
\usepackage{xcolor}
\usepackage[colorlinks=true]{hyperref}
\usepackage[all]{hypcap}
\usepackage{graphicx}
\usepackage[oldvoltagedirection,americanvoltages,americancurrents,siunitx]{circuitikz}
\usetikzlibrary{babel}
\usepackage{caption}
\usepackage{subcaption}
%\usepackage{subfig}
\usepackage[letterpaper,headheight=15pt]{geometry}
\usepackage{fancyhdr}
\usepackage{setspace}
%
%----------------------------------------------------
%Other configurations and layout
%
\sisetup{separate-uncertainty}
\captionsetup{font=small,labelfont=bf,margin=0.1\textwidth}
\pagestyle{fancy}
\fancyhf{}
\lhead{\textsl{GPH-2006/PHY-2002~---~Laboratoire~VII}}
\rhead{\textsl{Page \thepage}}
\setcounter{secnumdepth}{0}
\setlength{\parskip}{1.5ex plus0.5ex minus0.2ex}
%\onehalfspacing
\interfootnotelinepenalty=10000 %To avoid footnotes spreading on several pages.
%
%---------------------------------------------------
%
\title{\textbf{Atelier VII}\\Filtres passifs \& actifs\thanks{Auteurs: Claudine Allen, Jean-Raphaël Carrier \& Jérémie Guilbert}}
\renewcommand\footnotemark{}
\date{}


\begin{document}

\maketitle \vspace{-2cm}

\noindent\textit{\textbf{Prélude de contexte et déroulement des ateliers}}

\textit{Les ateliers se rapprochent d'un mode de travail expérimental plus autonome, s'éloignant du contexte pédagogique pour se rapprocher du milieu professionnel. En effet, après avoir étudié tous les éléments de la section \nameref{sec:lectures} et selon les consignes en \nameref{sec:livrables}, vous devez concevoir votre propre protocole en \nameref{sec:prep} afin d'atteindre les sous-objectifs sélectionnés de la section \nameref{sec:atelier}. Ces sous-objectifs sont généralement accompagnés d'illustrations de circuits, mais il y a toutefois progression vers une conception de plus en plus libre où vous devrez proposer vos propres designs de circuits.}

\textit{Les équipes pour ces ateliers et le projet de conception passent de 2 à 4 personnes et les évaluations se font simplement au laboratoire sur la réussite ou l'échec des sous-objectifs indiqués comme livrables. Pour consigner votre préparation et vos expériences dans votre cahier de recherche, choisissez-en un à partager avec vos nouveaux.elles coéquipiers.ères. Il n'y a aucune autre remise à faire pour les ateliers afin de vous donner le temps de travailler sur votre projet de conception final.}

\vspace{-2.5ex}
\section{Objectifs principaux}
\vspace{-1.5ex}
Une fonction essentielle accomplie par des circuits électriques dans le travail expérimental est celle de filtrer un signal. On peut alors retirer des fréquences indésirables et généralement obtenir une mesure plus fidèle avec un meilleur rapport signal sur bruit, par exemple. L’effet de différents filtres sur le gain et la phase de signaux en tension sera donc étudié dans ce laboratoire à l’aide du diagramme de Bode. En particulier, les performances de différents design de filtres avec des composants uniquement passifs seront comparées à un design intégrant un amplificateur opérationnel actif. Ce laboratoire, en association avec le deuxième devoir, a aussi pour objectif d’approfondir la compréhension de l’étudiant.e des domaines temporel et fréquentiel dans le cadre des transformations de Laplace et Fourier. La relation entre délai temporel et déphasage sera donc examinée pour le filtre RC.

%Les travaux effectués aborderont les objectifs d’ensemble 1, 2, 4, 5, 7, 9, 10, 11 et 12 du plan de cours + BAIC?
\vspace{-2ex}
\section{Lectures préparatoires}\label{sec:lectures}

\begin{itemize}
\item cours \textit{Analyse de circuits résistifs et réactifs}, votre compréhension des expériences sera largement supérieure en ayant étudié ce cours avant le laboratoire!
\item complément \textit{Analyse temporelle}
\item complément \textit{Analyse fréquentielle}
\end{itemize}


% \section{Matériel}

% La réalisation de ce laboratoire requiert l'utilisation de:
% \begin{itemize}
% \item un oscilloscope avec générateur de fonctions;
% \item un bloc d'alimentation;
% \item deux condensateurs de 0,47~$\mu$F et quatre condensateurs de 1~$\mu$F, dont un ayant une précision de $\pm5$~\%;
% \item deux bobines d'inductance de 1~mH;
% \item plusieurs résistances : 270~$\Omega$ (trois fois), 560~$\Omega$, 1~k$\Omega$ (trois fois), 10~k$\Omega$ (trois fois) et 100~k$\Omega$;
% \item un amplificateur opérationnel;
% \item une plaquette de montage.
% \end{itemize}

%\setlength{\parskip}{1ex plus 0.5ex minus 0.2ex}
\vspace{-2ex}
\section{Préparation}\label{sec:prep}
\vspace{-3ex}
AVANT la séance d'atelier, écrivez sommairement dans votre cahier de recherche tout ce que vous prévoyez faire au laboratoire pour accomplir les livrables de la prochaine section sans oublier de calculer toute valeur de référence donnée en complément aux fins de comparaison. En plus des manipulations expérimentales, des mesures à prendre et de leur analyse, n'oubliez pas de considérer des étapes de modélisation, conception et simulation\footnote{N'hésitez pas à explorer les possibilités du menu \texttt{Dessiner~>~Sorties et étiquettes} de l'application Web de Paul Falstad, entre autres pour exporter des données de simulation.}. Plusieurs logiciels spécialisés pour l'électronique listés sur le site de cours, dans la section \texttt{Logiciels} de l'onglet \texttt{Matériel didactique}, vous aideront dans cette préparation. En particulier, l'onglet \texttt{Circuits} de \href{https://www.falstad.com/circuit/}{l'app par P. Falstad} contient tous les blocs en atelier, montrant directement les "réponses" des comportements idéaux attendus. Comme la programmation LabVIEW s'avère souvent chronophage, vous pouvez être soulagés qu'aucun VI n'est demandé pour les ateliers. Toutefois pour mieux caractériser les filtres avec des diagrammes de Bode lorsqu'ils sont mentionnés en sous-objectifs ci-dessous, téléchargez et utilisez le VI LabVIEW déjà prêt dans le module de cet atelier.

\vspace{-2ex}
\section{Livrables}\label{sec:livrables}
\vspace{-3ex}
Le minimum de simulations, expériences et analyses à être accomplies par équipe de 4 personnes doivent satisfaire les sous-objectifs des figures~\ref{sch-RC2} et~\ref{sch-RC-ordre3} ainsi qu'un des deux autres sous-objectifs à votre choix. Divisez les tâches afin que personne ne se tourne les pouces en atelier et utilisez un maximum de plaquettes de montage sans rien démonter : un membre de l'équipe d'enseignement doit valider l'accomplissement des livrables avec tous vos montages et votre cahier de recherche juste AVANT votre départ. Finalement, nous vous encourageons fortement à consacrer tout temps restant à l'atelier pour avancer votre projet de conception final afin de mieux respirer en fin de session!

\section{Atelier}\label{sec:atelier}
%\subsection{Partie 1 --- Mesures temporelles}
\begin{figure}[h!]
\centering
\begin{circuitikz} \draw
(0,2) node[left]{$v_{\mathrm{in}}$} to[R=1~k$\Omega$,o-] (3,2) %to[short,-o] (4,2) node[right]{$v_{\mathrm{out}}$}
(3,2) to[C=1~$\mu$F] (3,0) node[ground]{}
;\end{circuitikz}
\caption{\label{sch-RC}\textbf{Sous-objectif}: observer et quantifier le délai temporel du signal en tension aux bornes des différents composants de ce circuit RC série.}
\end{figure}

% a) Sur la plaquette de montage, réalisez le filtre RC de la figure~\ref{sch-RC} en utilisant le condensateur de précision ($\pm5$~\%).
% \begin{figure}[h]
% \centering
% \begin{circuitikz} \draw
% (0,2) node[left]{$v_{\mathrm{g}}$} to[R=1~k$\Omega$,o-] (3,2) to[short,-o] (4,2) node[right]{$v_{\mathrm{out}}$}
% (3,2) to[C=1~$\mu$F] (3,0) node[ground]{}
% ;\end{circuitikz}
% \caption{\label{sch-RC}Circuit RC série.}
% \end{figure}

% b) Avec deux câbles coaxiaux et un «T», reliez la sortie du générateur de fonctions $v_{\mathrm{g}}$ à l'entrée du canal~1 de l'oscilloscope ainsi qu'à la plaquette de montage. À l'aide d'une sonde compensée (10X), reliez les bornes du condensateur au canal~2 de l'oscilloscope.

% c) Générez un signal carré à 50~Hz allant de 0~V à 1~V. Sur l'écran de l'oscilloscope, faites afficher simultanément les deux signaux. En rapetissant les divisions de l'échelle horizontale, observez d'abord qualitativement comment varie la tension $v_{\mathrm{out}}$ et comparez avec le modèle présenté dans le complément \textit{Analyse temporelle}.

% d) Mesurez le temps de montée (de 10~\% à 90~\%) et la constante de temps du circuit. Comparez avec les valeurs attendues.


%\subsection{Partie 2 --- Mesures fréquentielles et diagrammes de Bode}

\begin{figure}[h!]
\centering
\begin{circuitikz} \draw
(0,2) node[left]{$V_{\mathrm{in}}$} to[R=270~$\Omega$,o-] (3,2) to[short] (3,2) to[short,-o] (4,2) node[right]{$V_{\mathrm{out}}$}
(3,2) to[C=1~$\mu$F] (3,0) node[ground]{}
;\end{circuitikz}
\caption{\label{sch-RC2}\textbf{Sous-objectif} : démontrer que ce même circuit RC série agit d'abord en filtre passe-haut, puis en filtre passe-bas en inversant les deux composants. Il faut construire les diagrammes de Bode du gain ET de la phase en fonction de la fréquence pour faire cette démonstration.}
\end{figure}


\begin{figure}[h!]
\centering
\begin{circuitikz} \draw
(0,2) node[left]{$V_{\mathrm{in}}$} to[L,l=$L$,a=1<\milli\henry>,o-] (2.5,2) to[C,l=$C$, a=1<\micro\farad>] (5,2) to[short,-o] (6.5,2) node[right]{$V_{\mathrm{out}}$}
(5.5,2) to[R,l_=$R$,a^=270<\ohm>] (5.5,0) node[ground]{}
;\end{circuitikz}
\caption{\textbf{Sous-objectif} : démontrer, à l'aide des diagrammes de Bode, qu'un circuit RLC série agit en filtre passe-bande dont le facteur de qualité $Q$ varie en changeant les valeurs de $R, L$ et $C$.}
\label{sch-RLC}
\end{figure}

% \textbf{Note 1:} L'étudiant.e est dispensé.e d'évaluer les incertitudes pour le reste de ce laboratoire.

% a) Vous utiliserez maintenant un VI \textit{LabVIEW} pour réaliser automatiquement les acquisitions. Ce VI a déjà été créé pour vous et peut être téléchargé à partir du site du cours. Vous devrez cependant modifier l'adresse où le VI enregistrera les données.

% Le diagramme de Bode permet d'illustrer graphiquement la fonction de transfert en régime permanent d'un système, soit sa réponse en fréquence. Il est composé de deux graphiques basé sur la forme polaire des nombres complexes dans le domaine de Fourier, soit celui du gain en décibels en amplitude, $G\!\left(f\right)=20\,\textrm{log}_{10}\left(\frac{V_{out}}{V_{\mathrm{g}}}\right)$, et celui du déphasage par rapport à la source, $\phi\!\left(f\right)$. Graduez toujours l'axe horizontal (les fréquences) avec une échelle logarithmique.

% Sur la face-avant du VI, faites en sorte que les types des mesures 1, 2 et 3 soient respectivement \texttt{Amplitude}, \texttt{Amplitude} et \texttt{Phase}. Mettez $V_{\mathrm{min}}$ à $-2$~V et $V_{\mathrm{max}}$ à 2~V. La fréquence initiale $f_{\mathrm{ini}}$ doit être 1~Hz. Fixez le nombre d'itérations à 25. Les mesures seront prises une fois par octave, \textit{i.e.} aux fréquences définies par $f_{i}=2^i\,f_{\mathrm{ini}}$, donc de 1~Hz à 16,777~MHz (soit presque la totalité de la plage de l'oscilloscope). Lorsque vous tracerez les diagrammes de Bode, ne prenez pas en compte les points où la valeur est de 9,90E+37 ; ces mesures sont erronées et sont dues au fait que le signal à cette fréquence était trop faible pour une mesure adéquate avec l'oscilloscope.

% \begin{figure}[h]
% \centering
% \begin{circuitikz} \draw
% (0,2) node[left]{$V_{\mathrm{g}}$} to[R=270~$\Omega$,o-] (3,2) to[short,-o] (4,2) node[right]{$V_{\mathrm{out}}$}
% (3,2) to[C=1~$\mu$F] (3,0) node[ground]{}
% ;\end{circuitikz}
% \caption{\label{sch-RC2}Filtre RC série.}
% \end{figure}
% b) Faites cette acquisition pour le filtre RC de la figure~\ref{sch-RC2} ci-dessus. À partir des résultats obtenus, tracez les deux graphiques du diagramme de Bode. À l'aide de ce dernier, déterminez la fréquence de coupure du filtre.

% c) Intervertissez le condensateur et la résistance (afin de prendre la tension aux bornes de la résistance) et refaites l'étape b).

% d) Montez maintenant le circuit inductif de la figure~\ref{sch-RL} et refaites l'étape b).
% \begin{figure}[h]
% \centering
% \begin{circuitikz} \draw
% (0,2) node[left]{$v_{\mathrm{g}}$} to[R=270~$\Omega$,o-] (3,2) to[short,-o] (4,2) node[right]{$v_{\mathrm{out}}$}
% (3,2) to[L=1~mH] (3,0) node[ground]{}
% ;\end{circuitikz}
% \caption{\label{sch-RL}Filtre RL série.}
% \end{figure}

% e) Montez maintenant le circuit RLC représenté à la figure~\ref{sch-RLC}. Reliez les bornes de la résistance au canal~2. Faites l'étape b). Calculez le facteur de qualité $Q$.
% \begin{equation*}
% Q=\frac{f_0}{B}=\frac{f_0}{\Delta f}=\frac{f_0}{f_{c2}-f_{c1}}
% \end{equation*}
% \begin{figure}[h]
% \centering
% \begin{circuitikz} \draw
% (0,2) node[left]{$V_{\mathrm{g}}$} to[L=1~mH,o-] (2.5,2) to[C=1~$\mu$F] (5,2) to[short,-o] (6,2) node[right]{$V_{\mathrm{out}}$}
% (5,2) to[R=270~$\Omega$] (5,0) node[ground]{}
% ;\end{circuitikz}
% \caption{\label{sch-RLC}Filtre RLC série.}
% \end{figure}

% f) Envoyez dans ce circuit un signal carré de 4~V crête à crête, centré à 0~V, à une fréquence de 100~Hz. Ce signal correspond au signal «\texttt{labo\_8\_carre\_100Hz}». Faites afficher la tension de sortie à l'oscilloscope. S'il y a lieu, comparez ce signal avec celui calculé en devoir (filtrage numérique par transformée de Fourier).

% g) Refaites l'étape précédente, mais cette fois avec une fréquence de 1~kHz. S'il y a lieu, comparez vos résultats avec le filtrage du signal «\texttt{labo\_8\_carre\_1kHz}».


%\subsection{Partie 3 --- Coupures de différents filtres passe-bas}

%Pour répartir cette grosse figure sur plusieurs pages au besoin: https://en.wikibooks.org/wiki/LaTeX/Floats,_Figures_and_Captions#Figures_in_multiple_parts
\begin{figure}[h!]
\centering
%\subcaptionbox{}
%{\begin{circuitikz} \draw
%(0,2) node[left]{$V_{\mathrm{in}}$} to[R=270~$\Omega$,o-] (3,2) to[short,-o] (4,2) node[right]{$V_{\mathrm{out}}$}
%(3,2) to[C=1~$\mu$F] (3,0) node[ground]{}
%;\end{circuitikz}}
\subcaptionbox{}
{\begin{circuitikz} \draw
(0,2) node[left]{$V_{\mathrm{in}}$} to[R=135~$\Omega$,o-] (3,2) to[R=270~$\Omega$] (6,2) to[R=560~$\Omega$] (9,2) to[short,-o] (10,2) node[right]{$V_{\mathrm{out}}$}
(3,2) to[C=2~$\mu$F] (3,0) node[ground]{}
(6,2) to[C=1~$\mu$F] (6,0) node[ground]{}
(9,2) to[C=470~nF] (9,0) node[ground]{}
;\end{circuitikz}}
\subcaptionbox{}
{\begin{circuitikz} \draw
(0,2) node[left]{$V_{\mathrm{in}}$} to[L=1~mH,o-] (3,2) to[L=1~mH] (6,2) to[short,-o] (7,2) node[right]{$V_{\mathrm{out}}$}
(3,2) to[C=470~nF] (3,0) node[ground]{}
(6,2) to[R=270~$\Omega$] (6,0) node[ground]{}
;\end{circuitikz}}
\subcaptionbox{}
{\begin{circuitikz} \draw
(0,0) node[op amp](opamp){}
(opamp.+) to[short] (-1.2,-0.5) to[C,l_=470~nF] (-1.2,-3) node[ground]{}
(opamp.-) to[short] (-1.2,0.5) to[short] (-1.2,2.2)
(opamp.out) to[short,-o] (2,0) node[right]{$V_{\mathrm{out}}$}
(opamp.down) ++ (0,-0.5) node[below]{$-3$~V} -- (opamp.down)
(opamp.up) ++ (0,0.5) node[above]{3~V} -- (opamp.up)
(-3.2,2.2) node[ground]{} to[R=10~k$\Omega$] (-1.2,2.2) to[R=20~k$\Omega$] (1.2,2.2)
(1.2,0) to[short] (1.2,3.5) to[short] (-4.5,3.5) to[C,l_=470~nF] (-4.5,-0.5)
(-10.5,-0.5) node[left]{$V_{\mathrm{in}}$} to[R=1~k$\Omega$,o-] (-7.5,-0.5) to[R=560~$\Omega$] (-4.5,-0.5) to[R=1~k$\Omega$] (-1.2,-0.5)
(-7.5,-0.5) to[C=1~$\mu$F] (-7.5,-3) node[ground]{}
;\end{circuitikz}}
\caption{\label{sch-RC-ordre3}\textbf{Sous-objectif} : examiner comment le gain du filtre RC série passe-bas de premier ordre fait au début se compare à 1 des 3 filtres ci-dessus, soit (a) RC de troisième ordre, (b) Butterworth en topologie Cauer de troisième ordre et (c) Butterworth en topologie Sallen-Key de troisième ordre. Quantifier aussi les déviations du spectre de gain idéal d'un filtre passe-bas qui est une fonction échelon de Heaviside en symétrie miroir centrée sur la fréquence de coupure. Cette symétrie de parité paire est seulement pour exprimer l'opération $H(-x)\rightarrow H(x)$.}
\end{figure}

% Vous pouvez remarquer à partir des diagrammes de Bode précédemment tracés que la transition entre la bande passante et la bande rejetée n'est pas parfaitement verticale. Il est toutefois possible de complexifier le design d'un filtre afin que cette transition soit plus abrupte. Cette analyse sera faite sur des filtres passe-bas.
%
% a) Construisez un filtre passe-bas de troisième ordre, équivalent à trois filtres RC passe-bas en série, et faites l'acquisition habituelle du spectre à la sortie $V_{\mathrm{out}}$. Ce circuit est illustré à la figure~\ref{sch-RC-ordre3}.
% \begin{figure}[h]
% \centering
% \begin{circuitikz} \draw
% (0,2) node[left]{$V_{\mathrm{g}}$} to[R=135~$\Omega$,o-] (3,2) to[R=270~$\Omega$] (6,2) to[R=560~$\Omega$] (9,2) to[short,-o] (10,2) node[right]{$V_{\mathrm{out}}$}
% (3,2) to[C=2~$\mu$F] (3,0) node[ground]{}
% (6,2) to[C=1~$\mu$F] (6,0) node[ground]{}
% (9,2) to[C=470~nF] (9,0) node[ground]{}
% ;\end{circuitikz}
% \caption{\label{sch-RC-ordre3}Filtre RC de troisième ordre.}
% \end{figure}

% b) Refaites la même acquisition et tracez le diagramme de Bode, mais cette fois en utilisant un filtre passe-bas de type Butterworth en topologie Cauer (figure~\ref{sch-Butter-Cauer-ordre3}).
% \begin{figure}[h]
% \centering
% \begin{circuitikz} \draw
% (0,2) node[left]{$V_{\mathrm{g}}$} to[L=1~mH,o-] (3,2) to[L=1~mH] (6,2) to[short,-o] (7,2) node[right]{$V_{\mathrm{out}}$}
% (3,2) to[C=470~nF] (3,0) node[ground]{}
% (6,2) to[R=270~$\Omega$] (6,0) node[ground]{}
% ;\end{circuitikz}
% \caption{\label{sch-Butter-Cauer-ordre3}Filtre Butterworth en topologie Cauer de troisième ordre.}
% \end{figure}

% c) Par la suite, tracez le diagramme de Bode d'un troisième filtre passe-bas de troisième ordre. Il s'agira cette fois d'un filtre Butterworth actif en topologie Sallen-Key (figure~\ref{sch-Butter-Sallen-Key-ordre3}).
% \begin{figure}[h]
% \centering
% \begin{circuitikz} \draw
% (0,0) node[op amp](opamp){}
% (opamp.+) to[short] (-1.2,-0.5) to[C,l_=470~nF] (-1.2,-3) node[ground]{}
% (opamp.-) to[short] (-1.2,0.5) to[short] (-1.2,2.2)
% (opamp.out) to[short,-o] (2,0) node[right]{$V_{\mathrm{out}}$}
% (opamp.down) ++ (0,-0.5) node[below]{$-3$~V} -- (opamp.down)
% (opamp.up) ++ (0,0.5) node[above]{3~V} -- (opamp.up)
% (-3.2,2.2) node[ground]{} to[R=10~k$\Omega$] (-1.2,2.2) to[R=20~k$\Omega$] (1.2,2.2)
% (1.2,0) to[short] (1.2,3.5) to[short] (-4.5,3.5) to[C,l_=470~nF] (-4.5,-0.5)
% (-10.5,-0.5) node[left]{$V_{\mathrm{g}}$} to[R=1~k$\Omega$,o-] (-7.5,-0.5) to[R=560~$\Omega$] (-4.5,-0.5) to[R=1~k$\Omega$] (-1.2,-0.5)
% (-7.5,-0.5) to[C=1~$\mu$F] (-7.5,-3) node[ground]{}
% ;\end{circuitikz}
% \caption{\label{sch-Butter-Sallen-Key-ordre3}Filtre actif Butterworth en topologie Sallen-Key de troisième ordre.}
% \end{figure}

%\section{Questions et discussion}
%
%\begin{enumerate}
%\item Sur un même graphique, tracez les courbes du gain (en décibels) en fonction de la fréquence normalisée $f/f_{c}$ des quatre filtres passe-bas dont vous avez acquis les diagrammes de Bode dans ce laboratoire, soient les trois filtres de la troisième partie ainsi que le premier filtre RC de la deuxième partie. Utilisez une échelle logarithmique pour la fréquence normalisée. Mesurez les pentes (en décibels par décade) de la courbe de gain dans les hautes fréquences.
%\end{enumerate}

\end{document}
