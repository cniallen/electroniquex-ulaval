\documentclass[12pt,oneside,letterpaper]{article}

\usepackage[canadien]{babel}
\usepackage[ansinew]{inputenc}
\usepackage[T1]{fontenc}
\usepackage{lmodern}
\usepackage{graphicx}
\usepackage[letterpaper]{geometry}
\usepackage[americanvoltages,americancurrents]{circuitikz}
\usepackage{amsmath}
\usepackage{caption}
\usepackage{subfig}
\usepackage{hyperref}
\usepackage[all]{hypcap}


\captionsetup{font=small,labelfont=bf,margin=0.1\textwidth}
\pagestyle{myheadings}
\markboth{GPH-2006/PHY-2002~---~Transformateur}{GPH-2006/PHY-2002~---~Transformateur}


\begin{document}


\title{\textbf{Compl�ment}\\Transformateur}
\author{Jean-Rapha�l Carrier \& Claudine Allen}
\date{}
\maketitle


Un transformateur est un composant qui modifie le courant et la tension dans un circuit, tout en conservant la m�me puissance. Un transformateur amplifie ainsi l'intensit� du courant par un coefficient $A$ et r�duit la tension par le m�me coefficient. �videmment, dans le cas o� $A<1$, c'est la tension qui est amplifi�e et le courant qui est r�duit. Les transformateurs sont des composants habituellement tr�s efficaces, c'est-�-dire que la puissance de sortie est presque �gale � la puissance d'entr�e (peu de pertes).

\begin{center}
\begin{circuitikz} \draw
(0,0) node[transformer]{}
;\end{circuitikz}
\end{center}

Concr�tement, un transformateur est constitu� de deux bobines d'inductance coupl�es qui ne sont toutefois pas en contact. C'est un quadrip�le ; il y a deux entr�es et deux sorties. La bobine � l'entr�e s'appelle \textit{primaire} et celle � la sortie, \textit{secondaire}. Le coefficient $A$ qui multiplie le courant de l'enroulement primaire pour obtenir le courant circulant dans le secondaire est appel� \textit{rapport de transformation du courant}. Ce dernier est �gal au rapport entre les nombres de spires du primaire et du secondaire. Par exemple, un transformateur dont le primaire pr�sente 100 tours et le secondaire 1000 a un rapport 1 : 10 (soit $A=0,\!1$) ; ainsi, la tension aux bornes du secondaire est dix fois plus �lev�e que celle aux bornes du primaire.


\end{document}

�crit par Jean-Rapha�l Carrier
Derni�re modification : 11 janvier 2014

