%%%%%%%%%%%%%%%%%%%%%%%%%%%%%%%%%%%%%%%%%%%%%%%%%%%%%%%%%%%%%%%%%%%
%DC_003.tex : 1 source indépendante de 24 V, 1 source indépendante de 6 A, 1 source indépendante de 12 A et 5 résistances
%------------------------------------------------------------------
%SOURCE: ?
%UTILISATION (D=devoir, E=examen, A=atelier de dépannage, C=en classe - année): D-2013, D-2016, D-2018, D-2020, D-2023
%DIFFICULTÉ très courte et/ou simple? (Y/N): N
%
%Créé par Jean-Raphaël Carrier
%Contributions par Claudine Nì. Allen
%Compilateur pdfLaTeX, distribution TeX Live 2023
%Dernière modification : 5 février 2023
%%%%%%%%%%%%%%%%%%%%%%%%%%%%%%%%%%%%%%%%%%%%%%%%%%%%%%%%%%%%%%%%%%%
%
\documentclass[../ElectroX-Devoir.tex]{subfiles}
%---------------------------------------------------------
\begin{document}
%---------------------------------------------------------
\begin{preview}
\onlyinsubfile{\import{./}{HeaderCustom.tex}}
%
Évaluez le courant $i_0$ dans le circuit suivant.

\begin{center}
\begin{circuitikz} \draw
(-4,0) to[V,l^=$24\,\mathrm{V}$] (0,3) to[R,l^=$2\,\Omega$] (4,0) to[R,l^=$2\,\Omega$,i=$i_0$] (0,-3) to[R,l^=$2\,\Omega$] (-4,0) to[R,l^=$2\,\Omega$] (0,0) to[I,l^=$6\,\mathrm{A}$] (4,0)
(0,3) to[R,l^=$2\,\Omega$] (0,0) to[I,l^=$12\,\mathrm{A}$] (0,-3)
;\end{circuitikz}
\end{center}\par
\vspace{2ex}
%
\end{preview}
%---------------------------------------------------------
\end{document}
%---------------------------------------------------------