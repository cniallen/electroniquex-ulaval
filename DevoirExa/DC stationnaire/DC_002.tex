%%%%%%%%%%%%%%%%%%%%%%%%%%%%%%%%%%%%%%%%%%%%%%%%%%%%%%%%%%%%%%%%%%%
%DC_002.tex : sources indépendantes de 12 V, 4 A et 4 résistances | rectangulaire
%------------------------------------------------------------------
%SOURCE: ? 
%UTILISATION (D=devoir, E=examen, A=atelier de dépannage, C=en classe - année): D-2013, D-2018, D-2020, D-2021
%DIFFICULTÉ courte et/ou simple? (Y/N): Y
%
%Créé par Jean-Raphaël Carrier
%Contributions par Claudine Nì. Allen
%Compilateur pdfLaTeX, distribution TeX Live 2023
%Dernière modification : 13 février 2024
%%%%%%%%%%%%%%%%%%%%%%%%%%%%%%%%%%%%%%%%%%%%%%%%%%%%%%%%%%%%%%%%%%%
%
\documentclass[../ElectroX-DevoirDC.tex]{subfiles}
%---------------------------------------------------------
\begin{document}
%---------------------------------------------------------
\begin{preview}
\onlyinsubfile{\import{./}{HeaderCustom.tex}}
%
En utilisant {\Large\textbf{la loi des n{\oe}uds}} et non PAS la loi des mailles, déterminez la différence de potentiel électrique $v_0$ dans le circuit suivant.

\begin{center}
\begin{circuitikz} \draw
(8,0) to[short,o-] (0,0) node[ground]{} to[V,l^=$12\,\mathrm{V}$] (0,3) to[short] (2,3) to[R,l^=$2\,\Omega$] (5,3) to[short,-o] (8,3)
(2,0) to[R,l_=$1\,\Omega$] (2,3) to[short] (2,4.5) to[I,l^=$4\,\mathrm{A}$] (5,4.5) to[short] (5,3) to[R,l_=$4\,\Omega$] (5,0)
(7,3) to[R,l_=$2\,\Omega$] (7,0)
{[anchor=west] (8,0) node{$-$} (8,1.5) node{$v_0$} (8,3) node{$+$}}
;\draw[dashed]
(8.3,0.3) to[short] (8.3,1.2)
(8.3,1.8) to[short] (8.3,2.7)
;\end{circuitikz}
\end{center}

\end{preview}
%---------------------------------------------------------
\end{document}
%---------------------------------------------------------