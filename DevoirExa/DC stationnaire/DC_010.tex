%%%%%%%%%%%%%%%%%%%%%%%%%%%%%%%%%%%%%%%%%%%%%%%%%%%%%%%%%%%%%%%%%%%
%DC_010.tex : sources indépendantes de 36 V, 12 A et 4 résistances | rectangulaire
%------------------------------------------------------------------
%SOURCE: inconnue
%UTILISATION (D=devoir, E=examen, A=atelier de dépannage, C=en classe - année): D-2022, D-2025
%DIFFICULTÉ courte et/ou simple? (Y/N): à déterminer
%
%Créé par Jean-Raphaël Carrier
%Contributions par -
%Compilateur pdfLaTeX, distribution TeX Live 2024
%Dernière modification : 2 février 2025
%%%%%%%%%%%%%%%%%%%%%%%%%%%%%%%%%%%%%%%%%%%%%%%%%%%%%%%%%%%%%%%%%%%
%
\documentclass[../ElectroX-DevoirDC.tex]{subfiles}
%---------------------------------------------------------
\begin{document}
%---------------------------------------------------------
\begin{preview}
\onlyinsubfile{\import{./}{HeaderCustom.tex}}
%
Déterminez la tension $v_0$ dans le circuit suivant.

\begin{center}
\begin{circuitikz} \draw
(9,0) to[short,o-] (0,0) to[R,l^=$2\,\Omega$] (0,3) to[short] (2,3) to[V,l^=$36\,\mathrm{V}$] (5,3) to[R,l^=$4\,\Omega$] (8,3) to[short,-o] (9,3)
(2,0) to[I,l_=$12\,\mathrm{A}$] (2,3)
(5,0) to[R,l^=$6\,\Omega$] (5,3)
(8,0) to[R,l^=$8\,\Omega$] (8,3)
{[anchor=west] (9,0) node{$-$} (9,1.5) node{$v_0$} (9,3) node{$+$}}
;\draw[dashed]
(9.3,0.2) to[short] (9.3,1.2)
(9.3,1.8) to[short] (9.3,2.7)
;\end{circuitikz}
\end{center}

%
\end{preview}
%---------------------------------------------------------
\end{document}
%---------------------------------------------------------