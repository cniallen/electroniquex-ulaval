%%%%%%%%%%%%%%%%%%%%%%%%%%%%%%%%%%%%%%%%%%%%%%%%%%%%%%%%%%%%%%%%%%%
%DC_011.tex : source de courant commandée, sources indépendantes de 12 V, 2 A et 6 résistances | rectangulaire
%------------------------------------------------------------------
%SOURCE: inconnue
%UTILISATION (D=devoir, E=examen, A=atelier de dépannage, C=en classe - année): D-2022, D-2025
%DIFFICULTÉ courte et/ou simple? (Y/N): N
%
%Créé par Jean-Raphaël Carrier
%Contributions par -
%Compilateur pdfLaTeX, distribution TeX Live 2024
%Dernière modification : 4 février 2025
%%%%%%%%%%%%%%%%%%%%%%%%%%%%%%%%%%%%%%%%%%%%%%%%%%%%%%%%%%%%%%%%%%%
%
\documentclass[../ElectroX-DevoirDC.tex]{subfiles}
%---------------------------------------------------------
\begin{document}
%---------------------------------------------------------
\begin{preview}
\onlyinsubfile{\import{./}{HeaderCustom.tex}}
%
Calculez le potentiel $v_x$ dans le circuit suivant. La référence implicite de potentiel nul est évidemment au noeud de mise à la terre.

\begin{center}
\begin{circuitikz} \draw
(0,0) node[ground]{} to[V,l^=12~V] (0,6) to[short] (2,6) to[R,l^=1~$\Omega$] (5,6) to[short] (7,6) to[R,l^=1~$\Omega$] (7,3) to[short,-*] (5,3) to[R,l^=1~$\Omega$] (2,3)
(2,6) to[R,l_=1~$\Omega$] (2,3) to[R,l_=1~$\Omega$,i=$i_0$] (2,0)
(0,0) to[short] (7,0) to[american controlled current source,l_=$4\,i_0$] (7,3)
(5,0) to[R,l^=1~$\Omega$] (5,3) to[I,l_=2~A] (5,6)
{[anchor=north west] (5,3) node{$v_x$}}
;\end{circuitikz}
\end{center}
%
\end{preview}
%---------------------------------------------------------
\end{document}
%---------------------------------------------------------