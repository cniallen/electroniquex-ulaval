\begin{subequations}

On veut trouver $i_0$ dans le circuit suivant, en utilisant la loi des n{\oe}uds:
\begin{center}
\begin{circuitikz} \draw
(0,3) to[R=$2~\Omega$,i=$i_0$] (0,0) to[short] (7,0) to[current source,l_=4~A] (7,3) to[short] (5,3) to[R=$2~\Omega$] (5,0)
(5,3) to[current source=2~A] (2,3) to[R=$1~\Omega$] (2,0) node[ground]{}
(0,3) to[short] (2,3) to[short] (2,4.5) to[R=$2~\Omega$] (5,4.5) to[short] (5,3)
;\end{circuitikz}
\end{center}
Ce circuit contient trois n{\oe}uds, donc trois valeurs de potentiels distinctes, qu'on nommera arbitrairement $v_a$, $v_b$ et $v_c$. Sachant que le potentiel à la mise à la terre est nul ($v_b=0\,\mathrm{V}$), il ne restera que deux valeurs inconnues : il faudra deux équations. Afin de résoudre le problème, on posera des directions aux courants dans les composants où ce n'était pas déjà indiqué.
\begin{center}
\begin{circuitikz} \draw
(0,3) to[R=$2~\Omega$,i=$i_0$] (0,0) to[short] (3.5,0) to[short] (7,0) to[current source,l_=4~A] (7,3) to[short] (5,3) to[R=$2~\Omega$] (5,0)
(5,3) to[current source=2~A] (2,3) to[R=$1~\Omega$] (2,0) node[ground]{}
(0,3) to[short] (2,3) to[short] (2,4.5) to[R=$2~\Omega$] (5,4.5) to[short] (5,3); \draw[color=blue]
(2,0.7) to[short,i=$i_1$] (2,0.3)
(5,0.7) to[short,i=$i_2$] (5,0.3)
(4.3,4.5) to[short,i=$i_3$] (4.7,4.5)
{[anchor=south east] (2,3) node{$v_a$}}
{[anchor=north] (3.5,0) node{$v_b=0\,\mathrm{V}$}}
{[anchor=south west] (5,3) node{$v_c$}}
;\end{circuitikz}
\end{center}
À chaque n{\oe}ud du circuit, la somme des courants sortants égale la somme des courants entrants.

\subsubsection{N{\oe}ud $A$}

Au n{\oe}ud $A$, on obtient l'équation suivante:
\begin{equation}
i_0+i_1+i_3=2\,\mathrm{A}.
\end{equation}
Sachant que le courant traversant une résistance est égal au rapport de la tension à ses bornes et de sa résistance, on trouve:
\begin{equation}
\frac{v_a}{2\,\Omega}+\frac{v_a}{1\,\Omega}+\frac{v_a-v_c}{2\,\Omega}=2\,\mathrm{A}.
\end{equation}
En multipliant cette équation par 2~$\Omega$, on obtient la forme simplifiée suivante:
\begin{equation}
\label{eq:2.1}
4\,v_a-v_c=4\,\mathrm{V}.
\end{equation}


\subsubsection{N{\oe}ud $C$}

Au n{\oe}ud $C$, on obtient:
\begin{gather}
i_2+2\,\mathrm{A}=i_3+4\,\mathrm{A}\\
i_2-i_3=2\,\mathrm{A}\\
\frac{v_c}{2\,\Omega}-\frac{v_a-v_c}{2\,\Omega}=2\,\mathrm{A}\\
\label{eq:2.2}
-v_a+2\,v_c=4\,\mathrm{V}.
\end{gather}


\subsubsection{Résolution du système d'équations linéaires}

Les équations~\ref{eq:2.1} et \ref{eq:2.2} forment un système de deux équations linéaires qui permettent de trouver les potentiels inconnus $v_a$ et $v_c$. Mais pour connaître $i_0$, nous n'avons besoin que de $v_a$.
\begin{equation}
v_a=\frac{\left|\begin{array}{ c c } 4\,\mathrm{V} & -1 \\ 4\,\mathrm{V} & 2 \end{array}\right|}{\left|\begin{array}{ c c } 4 & -1 \\ -1 & 2 \end{array}\right|}=\frac{12}{7}\,\mathrm{V}
\end{equation}
On peut alors trouver la valeur de $_0$:
\begin{gather}
i_0=\frac{v_a}{2\,\Omega}=\frac{12}{7}\,\mathrm{V}\cdot\frac{1}{2\,\Omega}.\\
\boxed{i_0=\frac{6}{7}\,\mathrm{A}}
\end{gather}

\end{subequations}