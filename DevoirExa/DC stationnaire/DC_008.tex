%%%%%%%%%%%%%%%%%%%%%%%%%%%%%%%%%%%%%%%%%%%%%%%%%%%%%%%%%%%%%%%%%%%
%DC_008.tex : sources indépendantes de 2 A, 4 A et 4 résistances | rectangulaire
%------------------------------------------------------------------
%SOURCE: ?
%UTILISATION (D=devoir, E=examen, A=atelier de dépannage, C=en classe - année): D-2013, D-2018, D-2022, D-2024
%DIFFICULTÉ courte et/ou simple? (Y/N): Y
%
%Créé par Claudine Nì. Allen
%Contributions par -
%Compilateur pdfLaTeX, distribution TeX Live 2023
%Dernière modification : 13 février 2024
%%%%%%%%%%%%%%%%%%%%%%%%%%%%%%%%%%%%%%%%%%%%%%%%%%%%%%%%%%%%%%%%%%%
%
\documentclass[../ElectroX-DevoirDC.tex]{subfiles}
%---------------------------------------------------------
\begin{document}
%---------------------------------------------------------
\begin{preview}
\onlyinsubfile{\import{./}{HeaderCustom.tex}}
%
 En utilisant {\Large\textbf{la loi des mailles}} et non PAS la loi des n{\oe}uds, déterminez la puissance dissipée par la résistance de $2\,\Omega$ dans le circuit suivant.

\begin{center}
\begin{circuitikz} \draw
(0,3) to[R,l^=$4\,\Omega$] (3,3) to[I,l^=$4\,\mathrm{A}$] (6,3) to[R,l^=$3\,\Omega$] (6,0) to[short] (3,0) to[R,l^=$2\,\Omega$] (3,3)
(0,3) to[short] (0,5) to[R,l^=$1\,\Omega$] (6,5) to[short] (6,3)
(0,3) to[I,l_=$2\,\mathrm{A}$] (0,0) to[short] (3,0)
;\end{circuitikz}
\end{center}
%
\end{preview}
%---------------------------------------------------------
\end{document}
%---------------------------------------------------------