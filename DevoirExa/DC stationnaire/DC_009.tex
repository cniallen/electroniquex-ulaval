%%%%%%%%%%%%%%%%%%%%%%%%%%%%%%%%%%%%%%%%%%%%%%%%%%%%%%%%%%%%%%%%%%%
%DC_009.tex : 11 résistances équivalentes |
%------------------------------------------------------------------
%SOURCE: Steck, "Analog and Digital Electronics" (2020 ed. online), section 1.6, p.39, exemple 1.6.1
%UTILISATION (D=devoir, E=examen, A=atelier de dépannage, C=en classe - année): D-2021, D-2024
%DIFFICULTÉ courte et/ou simple? (Y/N): N
%
%Créé par Claudine Nì. Allen
%Contributions par -
%Compilateur pdfLaTeX, distribution TeX Live 2023
%Dernière modification : 13 février 2024
%%%%%%%%%%%%%%%%%%%%%%%%%%%%%%%%%%%%%%%%%%%%%%%%%%%%%%%%%%%%%%%%%%%
%
\documentclass[../ElectroX-DevoirDC.tex]{subfiles}
%---------------------------------------------------------
\begin{document}
%---------------------------------------------------------
\begin{preview}
\onlyinsubfile{\import{./}{HeaderCustom.tex}}
%
Sachant que toutes les résistances du circuit ci-dessous sont identiques et de valeur R, quelle est la résistance équivalente entre les points \textbf{A} et \textbf{B}?
%
\begin{center}
\begin{circuitikz} \draw
(0,0) to[R] (4.5,2) node[above]{\textbf{A}}
(3,0) to[R] (4.5,2)
(6,0) to[R] (4.5,2)
(9,0) to[R] (4.5,2)
(0,0) to[R] (3,0) to[R] (6,0) to[R] (9,0)
(0,0) to[R] (4.5,-2)
(3,0) to[R] (4.5,-2)
(6,0) to[R] (4.5,-2)
(9,0) to[R] (4.5,-2) node[below]{\textbf{B}}
;\end{circuitikz}
\end{center}
%
\end{preview}
%---------------------------------------------------------
\end{document}
%---------------------------------------------------------