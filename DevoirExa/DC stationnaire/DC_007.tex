%%%%%%%%%%%%%%%%%%%%%%%%%%%%%%%%%%%%%%%%%%%%%%%%%%%%%%%%%%%%%%%%%%%
%DC_007.tex : sources indépendantes de 3 A, 8 A, 25 A et 5 résistances | rectangulaire
%------------------------------------------------------------------
%SOURCE: Hayt & Kemmerly, "Titre?" (? ed., ? year, cf ULaval library), section 3.2, p.60, exemple fig. 3.2
%UTILISATION (D=devoir, E=examen, A=atelier de dépannage, C=en classe - année): D-2014, D-2015, D-2017, D-2021, D-2024
%DIFFICULTÉ courte et/ou simple? (Y/N): N
%
%Créé par Claudine Nì. Allen
%Contributions par -
%Compilateur pdfLaTeX, distribution TeX Live 2023
%Dernière modification : 13 février 2024
%%%%%%%%%%%%%%%%%%%%%%%%%%%%%%%%%%%%%%%%%%%%%%%%%%%%%%%%%%%%%%%%%%%
%
\documentclass[../ElectroX-DevoirDC.tex]{subfiles}
%---------------------------------------------------------
\begin{document}
%---------------------------------------------------------
\begin{preview}
\onlyinsubfile{\import{./}{HeaderCustom.tex}}
%
En utilisant {\Large\textbf{la loi des n{\oe}uds}} et non PAS la loi des mailles, évaluez la tension de sortie $v_0$ dans le circuit suivant.
%
\begin{center}
\begin{circuitikz} \draw
(0,0) to[short] (10,0) node[ground]{} to[R,l_=$\frac{1}{5}\Omega$] (10,3) to[short] (7,3) to[R,l_=$\frac{1}{2}\Omega$] (4,3) to[R,l^=$\frac{1}{3}\Omega$] (1,3) to[short] (0,3) to[I,l_=8~A] (0,0)
(1,3) to[short] (1,5) to[I,l^=3~A] (4,5) to[short] (4,3) to[R,l^=1~$\Omega$] (4,0)
(0,3) to[short] (0,6.5) to[R,l^=$\frac{1}{4}\Omega$] (7,6.5) to[short] (7,3)
(7,0) to[I,l_=25~A] (7,3)
(0,3) to[short,-o] (-1.5,3)
(0,0) to[short,-o] (-1.5,0)
{[anchor=east] (-1.5,0) node{$-$} (-1.5,1.5) node{$v_0$} (-1.5,3) node{$+$}}
;\draw[dashed]
(-1.7,0.3) to[short] (-1.7,1.2)
(-1.7,1.8) to[short] (-1.7,2.7)
;\end{circuitikz}
\end{center}
%
\end{preview}
%---------------------------------------------------------
\end{document}
%---------------------------------------------------------