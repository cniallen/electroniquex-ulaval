%%%%%%%%%%%%%%%%%%%%%%%%%%%%%%%%%%%%%%%%%%%%%%%%%%%%%%%%%%%%%%%%%%%
%DC_004.tex : 1 source indépendante de 4 A, 1 source indépendante de 2 A et 2 résistances | Rectangulaire
%------------------------------------------------------------------
%SOURCE: Irwin, "Basic engineering circuit analysis" (? ed., ? year, cf ULaval library), Chap.2 p.67 #2.21
%UTILISATION (D=devoir, E=examen, A=atelier de dépannage, C=en classe - année): D-2014, D-2015, D-2017, D-2023
%DIFFICULTÉ courte et/ou simple? (Y/N): Y
%
%Créé par Claudine Nì. Allen
%Contributions par -
%Compilateur pdfLaTeX, distribution TeX Live 2023
%Dernière modification : 5 février 2023
%%%%%%%%%%%%%%%%%%%%%%%%%%%%%%%%%%%%%%%%%%%%%%%%%%%%%%%%%%%%%%%%%%%
%
\documentclass[../ElectroX-Devoir.tex]{subfiles}
%---------------------------------------------------------
\begin{document}
%---------------------------------------------------------
\begin{preview}
\onlyinsubfile{\import{./}{HeaderCustom.tex}}
%
Trouvez le courant $i_0$ et le potentiel électrique $v_0$ par rapport au noeud de mise à la terre dans le circuit suivant.

\begin{center}
\begin{circuitikz} \draw
(0,0) to[R,l^=$6\,\Omega$] (0,3) to[short] (9,3) to[I,l^=$2\,\mathrm{A}$] (9,0) to[short] (0,0)
(3,0) node[ground]{} to[I,l^=$4\,\mathrm{A}$,-o] (3,3.4) 
{[anchor=south] (3,3.5) node{$v_0$}}
(6,3) to[R,l^=$3\,\Omega$,i=$i_0$] (6,0)
;\end{circuitikz}
\end{center}
%
\end{preview}
%---------------------------------------------------------
\end{document}
%---------------------------------------------------------