%%%%%%%%%%%%%%%%%%%%%%%%%%%%%%%%%%%%%%%%%%%%%%%%%%%%%%%%%%%%%%%%%%%
%DC_005.tex : 1 source indépendante de 12 V et 6 résistances | Rectangulaire
%------------------------------------------------------------------
%SOURCE: se retrouvait de manière louche sur http://stuffle.net/EE%202240/practice, mais le domaine a été détourné (hijacked).
%UTILISATION (D=devoir, E=examen, A=atelier de dépannage, C=en classe - année): D-2018, D-2023
%DIFFICULTÉ courte et/ou simple? (Y/N): N
%
%Créé par Claudine Nì. Allen
%Contributions par -
%Compilateur pdfLaTeX, distribution TeX Live 2023
%Dernière modification : 5 février 2023
%%%%%%%%%%%%%%%%%%%%%%%%%%%%%%%%%%%%%%%%%%%%%%%%%%%%%%%%%%%%%%%%%%%
%
\documentclass[../ElectroX-DevoirDC.tex]{subfiles}
%---------------------------------------------------------
\begin{document}
%---------------------------------------------------------
\begin{preview}
\onlyinsubfile{\import{./}{HeaderCustom.tex}}
%
Quelle est la valeur du courant $i_0$ dans le circuit suivant?

\begin{center}
\begin{circuitikz} \draw
(0,0) to[R,l^=$4\,\mathrm{k}\Omega$, i=$i_0$]
(0,3) to[R,l^=$12\,\mathrm{k}\Omega$]
(3,3) to[R,l^=$4\,\mathrm{k}\Omega$]
(6,3) to[R,l^=$6\,\mathrm{k}\Omega$]
(9,3) to[short]
(9,0) to[short]
(0,0)
(3,0) to[R,l^=$16\,\mathrm{k}\Omega$]
(3,3)
(6,0) node[ground]{} to[V,l_=$12\,$V]
(6,1.5) to[R,l_=$2\,\mathrm{k}\Omega$]
(6,3)
;\end{circuitikz}
\end{center}
%
\end{preview}
%---------------------------------------------------------
\end{document}
%---------------------------------------------------------