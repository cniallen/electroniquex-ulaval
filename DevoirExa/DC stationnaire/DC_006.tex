%%%%%%%%%%%%%%%%%%%%%%%%%%%%%%%%%%%%%%%%%%%%%%%%%%%%%%%%%%%%%%%%%%%
%DC_006.tex : source indépendante de 12 V et 7 résistances | rectangulaire
%------------------------------------------------------------------
%SOURCE: inconnue
%UTILISATION (D=devoir, E=examen, A=atelier de dépannage, C=en classe - année): D-2022, D-2025
%DIFFICULTÉ courte et/ou simple? (Y/N): Y
%
%Créé par Jean-Raphaël Carrier
%Contributions par -
%Compilateur pdfLaTeX, distribution TeX Live 2024
%Dernière modification : 4 février 2025
%%%%%%%%%%%%%%%%%%%%%%%%%%%%%%%%%%%%%%%%%%%%%%%%%%%%%%%%%%%%%%%%%%%
%
\documentclass[../ElectroX-DevoirDC.tex]{subfiles}
%---------------------------------------------------------
\begin{document}
%---------------------------------------------------------
\begin{preview}
\onlyinsubfile{\import{./}{HeaderCustom.tex}}
%
\textbf{a)} Évaluez le courant $i_0$ dans le circuit suivant.

\textbf{b)} Évaluez $i_0$ si on rajoutait un condensateur de $1\,\mu\mathrm{F}$ entre les points $\mathbf{A}$ et $\mathbf{B}$.

\begin{center}
\begin{circuitikz} \draw
(0,0) to[V,l^=$12\,\mathrm{V}$,i=$i_0$] (0,4) to[short,-*] (1.5,4) node[above]{$\mathbf{A}$} to[R,l^=$1\,\Omega$] (4.5,4) to[short] (6,4) to[R,l^=$2\,\Omega$] (8,4) to[R,l^=$4\,\Omega$] (8,2) to[R,l^=$8\,\Omega$] (8,0) to[short,-*] (1.5,0) node[above]{$\mathbf{B}$} to[short] (0,0)
(6,0) to[R,l^=$8\,\Omega$] (6,2) to[R,l^=$6\,\Omega$] (6,4)
(4.5,4) to[R,l_=$3\,\Omega$] (4.5,2) to[short] (8,2)
;\end{circuitikz}
\end{center}
%
\end{preview}
%---------------------------------------------------------
\end{document}
%---------------------------------------------------------