%%%%%%%%%%%%%%%%%%%%%%%%%%%%%%%%%%%%%%%%%%%%%%%%%%%%%%%%%%%%%%%%%%%
%DC_001.tex : 1 source de 12 V indépendante, 1 source de courant commandée et 3 résistances. | Rectangulaire
%------------------------------------------------------------------
%SOURCE: Irwin, "Basic engineering circuit analysis" (? ed., ? year, cf ULaval library), Chap.3 p.121 #3.46
%UTILISATION (D=devoir, E=examen, A=atelier de dépannage - session année): D-h14,D-a15,D-a17,D-a20,D-h23
%DIFFICULTÉ courte et/ou simple (Y/N): Y
%
%Créé par Claudine Nì. Allen
%Contributions par -
%Compilateur pdfLaTeX, distribution TeX Live 2023
%Dernière modification : 5 février 2023
%%%%%%%%%%%%%%%%%%%%%%%%%%%%%%%%%%%%%%%%%%%%%%%%%%%%%%%%%%%%%%%%%%%
%
\documentclass[../PES-Devoir.tex]{subfiles}
%---------------------------------------------------------
\begin{document}
%---------------------------------------------------------
\begin{preview}
\onlyinsubfile{\import{./}{HeaderCustom.tex}}
%
Déterminez le courant $i_0$ dans le circuit ci-dessous. Notez que la source de tension commandée fournit une différence de potentiel de 2$i_0$ : la constante 2 a donc des unités de $\Omega$.
%
\begin{center}
\begin{circuitikz} \draw
(0,0) to[V,l^=12~V] (0,3) to[short] (3,3) to[R,l^=2~$\Omega$,i=$i_0$] (6,3) to[short] (9,3) to[R,l^=2~$\Omega$] (9,0) to[short] (0,0)
(3,0) to[R,l_=2~$\Omega$] (3,3)
(6,0) to[controlled voltage source,l_=2$i_0$] (6,3)
;\end{circuitikz}
\end{center}
%
\end{preview}
%---------------------------------------------------------
\end{document}
%---------------------------------------------------------