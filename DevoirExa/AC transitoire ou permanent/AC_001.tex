%%%%%%%%%%%%%%%%%%%%%%%%%%%%%%%%%%%%%%%%%%%%%%%%%%%%%%%%%%%%%%%%%%%
%DC_001.tex : 3 condensateurs et 3 résistances | Passe-bande, signal d'entrée  ...constante de temps (fonction de transfert)
%------------------------------------------------------------------
%SOURCE: ?
%UTILISATION (D=devoir, E=examen, A=atelier de dépannage, C=en classe - année): D-2013, D-2015, D-2022, D-2023
%DIFFICULTÉ courte et/ou simple? (Y/N): N
%
%Créé par Jean-Raphaël Carrier
%Contributions par Claudine Nì. Allen
%Compilateur pdfLaTeX, distribution TeX Live 2023
%Dernière modification : 17 novembre 2023
%%%%%%%%%%%%%%%%%%%%%%%%%%%%%%%%%%%%%%%%%%%%%%%%%%%%%%%%%%%%%%%%%%%
%
\documentclass[../ElectroX-DevoirAC.tex]{subfiles}
%---------------------------------------------------------
\begin{document}
%---------------------------------------------------------
\begin{preview}
\onlyinsubfile{\import{./}{HeaderCustom.tex}}
%
Déterminez le courant $i_0$ dans le circuit ci-dessous. Notez que la source de tension commandée fournit une différence de potentiel de 2$i_0$ : la constante 2 a donc des unités de $\Omega$.
%
\begin{center}
\begin{circuitikz} \draw
(0,0) to[V,l^=12~V] (0,3) to[short] (3,3) to[R,l^=2~$\Omega$,i=$i_0$] (6,3) to[short] (9,3) to[R,l^=2~$\Omega$] (9,0) to[short] (0,0)
(3,0) to[R,l_=2~$\Omega$] (3,3)
(6,0) to[controlled voltage source,l_=2$i_0$] (6,3)
;\end{circuitikz}
\end{center}
%
\end{preview}
%---------------------------------------------------------
\end{document}
%---------------------------------------------------------