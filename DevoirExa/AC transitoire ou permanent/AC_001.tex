%%%%%%%%%%%%%%%%%%%%%%%%%%%%%%%%%%%%%%%%%%%%%%%%%%%%%%%%%%%%%%%%%%%
%AC_001.tex : 3 condensateurs et 3 résistances | Passe-bande, Signal d'entrée
%------------------------------------------------------------------
%SOURCE: ?
%UTILISATION (D=devoir, E=examen, A=atelier de dépannage, C=en classe - année): D-2013, D-2015, D-2022, D-2023
%DIFFICULTÉ courte et/ou simple? (Y/N): N
%
%Créé par Jean-Raphaël Carrier
%Contributions par Claudine Nì. Allen
%Compilateur pdfLaTeX, distribution TeX Live 2023
%Dernière modification : 17 novembre 2023
%%%%%%%%%%%%%%%%%%%%%%%%%%%%%%%%%%%%%%%%%%%%%%%%%%%%%%%%%%%%%%%%%%%
%
\documentclass[../ElectroX-DevoirAC.tex]{subfiles}
%---------------------------------------------------------
\begin{document}
%---------------------------------------------------------
\begin{preview}
\onlyinsubfile{\import{./}{HeaderCustom.tex}}
%
Soit le filtre illustré à la figure~\ref{fig:circuit-q2}.\\[4mm]
\textbf{a)} Évaluez sa fonction de transfert $H\!\left(s\right)$ dans le domaine de Laplace.\\[4mm]
\textbf{b)} Tracez le diagramme de Bode du gain $G\!\left(\omega\right)= \left|H\!\left(\omega\right)\right|$ en régime permanent. Mettez l'axe des fréquences en échelle logarithmique pour couvrir une plage de \qty{1}{\hertz} à \qty{15}{\mega\hertz}.\\[1mm]
\emph{Indice: L'analyse du circuit se fait plus simplement avec la loi des mailles après simplification.}\\[4mm]
\textbf{c)} S'agit-il d'un filtre passe-bas, passe-haut, passe-bande ou coupe-bande?\\[4mm]
\textbf{d)} Pour un signal périodique $v_{\mathrm{in}}\!\left(t\right)=4.44\cos(222\pi t)$ en entrée du filtre, quelle sera la fréquence du signal à la sortie, $v_{\mathrm{out}}\!\left(t\right)$, en régime permanent?
\vspace{4ex}

\begin{figure}[h]
\begin{center}
\begin{circuitikz} \draw
(2,5) to[short] (2,8) to[R,l^=270~$\Omega$] (12,8) to[short] (12,5)
(6,5) to[short,-o] (6,6)
(8,5) to[short,-o] (8,6)
{[anchor=north] (6,6.5) node{$+$} (7.7,6.3) node[left]{$v_{\mathrm{out}}\!\left(t\right)$} (8,6.5) node{$-$}}
(0,5) node[left]{$v_{\mathrm{in}}\!\left(t\right)$} to[short,o-] (2,5) to[C, l^=1~$\mu$F] (5,5) to[R,l_=12~$\Omega$] (9,5) to[C,l^=1~$\mu$F] (12,5)
(2,5) to[short] (2,3) to[C,l_=1~$\mu$F] (12,3)
(12,5) to[short] (12,3) to[R,l^=12~$\Omega$] (12,0.5) node[ground]{}
;\end{circuitikz}
\end{center}
\caption{Filtre du signal d'entrée $v_{\mathrm{in}}\!\left(t\right)$, dont la fonction est à déterminer. La tension de sortie $v_{\mathrm{out}}\!\left(t\right)$ est la différence de potentiel aux bornes de la résistance centrale de 12~$\Omega$.}\label{fig:circuit-q2}
\end{figure}
%
\end{preview}
%---------------------------------------------------------
\end{document}
%---------------------------------------------------------