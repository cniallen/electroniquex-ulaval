%%%%%%%%%%%%%%%%%%%%%%%%%%%%%%%%%%%%%%%%%%%%%%%%%%%%%%%%%%%%%%%%%%%
%AC_002.tex : 1 inductance, 1 condensateur et 2 résistances | Coupe-bande, Phase
%------------------------------------------------------------------
%SOURCE: ?
%UTILISATION (D=devoir, E=examen, A=atelier de dépannage, C=en classe - année): D-2015, D-2017, D-2018, D-2023
%DIFFICULTÉ courte et/ou simple? (Y/N): O
%
%Créé par Claudine Nì. Allen
%Contributions par -
%Compilateur pdfLaTeX, distribution TeX Live 2023
%Dernière modification : 17 novembre 2023
%%%%%%%%%%%%%%%%%%%%%%%%%%%%%%%%%%%%%%%%%%%%%%%%%%%%%%%%%%%%%%%%%%%
%
\documentclass[../ElectroX-DevoirAC.tex]{subfiles}
%---------------------------------------------------------
\begin{document}
%---------------------------------------------------------
\begin{preview}
\onlyinsubfile{\import{./}{HeaderCustom.tex}}
%
Soit le filtre illustré à la figure~\ref{fig:circuit-q1}.

\textbf{a)} Évaluez sa fonction de transfert $H\!\left(s\right)$ dans le domaine de Laplace.

\textbf{b)} Évaluez et tracez le diagramme de Bode du gain $\displaystyle G\!\left(\omega\right)=\left|H\!\left(\omega\right)\right|$ en régime permanent. Mettez l'axe des fréquences en échelle logarithmique pour couvrir une plage de \qty{1}{\hertz} à \qty{15}{\mega\hertz}.

\textbf{c)} Sans oublier de continuer à détailler votre démarche, tracez le diagramme de Bode de la phase $\varphi\!\left(\omega\right)$.

\textbf{d)} Est-ce qu'il s'agit d'un filtre passe-haut, passe-bas, passe-bande ou coupe-bande?

\textbf{e)} Déterminez s'il est possible de trouver une ou deux fréquence(s) de coupure correspondant à $-3$~dB de la valeur maximale de $G\!\left(\omega\right)$. Évaluez cette(ces) valeur(s) $\omega_{c1}$ (, $\omega_{c2}$), puis identifiez-les(la) sur le graphique en \textbf{b)}.
\vspace{4ex}

\begin{figure}[h]
\centering
\begin{circuitikz} \draw
(0,3) node[left]{$v_{\mathrm{in}}\!\left(t\right)$} to[R,l^=$270\,\Omega$,o-] (3,3) to[short] (6,3) to[short,-o] (8,3) node[right]{$v_{\mathrm{out}}\!\left(t\right)$}
(3,3) to[L,l_=1~mH] (3,1.5) to[C,l_=$1\,\mu\mathrm{F}$] (3,0) node[ground]{}
(6,3) to[R,l^=1~k$\Omega$] (6,0) node[ground]{}
;\end{circuitikz}
\caption{Filtre du signal d'entrée $v_{\mathrm{in}}\!\left(t\right)$, dont la fonction est à déterminer.}\label{fig:circuit-q1}
\end{figure}
%
\end{preview}
%---------------------------------------------------------
\end{document}
%---------------------------------------------------------