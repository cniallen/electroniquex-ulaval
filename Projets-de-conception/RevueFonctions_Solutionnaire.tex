%Format inspired from : https://cheatography.com/fredv/cheat-sheets/eq-tips/tex/
\documentclass[english,french,12pt]{article}
\usepackage{longtable}
\usepackage{multirow}
\usepackage{geometry}
\usepackage{fancyhdr}           % For header and footer
\usepackage{multicol}           % Allows multicols in tables
\usepackage{tabularx}           % Intelligent column widths
\usepackage{tabulary}           % Used in header and footer
\usepackage{hhline}             % Border under tables
\usepackage{graphicx}           % For images
\usepackage{xcolor}             % For hex colours
\usepackage[utf8x]{inputenc}    % For unicode character support
\usepackage[T1]{fontenc}        % Without this we get weird character replacements
\usepackage{colortbl}           % For coloured tables
\usepackage{setspace}           % For line height
\usepackage{lastpage}           % Needed for total page number
\usepackage{seqsplit}           % Splits long words.

% Lengths and widths
\geometry{  %%for geometry package
%    showframe,  %%uncomment to see lines delimiting the dimension boxes
    letterpaper,
    margin=0.75in}
\pagestyle{fancy}
\fancyhf{}
\fancyfoot[C]{}
%\addtolength{\textwidth}{6cm}
%\addtolength{\textheight}{-1cm}
%\addtolength{\hoffset}{-3cm}
%\addtolength{\voffset}{-2cm}
\setlength{\tabcolsep}{0.2cm} % Space between columns
%\setlength{\headsep}{-12pt} % Reduce space between header and content
%\setlength{\headheight}{85pt} % If less, LaTeX automatically increases it
\renewcommand{\footrulewidth}{0pt} % Remove footer line
\renewcommand{\headrulewidth}{0pt} % Remove header line
\renewcommand{\seqinsert}{\ifmmode\allowbreak\else\-\fi} % Hyphens in seqsplit
% This two commands together give roughly
% the right line height in the tables
\renewcommand{\arraystretch}{1.7}
\onehalfspacing

% Commands
\newcommand{\SetRowColor}[1]{\noalign{\gdef\RowColorName{#1}}\rowcolor{\RowColorName}} % Shortcut for row colour
\newcommand{\mymulticolumn}[3]{\multicolumn{#1}{>{\columncolor{\RowColorName}}#2}{#3}} % For coloured multi-cols
\newcolumntype{x}[1]{>{\raggedright}p{#1}} % New column types for ragged-right paragraph columns
\newcommand{\tn}{\tabularnewline} % Required as custom column type in use

% Font and Colours
\definecolor{HeadBackground}{HTML}{333333}
\definecolor{FootBackground}{HTML}{666666}
\definecolor{TextColor}{HTML}{333333}
\definecolor{DarkBackground}{HTML}{C48027}
\definecolor{LightBackground}{HTML}{FBF7F1}
\renewcommand{\familydefault}{\sfdefault}
\color{TextColor}

\begin{document}
\section*{Revue de blocs de circuits}
\raggedright
\raggedcolumns

\begin{tabularx}{\textwidth}{ 
   >{\raggedright\arraybackslash}X 
   >{\centering\arraybackslash}X  }%{x{8 cm} x{8 cm} }

\SetRowColor{DarkBackground}
\textbf{\large Fonction exécutée} & \textbf{\large Où se trouve ce circuit?} \tn 
\SetRowColor{LightBackground}
Dissipation de puissance (effet Joule) & Partout : résistance\tn 
\SetRowColor{white}
Emmagasinage d'énergie & Labo III : condensateur \& inductance\tn 
\SetRowColor{LightBackground}
Variation de résistance & Labo II : potentiomètre\tn 
\SetRowColor{white}
Division de tension & Cours 3\tn 
\SetRowColor{LightBackground}
Division de courant & Cours 3 \tn 
\SetRowColor{white}
Régulation de tension & Labo III : puce NCV7805 et diode Zener \tn 
\SetRowColor{LightBackground}
Délais et déphasages & Labo IV : condensateurs (et inductances)\tn 
\SetRowColor{white}
Polarisation de direction du courant dans un sens & Labo III : diode \tn
\SetRowColor{LightBackground}
Minimisation écologique des pertes de puissance & Labo V\tn 
\SetRowColor{white}
Minimisation écologique des réflexions parasites & Labo V \tn
\SetRowColor{LightBackground}
Redressement du courant alternatif & Labo VI\tn 
\SetRowColor{white}
Conversion CA$\rightarrow$ CC & C'est du redressement! Labo VI \tn
\SetRowColor{LightBackground}
Amplification de puissance & Labo VI\tn 
\SetRowColor{white}
Suivi de tension & Labo VI\tn
\SetRowColor{LightBackground}
Comparaison de tension & Labo VI\tn 
\SetRowColor{white}
Filtrage de fréquences & Cours 7 \& Labo VII\tn
 \SetRowColor{LightBackground}
Génération CA & Labo VIII (oscillateurs)\tn 
\SetRowColor{white}
Chronométrage & Labo VIII (oscillateurs)\tn
 \SetRowColor{LightBackground}
Sonorisation & Labo VIII (oscillateurs)\tn 
\SetRowColor{white}
Opération de calculs logique & Ancien Labo IX\tn
 \SetRowColor{LightBackground}
Conversion analogique-numérique & Ancien Labo IX\tn 
\end{tabularx}

\end{document}

vInitiale de Pierre Girard-Collins et Nicolas Payeur

Fonction	Provenance
Diviseur de tension	Cours asynchrone 3
Diviseur de courant	Cours asynchrone 3
Batterie (très non idéale)	Labo 1 (patate)
Ampèremètre (mauvais)	Labo 1 (dernier circuit avec le DAQ)
Dissipation de puissance – Effet Joule	Labo 2 (Un peu implicite - calculs de puissance max) + Labo 5
Résistance variable	Labo 2 (potentiomètre)
Polarisation du courant	Labo 3 (Diode standard + Zener)
Régulateur de tension	Labo 3 (puce NCV7805 et diode Zener)
Antenne	Labo 4
Induction électromagnétique	Labo 4 (avec fil entortillé autour de l’antenne)
Maximisation la puissance transmise	Lab 5 partie 1 et 2
Élimination d’ondes parasites	Labo 5 partie 2
Redressement de courant alternatif	Lab 6 partie 1
Utilisation d’un pont de diodes	Lab 6 partie 1
Amplification de signal avec transistor	Lab 6 partie 2 et 3
Amplification du signal avec ampli-op	Lab 6 partie 4
Suiveur de tension	Lab 6 partie 4
Comparateur de signaux	Lab 6 partie 5
Filtre passe-haut	Atelier 7
Filtre passe-bas	Atelier 7
Filtre passe-bande	Atelier 7
Génération d’oscillations	Atelier 8
Portes Logiques	Atelier 9
Conversion analogique-numérique	Atelier 9 (conception d’un voltmètre en base 1)
