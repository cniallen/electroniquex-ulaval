\documentclass{article}
\usepackage{graphicx}
\usepackage{longtable}
\usepackage{multirow}
\usepackage{caption}
\usepackage{geometry}

\begin{document}

\begin{center}
%\begin{table}[h]
\begin{longtable}{ p{0.75\textwidth} p{0.2\textwidth} } 
%\begin{tabular}{c|c} 
%\renewcommand{\arraystretch}{1.5}
\hline
Fonction exécutée par le circuit & Où?\\
\hline
\hline
Mesurage en conversion tension-courant & \\
Dissipation de puissance  &  \\
Variation de résistance & \\
Division de tension &  \\
Division de courant & \\
Régulation de tension & \\
Minimisation écologique des pertes de puissance & \\
Minimisation écologique des réflexions parasites & \\
Redressement du courant alternatif & \\
Conversion CA$\rightarrow$ CC & \\
Amplification de puissance & \\
Suivi de tension & \\
Comparaison de tension & \\
Filtrage de fréquences & \\
Génération CA & \\
Chronométrage & \\
Sonorisation & \\
Opération de calculs logique & \\
Conversion analogique-numérique & \\
%emmagasinage d'énergie et introduction de délais avec capaciteur et inducteurs
%polarisation de la direction du courant continu avec une diode
%il y en avait un autre auquel j'ai pensé pendant le lancement qui serait utile projet de conception, mais je ne m'en rappelle plus :(
\hline
%\end{tabular}
\caption{Révision des cours et expériences pour rassembler des fonctions qui peuvent être accomplies par un circuit électronique.}
\label{tab:electronic_fcts}
\end{longtable}
\end{center}

\end{document}

vInitiale de Pierre Girard-Collins et Nicolas Payeur

Fonction	Provenance
Diviseur de tension	Cours asynchrone (lequel?)
Diviseur de courant	Cours asynchrone (lequel?)
Batterie (très non idéale)	Labo 1 (patate)
Ampèremètre (mauvais)	Labo 1 (dernier circuit avec le DAQ)
Dissipation de puissance – Effet Joule	Labo 2 (Un peu implicite - calculs de puissance max) + Labo 5
Résistance variable	Labo 2 (potentiomètre)
Polarisation du courant	Labo 3 (Diode standard + Zener)
Régulateur de tension	Labo 3 (puce NCV7805 et diode Zener)
Antenne	Labo 4
Induction électromagnétique	Labo 4 (avec fil entortillé autour de l’antenne)
Maximisation la puissance transmise	Lab 5 partie 1 et 2
Élimination d’ondes parasites	Labo 5 partie 2
Redressement de courant alternatif	Lab 6 partie 1
Utilisation d’un pont de diodes	Lab 6 partie 1
Amplification de signal avec transistor	Lab 6 partie 2 et 3
Amplification du signal avec ampli-op	Lab 6 partie 4
Suiveur de tension	Lab 6 partie 4
Comparateur de signaux	Lab 6 partie 5
Filtre passe-haut	Atelier 7
Filtre passe-bas	Atelier 7
Filtre passe-bande	Atelier 7
Génération d’oscillations	Atelier 8
Portes Logiques	Atelier 9
Conversion analogique-numérique	Atelier 9 (conception d’un voltmètre en base 1)
