\documentclass{article}
\usepackage{graphicx}
\usepackage{longtable}
\usepackage{multirow}
\usepackage{caption}
\usepackage{geometry}

\begin{document}
% trucs du parfait petit inventeur.euse, chercheur.eus
% TEXTE D'INTRO DES ATELIERS : Avant la séance d’atelier, écrivez sommairement dans votre cahier de recherche tout ce que vous prévoyez faire au laboratoire pour accomplir les livrables de la prochaine section. N’oubliez pas de calculer toute valeur de référence donnée en complément aux fins de comparaison à l’expérience. En plus des manipulations expérimentales, des mesures à prendre et de leur analyse, n’oubliez pas de considérer des étapes de modélisation, conception et simulation. Plusieurs logiciels spécialisés pour l’électronique listés sur le site de cours, dans la section Logiciels de l’onglet Matériel didactique, vous aideront dans cette préparation. En particulier, l’onglet Circuits de l’application Web de Paul Falstad contient tous les blocs en atelier, montrant directement les "réponses" des comportements idéaux attendus 1. Comme la programmation LabVIEW s’avère souvent chronophage, vous pouvez être soulagés qu’aucun VI n’est demandé pour les ateliers. Toutefois pour mieux caractériser les filtres avec des diagrammes de Bode lorsqu’ils sont mentionnés en sous-objectifs ci-dessous, utilisez le VI LabVIEW déjà prêt dans le module de cet atelier.
% KISS
% design big but test the smallest part you can while making the circuit, same for code
% know your basic tools and continue to learn (remplir la revue ci-dessous!), but with the internet getting increasingly messier -> médiagraphie du cours (quand elle sera bonne) & cet endroit qui existe aussi virtuellement appelé bibliothèque
% Recherche de litérature pour des cas de design similaires, transférer mon blabla DIY ici, ajouter les forums Reddit etc. de ce monde qui fittent, et voir avec Dario + son collègue GEL pour des revues/journaux appropriés à la "Popular mechanics" qui se seraient développés en même temps que l'ère électronique et les ordis
% love your Kirchhoff laws as soon as you see linear circuit parts 
% copy-paste of error messages from program code to search online
% with just that, we made a simple gated amplifier synchronized to a particle accelerator that improved the SNR of the measured scintillation signal :)
\begin{center}
%\begin{table}[h]
\begin{longtable}{ p{0.75\textwidth} p{0.2\textwidth} } 
%\begin{tabular}{c|c} 
%\renewcommand{\arraystretch}{1.5}
\hline
Fonction exécutée par le circuit & Où?\\
\hline
\hline
Mesurage en conversion tension-courant & \\
Dissipation de puissance  &  \\
Variation de résistance & \\
Division de tension &  \\
Division de courant & \\
Régulation de tension & \\
Minimisation écologique des pertes de puissance & \\
Minimisation écologique des réflexions parasites & \\
Redressement du courant alternatif & \\
Conversion CA$\rightarrow$ CC & \\
Amplification de puissance & \\
Suivi de tension & \\
Comparaison de tension & \\
Filtrage de fréquences & \\
Génération CA & \\
Chronométrage & \\
Sonorisation & \\
Opération de calculs logique & \\
Conversion analogique-numérique & \\
%emmagasinage d'énergie et introduction de délais avec capaciteur et inducteurs
%polarisation de la direction du courant continu avec une diode
%il y en avait un autre auquel j'ai pensé pendant le lancement qui serait utile projet de conception, mais je ne m'en rappelle plus :(
\hline
%\end{tabular}
\caption{Révision des cours et expériences pour rassembler des fonctions qui peuvent être accomplies par un circuit électronique.}
\label{tab:electronic_fcts}
\end{longtable}
\end{center}

\end{document}