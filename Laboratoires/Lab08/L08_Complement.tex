%Dernière modification JG: 11 novembre 2020
\RequirePackage[l2tabu, orthodox]{nag} %Check for obsolete commands
\documentclass[canadien,12pt,oneside,letterpaper]{article}
%
%-----------------------------------------------------
%Loading packages
%
\usepackage[utf8]{inputenc}
\usepackage[T1]{fontenc}
\usepackage[canadien]{babel}
\usepackage{lmodern}
\usepackage{textcomp}
\usepackage{amsmath,amssymb}
\usepackage{siunitx}
\usepackage{xcolor}
\usepackage{hyperref}
\usepackage[all]{hypcap}
\usepackage{graphicx}
\usepackage[americanvoltages,americancurrents,siunitx]{circuitikz}
\usetikzlibrary{babel}
\usepackage{caption}
\usepackage[letterpaper,headheight=15pt]{geometry}
\usepackage{fancyhdr}
\usepackage{setspace}
%
%----------------------------------------------------
%Other configurations and layout
%
\sisetup{separate-uncertainty}
\captionsetup{font=small,labelfont=bf,margin=0.1\textwidth}
\pagestyle{fancy}
\fancyhf{}
\lhead{\textsl{GPH-2006/PHY-2002~---~Laboratoire~VIII}}
\rhead{\textsl{Page \thepage}}
\setcounter{secnumdepth}{0}
\setlength{\parskip}{1.5ex plus0.5ex minus0.2ex}
\setlength\parindent{0pt}
%\onehalfspacing
\interfootnotelinepenalty=10000 %To avoid footnotes spreading on several pages.
%
%---------------------------------------------------
%
\title{\textbf{Complément}\\Oscillateurs électroniques\thanks{Auteurs: Claudine Allen, Jérémie Guilbert \& Jean-Raphaël Carrier}}
\renewcommand\footnotemark{}
\date{}

\begin{document}

\section{Oscillateur à relaxation}
%Ancienne légende de la figure 1
Le condensateur qui se charge cause éventuellement l'entrée inverseuse de l'amplificateur opérationnel à dépasser celle non inverseuse et la tension de sortie de ce comparateur tombe alors négative, renversant ainsi la polarité appliquée sur le condensateur. Celui-ci maintenant se décharge jusqu'à ce l'entrée inverseuse repasse sous celle non inverseuse et remet la sortie positive pour recommencer le cycle.

\section{Oscillateur harmonique à rétroaction (pont de Vien)}
%Ancienne sous-section de la partie 2
Le signal à l'entrée non inverseuse (\textit{a priori} ce signal n'est que du bruit) est amplifié, puis il passe au travers d'un pont de Wien qui le filtre, avant de retourner à l'entrée de l'amplificateur, pour être à nouveau amplifié et filtré, et ainsi de suite, \textit{ad infinitum}. Le pont de Wien est \textit{grosso modo} l'addition d'un filtre passe-bas et d'un filtre passe-haut, pour former un filtre passe-bande. La fréquence du signal de sortie est $f=\left(2\,\pi\,R\,C\right)^{-1}$. L'amplitude finale du signal est fixée par la source d'alimentation. % Cet oscillateur doit son oscillation au chargement/déchargement du condensateur qui relie l'entrée non inverseuse de l'amplificateur au \textit{ground}.

\section{Oscillateur harmonique à rétroaction (LC)}
%Ancienne intro partie 3
Un autre oscillateur plus simple reposant sur le même principe de filtrage et amplification en boucle du signal est l'oscillateur LC. Vous avez étudié au dernier laboratoire le filtre RLC passe-bande dont la fréquence de résonance correspond à une oscillation, tel qu'attendu pour tout système linéaire qui se modélise par une équation différentielle ordinaire d'ordre 2. Ce filtre peut donc jouer le rôle du pont de Wien. 
\section{Oscillations avec la puce 555}
%Ancienne sous-section
\begin{figure}[h]
\centering
\begin{circuitikz} \draw[thick]
(0,0) to[short] (0,0.5) node[right]{4} to[short,*-*] (0,1.5) node[right]{3} to[short] (0,2.5) node[right]{2} to[short,*-*] (0,3.5) node[right]{1} to[short] (0,4) to[short] (3,4) to[short] (3,3.5) node[left]{8} to[short,*-*] (3,2.5) node[left]{7} to[short] (3,1.5) node[left]{6} to[short,*-*] (3,0.5) node[left]{5} to[short] (3,0) to[short] (0,0)
{[anchor=east] (0,3.5) node{\textbf{GND}} (0,2.5) node{\textbf{TRIG}} (0,1.5) node{\textbf{OUT}} (0,0.5) node{\textbf{RESET}}}
{[anchor=west] (3,3.5) node{\textbf{V$_{\text{CC}}$}} (3,2.5) node{\textbf{DISCH}} (3,1.5) node{\textbf{THRES}} (3,0.5) node{\textbf{CONT}}}
;\end{circuitikz}
\caption{\label{sch-555}Schéma d'une puce 555.}
\end{figure}
La puce 555 est un circuit intégré qui est très souvent utilisé en électronique pour, entre autres, bâtir un oscillateur.Les niveaux de déclenchement (\textit{trigger}) et de seuil (\textit{threshold}) valent un tiers et deux tiers de l'alimentation, respectivement. Lorsque l'entrée \textbf{TRIG} est inférieure au niveau de déclenchement, la sortie de la puce (\textbf{OUT}) vaut \textbf{V$_{\text{CC}}$}. Lorsque l'entrée \textbf{TRIG} est supérieure au niveau de déclenchement et qu'en plus l'entrée \textbf{THRES} est supérieure au niveau de seuil, alors la sortie devient nulle. Lorsque la sortie devient nulle, un court-circuit se fait entre l'entrée \textbf{DISCH} (\textit{discharge}) et la mise à la terre. Mais vous comprendrez mieux en voyant un circuit complet.
\begin{table}[h]
\centering
\begin{tabular}{|c|c|c|}
\hline
\textbf{TRIG} & \textbf{THRES} & \textbf{OUT} \\
\hline
$<\frac{1}{3}$\textbf{V$_{\text{CC}}$} & --- & \textbf{V$_{\text{CC}}$} \\
\hline
$>\frac{1}{3}$\textbf{V$_{\text{CC}}$} & $>\frac{2}{3}$\textbf{V$_{\text{CC}}$} & 0 \\
\hline
$>\frac{1}{3}$\textbf{V$_{\text{CC}}$} & $<\frac{2}{3}$\textbf{V$_{\text{CC}}$} & Conserve l'état \\
\hline
\end{tabular}
\caption{\label{table-555}Sortie d'une puce 555 en fonction des tensions aux entrées \textbf{TRIG} et \textbf{THRES}.}
\end{table}
\end{document}